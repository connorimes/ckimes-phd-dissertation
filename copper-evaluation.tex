\section{Evaluation}
\label{sec:copper-evaluation}

This section evaluates CoPPer.
We first show that CoPPer achieves similar error and higher energy efficiency than both a simple and a sophisticated DVFS controller.
Next, we show that CoPPer improves energy efficiency for memory-bound applications and that its gain limit avoids over-allocating power when performance targets are not achievable.
We then show the advantages of using an adaptive controller by demonstrating its behavior for an application with a phased input and in response to interference caused by multiple concurrent applications.
Finally, we quantify CoPPer's runtime overhead.


\subsection{Efficiently Meeting Performance Goals}
\label{sec:copper-eval-perf}

We begin by quantifying CoPPer's ability to achieve high energy efficiency while meeting soft performance goals.
We use \emph{gain limits} of $0.0$ (disabled) and $0.5$.
For this analysis, we consider the steady-state behavior of the controllers.
Therefore, each controller is initialized with the same $s(t)$ value for $t=0$ (see \eqnref{copper-speedup-control} in \secref{copper-framework-formulation}).

For each application, we define and evaluate three different performance goals which specify how much to favor performance over energy consumption: $\mathsf{high}$, $\mathsf{medium}$, and $\mathsf{low}$.
We define the $\mathsf{high}$ performance goal to mean that the application must maintain at least 90\% of top performance.
The $\mathsf{medium}$ and $\mathsf{low}$ goals correspond to maintaining 70\% and 50\% of top performance respectively.
While it is likely that most users would want $\mathsf{high}$ performance, we include the others to demonstrate CoPPer's ability to meet a range of different goals.
We note that actual performance values provided to CoPPer are application-specific, as described in \secref{copper-implementation} (\ie not a percentage).

We quantify the ability to meet performance goals with low energy with two metrics:
\begin{itemize}
\item \emph{Energy efficiency} is the ratio of the ideal \emph{performance} governor's energy consumption (as computed by the oracle) to the actual energy consumption achieved.  
\item \emph{Mean Absolute Percentage Error} (MAPE) quantifies the error between the desired performance and the achieved performance; it is a standard metric for evaluating control systems~\cite{ICSE2014}.  
\end{itemize}
MAPE computes the performance error for an application with $n$ jobs and a performance goal of $R_{ref}$ as:
\begin{equation}
\text{MAPE} = 100\% \cdot \frac{1}{n} \sum\limits_{i=1}^{n}
\left \{
\begin{array}{ll}
r_m(i) < R_{ref}  :& \frac{R_{ref} - r_m(i)}{R_{ref}} \\
r_m(i) \ge R_{ref}  :& 0
\end{array} \right.
\end{equation}
where $r_m(i)$ is the achieved performance for the $i$-th job.
Each failure to achieve the performance target increases MAPE by an amount relative to how badly the target was missed.

\begin{figure}[t]
  \centering
  \begin{tikzpicture}
\definecolor{s1}{RGB}{228, 26, 28}
\definecolor{s2}{RGB}{55, 126, 184}
\definecolor{s3}{RGB}{77, 175, 74}
\definecolor{s4}{RGB}{152, 78, 163}
\definecolor{s5}{RGB}{255, 127, 0}

\begin{groupplot}[
    group style={
        group name=plots,
        group size=1 by 3,
        xlabels at=edge top,
        xticklabels at=edge top,
        vertical sep=15pt
    },
axis x line* = top,
xlabel near ticks,
major x tick style = transparent,
height=3.0cm,
width=0.88\textwidth,
xmin=0,
xmax=9,
enlargelimits=false,
tick align = outside,
tick style={white},
ylabel style={align=center},
ytick=\empty,
xtick=\empty,
xticklabels={},
yticklabels={},
ymin=0,
ymax=1,
]

\nextgroupplot[
ylabel shift={12mm},
y label style={rotate=270},
ylabel={$\mathsf{high}$},
]
\addplot[thick,solid, color=black] coordinates {(0,1) (22,1)};


\nextgroupplot[
ylabel shift={12mm},
ylabel={$\mathsf{medium}$},
y label style={rotate=270},
]
\addplot[thick,solid, color=black] coordinates {(0,1) (22,1)};

\nextgroupplot[
ylabel shift={12mm},
ylabel={$\mathsf{low}$},
y label style={rotate=270},
]
\addplot[thick,solid, color=black] coordinates {(0,1) (22,1)};

\end{groupplot}

\begin{groupplot}[
    group style={
        group name=plots,
        group size=1 by 3,
        xlabels at=edge bottom,
        xticklabels at=edge bottom,
        vertical sep=15pt
    },
axis x line* = bottom,
xlabel near ticks,
major x tick style = transparent,
xlabel={},
height=3.0cm,
width=0.88\textwidth,
xmin=0,
xmax=22,
enlargelimits=false,
tick align = outside,
tick style={white},
ylabel style={align=center},
ytick=\empty,
ymin=0.8,
ymax=1.5,
ytick={0,0.2,0.4,0.6,0.8,1.0,1.2,1.4,1.6,1.8,2.0},
yticklabels={,,,,0.8,1.0,1.2,1.4,1.6,1.8,2.0},
legend cell align=left, 
legend style={ column sep=1ex },
ymajorgrids,
grid style={dashed},
]

\nextgroupplot[ybar=\pgflinewidth,
legend entries = {DVFS-Simple,DVFS-Sophisticated,CoPPer-0.0,CoPPer-0.5},
legend style={draw=none,legend columns=4,at={(.5,1.5)},anchor=north,font=\footnotesize},
ylabel shift={0mm},
bar width=2.4pt,
%ymin=.9,
ymax=1.6,
%ytick={1,2,3,4,5,6,7,8},
%yticklabels={1.0,,,,5.0,,,8.0},
]
\addplot table[x index=0,y index=6, col sep=space] {img/copper/ee/clover/ee-dvfs-simple.txt};
\addplot table[x index=0,y index=6, col sep=space] {img/copper/ee/clover/ee-dvfs.txt};
\addplot table[x index=0,y index=6, col sep=space] {img/copper/ee/clover/ee-copper-0.0.txt};
%\addplot table[x index=0,y index=6, col sep=space] {img/copper/ee/clover/ee-copper-0.10.txt};
%\addplot table[x index=0,y index=6, col sep=space] {img/copper/ee/clover/ee-copper-0.20.txt};
\addplot table[x index=0,y index=6, col sep=space] {img/copper/ee/clover/ee-copper-0.50.txt};
\node at (axis cs:13.0,0.925) [rotate=0] {\scriptsize{$\approx0.4$}};


\nextgroupplot[ybar=\pgflinewidth,
ylabel shift={0mm},
bar width=2.4pt,
ylabel={\footnotesize Energy Efficiency (Normalized)},
]
\addplot table[x index=0,y index=5, col sep=space] {img/copper/ee/clover/ee-dvfs-simple.txt};
\addplot table[x index=0,y index=5, col sep=space] {img/copper/ee/clover/ee-dvfs.txt};
\addplot table[x index=0,y index=5, col sep=space] {img/copper/ee/clover/ee-copper-0.0.txt};
%\addplot table[x index=0,y index=5, col sep=space] {img/copper/ee/clover/ee-copper-0.10.txt};
%\addplot table[x index=0,y index=5, col sep=space] {img/copper/ee/clover/ee-copper-0.20.txt};
\addplot table[x index=0,y index=5, col sep=space] {img/copper/ee/clover/ee-copper-0.50.txt};
\node at (axis cs:13.0,0.925) [rotate=0] {\scriptsize{$\approx0.4$}};


\nextgroupplot[ybar=\pgflinewidth,
bar width=2.4pt,
ylabel shift={0mm},
xticklabel shift={0pt},
x tick label style={rotate=35, anchor=east, font=\scriptsize},
xtick={1,2,3,4,5,6,7,8,9,10,11,12,13,14,15,16,17,18,19,20,21},
xticklabels={
{blackscholes},
{bodytrack},
{facesim},
{ferret},
{fluidanimate},
{frequmine},
{raytrace},
{swaptions},
{vips},
{x264},
{canneal},
{dedup},
{streamcluster},
{STREAM},
{SWISH++},
{HOP},
{KMeans},
{KMeans-Fuzzy},
{ScalParC},
{SVM-RFE},
{\textbf{Average}}},
]
\addplot table[x index=0,y index=4, col sep=space] {img/copper/ee/clover/ee-dvfs-simple.txt};
\addplot table[x index=0,y index=4, col sep=space] {img/copper/ee/clover/ee-dvfs.txt};
\addplot table[x index=0,y index=4, col sep=space] {img/copper/ee/clover/ee-copper-0.0.txt};
%\addplot table[x index=0,y index=4, col sep=space] {img/copper/ee/clover/ee-copper-0.10.txt};
%\addplot table[x index=0,y index=4, col sep=space] {img/copper/ee/clover/ee-copper-0.20.txt};
\addplot table[x index=0,y index=4, col sep=space] {img/copper/ee/clover/ee-copper-0.50.txt};
\node at (axis cs:13.0,0.925) [rotate=0] {\scriptsize{$\approx0.4$}};

\end{groupplot}
\end{tikzpicture}

  % \vskip -1em
  \caption{Application energy efficiency for DVFS controllers and CoPPer, with and without a gain limit, for $\mathsf{high}$, $\mathsf{medium}$, and $\mathsf{low}$ performance targets (higher is better).
  Results are normalized to an ideal \emph{performance} DVFS governor.}
  \label{fig:copper-ee}
\end{figure}

\begin{figure}[t]
  \centering
  \begin{tikzpicture}
\definecolor{s1}{RGB}{228, 26, 28}
\definecolor{s2}{RGB}{55, 126, 184}
\definecolor{s3}{RGB}{77, 175, 74}
\definecolor{s4}{RGB}{152, 78, 163}
\definecolor{s5}{RGB}{255, 127, 0}

\begin{groupplot}[
    group style={
        group name=plots,
        group size=1 by 3,
        xlabels at=edge top,
        xticklabels at=edge top,
        vertical sep=15pt
    },
axis x line* = top,
xlabel near ticks,
major x tick style = transparent,
height=3.0cm,
width=0.88\textwidth,
xmin=0,
xmax=9,
enlargelimits=false,
tick align = outside,
tick style={white},
ylabel style={align=center},
ytick=\empty,
xtick=\empty,
xticklabels={},
yticklabels={},
ymin=0,
ymax=1,
]

\nextgroupplot[
ylabel shift={12mm},
y label style={rotate=270},
ylabel={$\mathsf{high}$},
]
\addplot[thick,solid, color=black] coordinates {(0,1) (22,1)};


\nextgroupplot[
ylabel shift={12mm},
ylabel={$\mathsf{medium}$},
y label style={rotate=270},
]
\addplot[thick,solid, color=black] coordinates {(0,1) (22,1)};

\nextgroupplot[
ylabel shift={12mm},
ylabel={$\mathsf{low}$},
y label style={rotate=270},
]
\addplot[thick,solid, color=black] coordinates {(0,1) (22,1)};

\end{groupplot}

\begin{groupplot}[
    group style={
        group name=plots,
        group size=1 by 4,
        xlabels at=edge bottom,
        xticklabels at=edge bottom,
        vertical sep=15pt
    },
axis x line* = bottom,
xlabel near ticks,
major x tick style = transparent,
xlabel={},
height=3.0cm,
width=0.88\textwidth,
xmin=0,
xmax=22,
enlargelimits=false,
tick align = outside,
tick style={white},
ylabel style={align=center},
ytick=\empty,
ymin=0,
ymax=20,
ytick={0,5,10,15,20},
yticklabels={0,5,10,15,20},
legend cell align=left, 
legend style={ column sep=1ex },
ymajorgrids,
grid style={dashed},
]

\nextgroupplot[ybar=\pgflinewidth,
ylabel shift={0mm},
bar width=2.4pt,
legend entries = {DVFS-Simple,DVFS-Sophisticated,CoPPer-0.0,CoPPer-0.5},
legend style={draw=none,legend columns=4,at={(.5,1.5)},anchor=north,font=\footnotesize},
]
\addplot table[x index=0,y index=6, col sep=space] {img/copper/mape/clover/mape-dvfs-simple.txt};
\addplot table[x index=0,y index=6, col sep=space] {img/copper/mape/clover/mape-dvfs.txt};
\addplot table[x index=0,y index=6, col sep=space] {img/copper/mape/clover/mape-copper-0.0.txt};
%\addplot table[x index=0,y index=6, col sep=space] {img/copper/mape/clover/mape-copper-0.10.txt};
%\addplot table[x index=0,y index=6, col sep=space] {img/copper/mape/clover/mape-copper-0.20.txt};
\addplot table[x index=0,y index=6, col sep=space] {img/copper/mape/clover/mape-copper-0.50.txt};


\nextgroupplot[ybar=\pgflinewidth,
ylabel shift={0mm},
bar width=2.4pt,
ylabel={\footnotesize MAPE (\%)},
]
\addplot table[x index=0,y index=5, col sep=space] {img/copper/mape/clover/mape-dvfs-simple.txt};
\addplot table[x index=0,y index=5, col sep=space] {img/copper/mape/clover/mape-dvfs.txt};
\addplot table[x index=0,y index=5, col sep=space] {img/copper/mape/clover/mape-copper-0.0.txt};
%\addplot table[x index=0,y index=5, col sep=space] {img/copper/mape/clover/mape-copper-0.10.txt};
%\addplot table[x index=0,y index=5, col sep=space] {img/copper/mape/clover/mape-copper-0.20.txt};
\addplot table[x index=0,y index=5, col sep=space] {img/copper/mape/clover/mape-copper-0.50.txt};


\nextgroupplot[ybar=\pgflinewidth,
bar width=2.4pt,
ylabel shift={0mm},
xticklabel shift={0pt},
x tick label style={rotate=35, anchor=east, font=\scriptsize},
xtick={1,2,3,4,5,6,7,8,9,10,11,12,13,14,15,16,17,18,19,20,21},
xticklabels={
{blackscholes},
{bodytrack},
{facesim},
{ferret},
{fluidanimate},
{frequmine},
{raytrace},
{swaptions},
{vips},
{x264},
{canneal},
{dedup},
{streamcluster},
{STREAM},
{SWISH++},
{HOP},
{KMeans},
{KMeans-Fuzzy},
{ScalParC},
{SVM-RFE},
{\textbf{Average}}},
]
\addplot table[x index=0,y index=4, col sep=space] {img/copper/mape/clover/mape-dvfs-simple.txt};
\addplot table[x index=0,y index=4, col sep=space] {img/copper/mape/clover/mape-dvfs.txt};
\addplot table[x index=0,y index=4, col sep=space] {img/copper/mape/clover/mape-copper-0.0.txt};
%\addplot table[x index=0,y index=4, col sep=space] {img/copper/mape/clover/mape-copper-0.10.txt};
%\addplot table[x index=0,y index=4, col sep=space] {img/copper/mape/clover/mape-copper-0.20.txt};
\addplot table[x index=0,y index=4, col sep=space] {img/copper/mape/clover/mape-copper-0.50.txt};
\end{groupplot}
\end{tikzpicture}

  % \vskip -1em
  \caption{Application performance error for DVFS controllers and CoPPer, with and without a gain limit, for for $\mathsf{high}$, $\mathsf{medium}$, and $\mathsf{low}$ performance targets (lower is better).}
  \label{fig:copper-mape}
\end{figure}

\figsref{copper-ee}{copper-mape} present the energy efficiency and MAPE values for all applications and targets.
Despite the challenges described in \secref{copper-motivation-power} (\eg non-linearity and larger range in the power cap/performance relationship), CoPPer achieves higher energy efficiency and similar MAPE compared to both the simple and sophisticated DVFS controllers for most applications and performance targets.
Compared to the sophisticated DVFS controller, CoPPer is on average 3\% more energy-efficient with no gain limit and a 6\% more efficient with gain limit 0.5.

\begin{table}[t]
\small
\centering
\caption{Energy efficiency of CoPPer with gain limits of 0.0 and 0.5 compared to the sophisticated DVFS controller (higher is better).}
\begin{tabular}{ccc}
  &\multicolumn{2}{c}{\textbf{Energy Efficiency vs DVFS}} \\
  \textbf{Performance} & \textbf{CoPPer-0.0} & \textbf{CoPPer-0.5} \\
  \hline
  \hline
  $\mathsf{high}$   & 1.05  & 1.08 \\
  $\mathsf{medium}$  & 1.03  & 1.06 \\
  $\mathsf{low}$   & 1.02  & 1.04 \\
  \textbf{Average}& \textbf{1.03}  & \textbf{1.06} \\
  \hline
  \hline
\end{tabular}
\label{tbl:copper-summary}
\end{table}

\tblref{copper-summary} shows the average energy efficiency gains of CoPPer compared to the sophisticated DVFS controller for different performance goals.
Note that CoPPer's energy efficiency gains increase as the performance goal increases.
For high goals, the DVFS approach must use TurboBoost, which is inefficient.
CoPPer, however, can set a power cap that allows the performance goal to be met and leave the decision to Turbo or not to hardware, which has more information about whether that choice is appropriate---exactly the motivation to move DVFS control to hardware and allow software to cap power instead.

\begin{figure}[t]
  \centering
  % \vskip -1.8em
  \subfloat[DVFS.]%
  {\begin{tikzpicture}
\begin{centering}


\definecolor{s1}{RGB}{228, 26, 28}
\definecolor{s2}{RGB}{55, 126, 184}
\definecolor{s3}{RGB}{77, 175, 74}
\definecolor{s4}{RGB}{152, 78, 163}
\definecolor{s5}{RGB}{255, 127, 0}

\begin{groupplot}[
    group style={
        group name=plots,
        group size=1 by 1,
        xlabels at=edge bottom,
        xticklabels at=edge bottom,
        vertical sep=5pt
    },
xlabel={\footnotesize $Frequency~(Normalized)$ 
%(normalized to worst case)
},
xlabel near ticks,
height=4cm,
%width=0.5\textwidth,
width = 4.5cm,
xmajorgrids,
ymajorgrids,
grid style={dashed},
% xtick={1.2,1.6,2.0,2.4,2.8,3.3,3.8},
% xticklabels={1.2,1.6,2.0,2.4,2.8,${-}$,3.8},
xtick={1,1.5,2,2.5,3,3.5,4,4.5},
xticklabels={1,1.5,2,2.5,3,3.5,4,4.5},
xticklabel style={font=\footnotesize},
xmin=0.75,
xmax=3,
yticklabel pos=left,
enlargelimits=false,
tick align = outside,
tick style={white},
xticklabel shift={-5pt},
yticklabel shift={-5pt},
ylabel shift={-2pt},
ylabel style={align=center},
unbounded coords=jump,
]

\nextgroupplot[ylabel={\footnotesize Speedup}, 
%ylabel shift={6mm},
ytick={0,0.5,1.0,1.5,2.0,2.5,3.0},
yticklabels={0,0.5,1.0,1.5,2.0,2.5,3.0},
yticklabel style={font=\footnotesize},
ymin=0.75,
ymax=3.0,
% legend entries={{\footnotesize $\mathsf{streamcluster}~$}},
% legend style={draw=none,at={(0.5,1.3)},anchor=north,legend columns=4,line width=5pt},
]
\addplot[thick, solid, color=s1, only marks, mark=square*, mark size=1.2] table[x index=6,y index=4,col sep=comma] {img/copper/tradeoffs/tradeoffs-streamcluster-dvfs.csv};


\end{groupplot}
\end{centering}

\end{tikzpicture}
%
  \label{fig:copper-tradeoffs-streamcluster-dvfs}}
  \hspace*{0.5cm}
  \subfloat[Power cap.]%
  {\begin{tikzpicture}
\begin{centering}


\definecolor{s1}{RGB}{228, 26, 28}
\definecolor{s2}{RGB}{55, 126, 184}
\definecolor{s3}{RGB}{77, 175, 74}
\definecolor{s4}{RGB}{152, 78, 163}
\definecolor{s5}{RGB}{255, 127, 0}

\begin{groupplot}[
    group style={
        group name=plots,
        group size=1 by 1,
        xlabels at=edge bottom,
        xticklabels at=edge bottom,
        vertical sep=5pt
    },
xlabel={\footnotesize $Power~Cap~(Normalized)$ 
%(normalized to worst case)
},
xlabel near ticks,
height=4cm,
%width=0.5\textwidth,
width = 4.5cm,
xmajorgrids,
ymajorgrids,
grid style={dashed},
xtick={1,1.5,2,2.5,3,3.5,4,4.5,5,5.5,6,6.5},
xticklabels={1,,2,,3,,4,,5,,6},
xticklabel style={font=\footnotesize},
xmin=0.5,
xmax=6.5,
yticklabel pos=left,
enlargelimits=false,
tick align = outside,
tick style={white},
xticklabel shift={-5pt},
yticklabel shift={-5pt},
ylabel shift={-2pt},
ylabel style={align=center},
unbounded coords=jump,
]

\nextgroupplot[ylabel={\footnotesize Speedup}, 
%ylabel shift={6mm},
ytick={0,0.5,1.0,1.5,2.0,2.5,3.0},
yticklabels={0,0.5,1.0,1.5,2.0,2.5,3.0},
yticklabel style={font=\footnotesize},
ymin=0.75,
ymax=3.0,
]
\addplot[thick, solid, color=s1, only marks, mark=square*, mark size=1.2] table[x index=6,y index=4,col sep=comma] {img/copper/tradeoffs/tradeoffs-streamcluster-pwr.csv};


\end{groupplot}
\end{centering}

\end{tikzpicture}
%
  \label{fig:copper-tradeoffs-streamcluster-pwr}}
  \caption{DVFS and power cap performance impacts for \app{streamcluster}.}
  \label{fig:copper-tradeoffs-streamcluster}
  \vskip -1em
\end{figure}

\app{Freqmine} and \app{streamcluster} are outliers for both energy efficiency and MAPE.
\app{Freqmine} is composed of repeated jobs, but its behavior is not predictable with feedback---the application uses a recursive algorithm that causes job performance to continually slow as it progresses.
This behavior is quantified by its high job variability as shown in \figref{copper-variation} in \secref{copper-inputs}.
\app{Streamcluster} exhibits a performance/power tradeoff space that does not scale well beyond a fairly low DVFS setting or power cap, as demonstrated in \figref{copper-tradeoffs-streamcluster}.
In fact, its performance degrades dramatically as resource allocation increases.
Even CoPPer's gain limit cannot adapt since it detects a change in performance when trying to reduce the power cap.
The ideal \emph{performance} DVFS governor (the oracle) knows not to allocate higher frequencies since it has access to the application-specific characterizations, but our runtime controllers do not have this information.
If we were to specify \app{streamcluster}-specific power cap ranges for CoPPer, the results would be similar to the other applications; the DVFS controllers, however, would need entirely new models.


\subsection{Controlling Memory-bound Applications}

Our benchmark set contains 6 memory-bound applications: \app{KMeans}, \app{KMeans-Fuzzy}, \app{ScalParC}, \app{STREAM}, \app{streamcluster}, and \app{SVM-RFE}.
CoPPer achieves noticeably higher energy efficiency for these applications than with DVFS.
\tblref{copper-mem} summarizes the average ratio of energy efficiencies across all performance targets for these, comparing CoPPer with and without a gain limit to the sophisticated DVFS controller.

\begin{table}[t]
\small
\centering
\caption{Energy efficiency of CoPPer with gain limits of 0.0 and 0.5 compared to the sophisticated DVFS controller for memory-bound applications (higher is better).}
\begin{tabular}{ccc}
  &\multicolumn{2}{c}{\textbf{Energy Efficiency vs DVFS}} \\
  \textbf{Performance} & \textbf{CoPPer-0.0} & \textbf{CoPPer-0.5} \\
  \hline
  \hline
    $\mathsf{high}$  & 1.15  & 1.18 \\
   $\mathsf{medium}$   & 1.09  & 1.11 \\
   $\mathsf{low}$   & 1.06  & 1.08 \\
  \textbf{Average}& \textbf{1.10}  & \textbf{1.12} \\
  \hline
  \hline
\end{tabular}
\label{tbl:copper-mem}
\end{table}

We see that even without a gain limit, CoPPer already improves on the sophisticated DVFS controller's energy efficiency by 10\% on average.
With a gain limit of 0.5, the improvement rises to 12\%.
These are significant energy savings.
CoPPer performs especially well compared to DVFS with these memory-bound applications for the higher performance targets.
Again, even without a gain limit, CoPPer improves energy efficiency by 15\% for the $\mathsf{high}$ performance target.
DVFS can benefit from TurboBoost at high performance targets for many applications, but the higher DVFS frequencies also result in unnecessarily high energy consumption for memory-bound applications.
By setting power caps instead of forcing DVFS frequencies, CoPPer achieves better energy savings by allowing the processor to scale frequencies more quickly between computational and memory-intensive periods.
The gain limit provides significant energy savings for memory-bound applications with only a small loss in performance.
In general, we advocate the use of 0.5 gain limit in practice since it produces almost no difference in MAPE, but can provide significant energy savings for memory-bound applications.


\subsection{Reducing Power for Unachievable Goals}
\label{sec:copper-eval-impossible}

Sometimes performance targets simply are not achievable.
This could be due to a user requesting too much from an application given the available processing capability, or the application may just want to run as fast as possible.
When a performance target is unachievable, a naive resource controller will continue to increase resource allocations like DVFS frequencies or power caps in an attempt to improve performance, needlessly wasting energy.
In this part of the evaluation, we demonstrate that CoPPer's \emph{gain limit} helps avoid this pitfall.
In \secref{copper-framework-gain}, we explained that the gain limit's effectiveness is constrained by the accuracy of the minimum and maximum power values.
For this experiment, we use a more reasonable (and safer) maximum power limit, the evaluation system's TDP of 270 W.

\begin{table}[t]
% \vskip -1.5em
\small
\centering
\caption{Average energy efficiency for unachievable performance targets, normalized to the sophisticated DVFS controller.}
\begin{tabular}{cc}
  \textbf{Controller} & \textbf{Energy Efficiency} \\
  \hline
  \hline
  DVFS-Sophisticated &  1.00 \\
  CoPPer-0.0         &  1.00  \\
  % CoPPer-0.1         &  1.00  \\
  CoPPer-0.2         &  1.01 \\
  % CoPPer-0.3         &  1.02 & 0.98 \\
  % CoPPer-0.4         &  1.05 & 0.99 \\
  CoPPer-0.5         &  1.10  \\
  CoPPer-0.6         &  1.16  \\
  % CoPPer-0.7         &  1.22 \\
  CoPPer-0.8         &  1.29  \\
  % CoPPer-0.9         &  1.36  \\
  CoPPer-0.99        &  1.46  \\
  % Linux-ondemand     &  1.33 & 1.11 \\
  % Linux-performance  &  1.00 & 1.00 \\
  \hline
  \hline
\end{tabular}
\label{tbl:copper-impossible}
\end{table}

For each application, we set an unrealistically high performance target---$1000\times$ greater than what the system can actually achieve.
We then execute both the sophisticated DVFS controller and CoPPer with a range of gain limit values.
As the performance target is not actually achievable, MAPE is meaningless.
Instead, we normalize energy efficiency to the sophisticated DVFS controller.
\tblref{copper-impossible} presents the results for select gain limits.

The DVFS controller runs in the TurboBoost setting for the entirety of each execution.
We also verified that the simple DVFS controller and the evaluation system's real Linux performance governor achieved nearly identical results as the sophisticated controller.
As should be expected, CoPPer without a gain limit behaves similarly.
With gain limits enabled, CoPPer achieves increasingly better energy efficiency for small increases in application runtime.
A $0.5$ gain limit demonstrates a significant improvement in energy efficiency---a 10\% increase over the sophisticated DVFS controller.
A gain limit of 0.99 increases energy efficiency by 46\% over the DVFS controller, though suffers a 20\% loss in performance as it pulls power consumption back too far.

%The performance loss can be attributed to using such a high performance target.
%In \eqnrref{cost-pole}{dens}, $e_{n}(t) \approx 1$, making $e_{ns}(t) \approx 0.5$, and $\Delta e_{ns}(t)$ is never 0 in practice due to application performance variability (again, see \figref{variation} in \secref{inputs}).
%The gain limit therefore results in some performance loss on real systems with real applications, particularly for higher values.
%The performance losses demonstrated here can reasonably be considered near-worst case.
%If an application really does desire to run at maximum performance, it should set a more appropriate maximum goal than $1000\times$ to reduce the impact of $e_{ns}(t)$ on $gain(t)$.

These results clearly demonstrate the gain limit's advantages.
For achievable performance targets, it has minimal impact on controller behavior.
For unachievable targets, it can greatly improve energy efficiency over the sophisticated DVFS controller.


\subsection{Adapting to Runtime Changes}
\label{sec:copper-eval-runtime}

\begin{figure}[t]
  \centering
  % \vskip -1.8em
  \subfloat[Uncontrolled behavior.]%
  {\begin{tikzpicture}
\begin{centering}

\definecolor{s1}{RGB}{228, 26, 28}
\definecolor{s2}{RGB}{55, 126, 184}
\definecolor{s3}{RGB}{77, 175, 74}
\definecolor{s4}{RGB}{152, 78, 163}
\definecolor{s5}{RGB}{255, 127, 0}

\begin{groupplot}[
    group style={
        group name=plots,
        group size=1 by 2,
        xlabels at=edge bottom,
        xticklabels at=edge bottom,
        vertical sep=5pt
    },
height=3.5cm,
width=0.95\columnwidth,
xmajorgrids,
ymajorgrids,
grid style={dashed},
xmin=0,
xmax=4500,
yticklabel pos=left,
enlargelimits=false,
tick align = outside,
tick style={white},
xticklabel shift={-5pt},
yticklabel shift={-5pt},
ylabel shift={-2pt},
ylabel style={align=center},
unbounded coords=jump,
]

\nextgroupplot[ylabel={\footnotesize Performance \\ (Normalized)}, % Performance
xtick={0,500,1000,1500,2000,2500,3000,3500,4000,4500},
ytick={0,0.2,0.4,0.6,0.8,1.0,1.2},
yticklabels={,0.2,0.4,0.6,0.8,1.0,1.2},
yticklabel style={font=\footnotesize},
ymin=0,
ymax=1.0,
% legend entries={{\footnotesize $\mathsf{Server}$}},
% legend style={draw=none,at={(0.5,1.35)},anchor=north,legend columns=4,line width=5pt},
]
\addplot[thick, solid, color=s2] table[x index=0,y index=1,col sep=tab] {img/copper/phases/x264-phases-default-clover.txt};
\addplot[thick, dashed, black] coordinates {(1500,0) (1500, 2)};
\addplot[thick, dashed, black] coordinates {(3000,0) (3000, 2)};


\nextgroupplot[ylabel={\footnotesize Power \\ (Watts)}, % Power
ytick={0,50,100,150,200,250},
yticklabels={,50,100,150,200,250},
yticklabel style={font=\footnotesize},
ymin=50,
ymax=250,
xlabel={$time$ [frame]},
xlabel near ticks,
xtick={0,500,1000,1500,2000,2500,3000,3500,4000,4500},
xticklabels={0,,,1500,,,3000,,,4500},
xticklabel style={font=\footnotesize},
]
\addplot[thick, solid, color=s1] table[x index=0,y index=2,col sep=tab] {img/copper/phases/x264-phases-default-clover.txt};
\addplot[thick, dashed, black] coordinates {(1500,0) (1500, 250)};
\addplot[thick, dashed, black] coordinates {(3000,0) (3000, 250)};

\end{groupplot}
\end{centering}

\end{tikzpicture}%
  \label{fig:copper-phases-default}}
  \newline
  \subfloat[Meeting a performance target with CoPPer.]%
  {\begin{tikzpicture}
\begin{centering}

\definecolor{s1}{RGB}{228, 26, 28}
\definecolor{s2}{RGB}{55, 126, 184}
\definecolor{s3}{RGB}{77, 175, 74}
\definecolor{s4}{RGB}{152, 78, 163}
\definecolor{s5}{RGB}{255, 127, 0}

\begin{groupplot}[
    group style={
        group name=plots,
        group size=1 by 2,
        xlabels at=edge bottom,
        xticklabels at=edge bottom,
        vertical sep=5pt
    },
height=3.5cm,
width=0.95\columnwidth,
xmajorgrids,
ymajorgrids,
grid style={dashed},
xmin=0,
xmax=4500,
yticklabel pos=left,
enlargelimits=false,
tick align = outside,
tick style={white},
xticklabel shift={-5pt},
yticklabel shift={-5pt},
ylabel shift={-2pt},
ylabel style={align=center},
unbounded coords=jump,
]

\nextgroupplot[ylabel={\footnotesize Performance \\ (Normalized)}, % Performance
xtick={0,500,1000,1500,2000,2500,3000,3500,4000,4500},
ytick={0.0,0.5,1.0,1.5,2.0},
yticklabels={,0.5,1.0,1.5,2.0},
yticklabel style={font=\footnotesize},
ymin=0,
ymax=2,
]
\addplot[thick, solid, color=s2] table[x index=0,y index=1,col sep=tab] {img/copper/phases/x264-phases-clover-copper.txt};
\addplot[thick, solid, black] coordinates {(0, 1) (4500, 1)};
\addplot[thick, dashed, black] coordinates {(1500,0) (1500, 2)};
\addplot[thick, dashed, black] coordinates {(3000,0) (3000, 2)};


\nextgroupplot[ylabel={\footnotesize Power \\ (Watts)}, % Power
ytick={0,50,100,150,200,250},
yticklabels={,50,100,150,200,250},
yticklabel style={font=\footnotesize},
ymin=50,
ymax=200,
xlabel={$time$ [frame]},
xlabel near ticks,
xtick={0,500,1000,1500,2000,2500,3000,3500,4000,4500},
xticklabels={0,,,1500,,,3000,,,4500},
xticklabel style={font=\footnotesize},
]
\addplot[thick, solid, color=s1] table[x index=0,y index=2,col sep=tab] {img/copper/phases/x264-phases-clover-copper.txt};
\addplot[thick, dashed, black] coordinates {(1500,0) (1500, 250)};
\addplot[thick, dashed, black] coordinates {(3000,0) (3000, 250)};

\end{groupplot}
\end{centering}

\end{tikzpicture}
%
  \label{fig:copper-phases-x264}}
  \caption{Processing \app{x264} input with distinct phases.}
  \label{fig:copper-phases}
\end{figure}

This experiment demonstrates CoPPer's ability to respond to changes in application behavior at runtime.
We run \app{x264} with a video input that exhibits three distinct levels of encoding difficulty.
\figref{copper-phases-default} demonstrates the uncontrolled behavior of the input, with performance normalized to the maximum achieved.
Dashed vertical lines denote where phase changes occur.
The first phase has the lowest average performance and is therefore the most difficult to encode, followed closely by the third phase.
The second phase has the highest performance, meaning it is the easiest part of the video to encode.
Frames that are easier to encode offer an opportunity to save energy when meeting a performance target, as fewer resources are needed to satisfy the constraint.
In the uncontrolled execution, power is consistently high as no changes to resource allocations are being made.

\figref{copper-phases-x264} shows the time series for CoPPer with a gain limit of 0.0 for the $\mathsf{medium}$ performance target and a window size of 50 frames.
Performance is normalized to the target.
Now the performance remains mostly fixed (per the constraint), whereas the power consumption fluctuates as the power cap changes.
Power values are now inversely proportional to the performance behavior seen in \figref{copper-phases-default} since CoPPer is able to reduce power consumption to save energy during phases of easier encoding.
Of course, the actual power consumption recorded for any given frame does not necessarily match the power cap---it may be lower.

The fluctuations around the performance target in each phase are a result of input variability.
The uncontrolled execution in \figref{copper-phases-default} testifies to the variability's presence in the input.
We see similar behavior in other applications to varying degrees, but this visual clarifies where performance error (MAPE) comes from, and why some applications are difficult to control.
Still, CoPPer meets the performance target with high energy efficiency and low error.
For this execution, energy efficiency is 1.25 compared to the ideal \emph{performance} governor (the oracle) and MAPE is 6.48\%.


\subsection{Multiple Applications}
\label{sec:copper-eval-multiapp}

\begin{figure}[t]
  \centering
  \begin{tikzpicture}
\definecolor{s1}{RGB}{228, 26, 28}
\definecolor{s2}{RGB}{55, 126, 184}
\definecolor{s3}{RGB}{77, 175, 74}
\definecolor{s4}{RGB}{152, 78, 163}
\definecolor{s5}{RGB}{255, 127, 0}

\begin{groupplot}[
    group style={
        group name=plots,
        group size=1 by 1,
        xlabels at=edge top,
        xticklabels at=edge top,
        vertical sep=5pt
    },
axis x line* = top,
xlabel near ticks,
major x tick style = transparent,
height=3.5cm,
width=0.96\textwidth,
xmin=0,
xmax=9,
enlargelimits=false,
tick align = outside,
tick style={white},
ylabel style={align=center},
ytick=\empty,
xtick=\empty,
xticklabels={},
yticklabels={},
ymin=0,
ymax=1,
]
\nextgroupplot[ylabel={},
y label style={rotate=270},
ylabel shift={12mm},
]
\addplot[thick,solid, color=black] coordinates {(0,1) (22,1)};


\end{groupplot}

\begin{groupplot}[
    group style={
        group name=plots,
        group size=1 by 1,
        xlabels at=edge bottom,
        xticklabels at=edge bottom,
        vertical sep=5pt
    },
axis x line* = bottom,
xlabel near ticks,
major x tick style = transparent,
xlabel={},
height=3.5cm,
width=0.96\textwidth,
xmin=0,
xmax=22,
enlargelimits=false,
tick align = outside,
tick style={white},
ylabel style={align=center},
ytick=\empty,
ymin=0.25,
ymax=1.8,
ytick={0,0.25,0.5,0.75,1.0,1.25,1.5,1.75,2.0},
yticklabels={,,0.5,,1.0,,1.5,,2.0},
legend cell align=left, 
legend style={ column sep=1ex },
ymajorgrids,
grid style={dashed},
]


\nextgroupplot[ybar=\pgflinewidth,
ylabel={\footnotesize EE (Norm)},
bar width=2.4pt,
legend entries = {DVFS-Simple,DVFS-Sophisticated,CoPPer-0.0,CoPPer-0.5},
legend style={draw=none,legend columns=4,at={(.5,1.5)},anchor=north,font=\footnotesize},
ylabel shift={0mm},
xticklabel shift={0pt},
x tick label style={rotate=35, anchor=east, font=\scriptsize},
xtick={1,2,3,4,5,6,7,8,9,10,11,12,13,14,15,16,17,18,19,20,21},
xticklabels={
{blackscholes},
{bodytrack},
{facesim},
{ferret},
{fluidanimate},
{frequmine},
{raytrace},
{swaptions},
{vips},
{x264},
{canneal},
{dedup},
{streamcluster},
{STREAM},
{SWISH++},
{HOP},
{KMeans},
{KMeans-Fuzzy},
{ScalParC},
{SVM-RFE},
{\textbf{Average}}},
]
\addplot table[x index=0,y index=2, col sep=space] {img/copper/multi_app/multi_ee-dvfs-simple.txt};
\addplot table[x index=0,y index=2, col sep=space] {img/copper/multi_app/multi_ee-dvfs.txt};
\addplot table[x index=0,y index=2, col sep=space] {img/copper/multi_app/multi_ee-0.0.txt};
%\addplot table[x index=0,y index=2, col sep=space] {img/copper/multi_app/multi_ee-0.1.txt};
%\addplot table[x index=0,y index=2, col sep=space] {img/copper/multi_app/multi_ee-0.2.txt};
\addplot table[x index=0,y index=2, col sep=space] {img/copper/multi_app/multi_ee-0.5.txt};
\node at (axis cs:11.0,0.4) [rotate=0] {\scriptsize{$\approx0.14$}};

\end{groupplot}
\end{tikzpicture}
  \caption{Application energy efficiency for DVFS controllers and CoPPer, with and without a gain limit, under interference by a second application (higher is better). Results are normalized to an ideal \emph{performance} DVFS governor.}
  \label{fig:copper-multi-ee}
\end{figure}

\begin{figure}[t]
  \centering
  \begin{tikzpicture}
\definecolor{s1}{RGB}{228, 26, 28}
\definecolor{s2}{RGB}{55, 126, 184}
\definecolor{s3}{RGB}{77, 175, 74}
\definecolor{s4}{RGB}{152, 78, 163}
\definecolor{s5}{RGB}{255, 127, 0}

\begin{groupplot}[
    group style={
        group name=plots,
        group size=1 by 5,
        xlabels at=edge top,
        xticklabels at=edge top,
        vertical sep=5pt
    },
axis x line* = top,
xlabel near ticks,
major x tick style = transparent,
height=3.0cm,
width=0.96\textwidth,
xmin=0,
xmax=9,
enlargelimits=false,
tick align = outside,
tick style={white},
ylabel style={align=center},
ytick=\empty,
xtick=\empty,
xticklabels={},
yticklabels={},
ymin=0,
ymax=1,
]
\nextgroupplot[ylabel={},
y label style={rotate=270},
ylabel shift={12mm},
]
\addplot[thick,solid, color=black] coordinates {(0,1) (22,1)};

\end{groupplot}

\begin{groupplot}[
    group style={
        group name=plots,
        group size=1 by 5,
        xlabels at=edge bottom,
        xticklabels at=edge bottom,
        vertical sep=5pt
    },
axis x line* = bottom,
xlabel near ticks,
major x tick style = transparent,
xlabel={},
height=3cm,
width=0.96\textwidth,
xmin=0,
xmax=22,
enlargelimits=false,
tick align = outside,
tick style={white},
ylabel style={align=center},
ytick=\empty,
ymin=0,
ymax=50,
ytick={0,10,20,30,40,50,60,70},
yticklabels={0,10,20,30,40,50,60,70},
legend cell align=left, 
legend style={ column sep=1ex },
ymajorgrids,
grid style={dashed},
]


\nextgroupplot[ybar=\pgflinewidth,
ylabel={\footnotesize MAPE (\%)},
bar width=2.4pt,
%legend entries = {{$\mathsf{DVFS}$},{$\mathsf{0.0}$},{$\mathsf{0.10}$},{$\mathsf{0.20}$},{$\mathsf{0.50}$}},
legend entries = {DVFS-Simple,DVFS-Sophisticated,CoPPer-0.0,CoPPer-0.5},
legend style={draw=none,legend columns=4,at={(.5,1.5)},anchor=north,font=\footnotesize},
ylabel shift={0mm},
xticklabel shift={0pt},
x tick label style={rotate=35, anchor=east, font=\scriptsize},
xtick={1,2,3,4,5,6,7,8,9,10,11,12,13,14,15,16,17,18,19,20,21},
xticklabels={
{blackscholes},
{bodytrack},
{facesim},
{ferret},
{fluidanimate},
{frequmine},
{raytrace},
{swaptions},
{vips},
{x264},
{canneal},
{dedup},
{streamcluster},
{STREAM},
{SWISH++},
{HOP},
{KMeans},
{KMeans-Fuzzy},
{ScalParC},
{SVM-RFE},
{\textbf{Average}}},
]
\addplot table[x index=0,y index=2, col sep=space] {img/copper/multi_app/multi_mape-dvfs-simple.txt};
\addplot table[x index=0,y index=2, col sep=space] {img/copper/multi_app/multi_mape-dvfs.txt};
\addplot table[x index=0,y index=2, col sep=space] {img/copper/multi_app/multi_mape-0.0.txt};
%\addplot table[x index=0,y index=2, col sep=space] {img/copper/multi_app/multi_mape-0.1.txt};
%\addplot table[x index=0,y index=2, col sep=space] {img/copper/multi_app/multi_mape-0.2.txt};
\addplot table[x index=0,y index=2, col sep=space] {img/copper/multi_app/multi_mape-0.5.txt};
\end{groupplot}
\end{tikzpicture}
  \caption{Application performance error for DVFS controllers and CoPPer, with and without a gain limit, under interference by a second application (lower is better).}
  \label{fig:copper-multi-mape}
\end{figure}

This section evaluates CoPPer's resilience to interference from
another application.
% Existing work in the literature addresses the problems of co-scheduling resources like cores and memory channels for applications \cite{Pandia,Callisto}, but that is beyond the scope of CoPPer.
We begin the experiment by launching each application with a performance target.
Roughly halfway through each execution, we launch a second application which was randomly selected from the PARSEC benchmark suite.
The second application does not perform any DVFS or power control, but introduces interference into the system by consuming resources.

\figsref{copper-multi-ee}{copper-multi-mape} present the energy efficiency and MAPE results for each application.
As should be expected, MAPE is higher than in previous experiments given that there is significant disturbance to system resources, which also makes the application more difficult to control even when the controller recognizes the disturbance and adapts to it.
Some applications (\eg \app{facesim}, \app{canneal}, \app{dedup}, \app{streamcluster}, \app{STREAM}, and \app{SVM-RFE}) were not able to achieve the performance target after the second application is started, simply because there were not sufficient resources remaining in the system.
Instead, the controller makes a best effort.
These applications drag down the average energy efficiency, which otherwise remains high like in the single application analysis (\eg \app{KMeans and \app{ScalParC}}).
Still, the average across all applications is nearly as good as the ideal \emph{performance} governor would achieve in the absence of interference.


\subsection{Overhead Analysis}
\label{sec:copper-eval-overhead}

This section quantifies the runtime overhead of changing power caps and DVFS frequencies.

The granularity for configuring DVFS settings varies between systems.
In some cases cores are individually configurable, sometimes they are grouped into multiple voltage domains per socket, and other times simply managed at a socket level.
As the number of cores increases, the overhead of managing their DVFS settings in software can become prohibitively high.
Naively, we have to set 32 DVFS frequencies, or one for each logical core.
We reduce this to 16 by limiting ourselves to the physical cores, which requires additional knowledge of the processor topology and virtual core number to physical core mapping.
In practice, having to discover this mapping is an additional burden on developers looking to reduce their overhead when using DVFS.
The RAPL interface exposes power capping with socket granularity, which allows the hardware to more efficiently manage voltage and frequency settings at smaller component and time scales than software can.
As a result, less work needs to be performed by software.
For example, our evaluation system requires us to set only two power caps---one per socket.

We compare the overhead of power capping and setting DVFS frequencies on our evaluation system.
Each test is run one million times, with the average of 5 runs presented.
Power capping alternates between applying the lowest and highest power caps (50 and 430 W, split between the two processor sockets) in each iteration, and achieves an average iteration time of \textbf{15.6\us}.
DVFS alternates between setting the minimum and maximum frequencies (1.2 GHz and TurboBoost) on all 16 physical cores, achieving an average iteration time of \textbf{118.5\us}.
As noted in \secref{copper-analysis}, our experiments use a better actuation function than the sophisticated DVFS controller comes with, so this average overhead is much improved over the double-digit millisecond overhead the controller previously achieved~\cite{POETMCSoC}.
Yet, the overhead imposed by power capping is still another order of magnitude lower than the improved DVFS overhead.
Limiting software to coarse-grained power management at a socket level provides clear runtime benefits over more fine-grained DVFS management.

These values do not even include the overhead of actually computing the correct settings to apply for each window period.
The simple DVFS controller runs in $O(log(n))$ time, where $n$ is the number of available DVFS frequencies, and actuates once per window period.
The sophisticated DVFS controller we compared CoPPer with computes a true optimal DVFS schedule which requires $O(n^2)$ time, and usually requires actuating two different DVFS settings per window period.
This optimal algorithm is what allows the DVFS controller to achieve such good energy efficiency results, and remains practical due to the small number of DVFS settings.
CoPPer runs in constant $O(1)$ time and requires only one actuation per window period.
% Yet, CoPPer achieves better energy efficiency with similar performance error using a constant-time approach rather than requiring a polynomial-time algorithm.

% The DVFS controller will always require a model that maps speedup values to specific DVFS settings.
% If designers were to decide to expose more fine-grained DVFS frequencies, the $O(n^2)$ algorithm would quickly become impractical.
% The approach used in CoPPer only needs a sufficient number of points in the model such that it can assume a mostly linear change in performance between any two power values, so could require fewer data points.
% In fact, as few as two configurations should be sufficient for applications that scale well -- see the mostly linear behavior in \figref{copper-tradeoffs-vips} in \secref{copper-framework}.
% Additional data points are more helpful as the tradeoff space becomes more convex, though.

Our evaluation demonstrates that CoPPer's approach to meeting software performance goals achieves better energy efficiency and similar error to both a simple and a state-of-the-art optimal DVFS controller.
Furthermore, CoPPer is much lower overhead in both computing resource schedules and applying changes to system settings.
