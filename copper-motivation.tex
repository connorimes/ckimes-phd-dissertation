\section{Motivation}
\label{sec:copper-motivation}

We discuss the current processor landscape as justification for why DVFS might soon no longer be controllable by software.
We then illustrate how the simple linear models that work well for meeting performance requirements by tuning DVFS can produce sub-optimal behavior when directly applied to power capping.


\subsection{The Future of Software DVFS}
There are strong indications that DVFS will not be directly controllable by software in future processors.
Since SandyBridge, Intel processors take software DVFS settings as suggestions, and hardware has been free to dynamically alter the actual clock speed and voltage independently from the software-specified setting \cite{lwn602479,KernelPstate}.
With the Skylake architecture, Intel has been actively campaigning to move DVFS management wholly to hardware and instead have software specify power.
The hardware is then free to rapidly change DVFS settings to achieve better performance while still respecting those power limits \cite{SpeedShift}.
For example, if software sets power limits requiring any 50 ms time window to average 100 W, hardware is free to use turbo mode to speed up the processing of any bursty work within that 50 ms, as long as it compensates by running in a low-power state for some of that time.

Of course, even as DVFS shifts to hardware, it will still be software's responsibility to provide its own notion of either ``best'' or ``good enough'' performance, subject to hardware-imposed constraints like thermal design power and energy consumption costs.
The capability to specify power caps and simultaneously provide some optimization is already provided by interfaces like Intel's Running Average Power Limit (RAPL) \cite{RAPL}.
Recent work shows that a combination of RAPL and software resource management can achieve even better performance while guaranteeing power consumption \cite{pupil}.
What is still needed, however, is the software component that guarantees performance without using DVFS.
We address this need with CoPPer, which provides soft performance guarantees by manipulating hardware power caps, thus allowing the hardware to perform fine-grained optimizations.
Because CoPPer operates at the software level, it can still benefit from additional hardware-level optimizations to further increase energy efficiency, like on-chip voltage regulators \cite{Bai2017} or adaptive management of power circuitry \cite{He2013}.


\subsection{The Challenges of Actuating Power}
\label{sec:copper-motivation-power}

DVFS is being replaced with hardware power capping, but meeting performance targets with power caps instead of DVFS settings introduces new challenges.
\figref{copper-tradeoffs-vips} demonstrates how the compute-bound \app{vips} application's performance is affected by DVFS frequencies (\figref{copper-tradeoffs-vips-dvfs}) compared to processor power caps (\figref{copper-tradeoffs-vips-pwr}) on our evaluation system.
Three challenges are immediately apparent from the figures.
First, DVFS produces a linear response in performance, but power capping is \emph{non-linear}.
Second, power capping has \emph{diminishing returns}: as power increases, the change in performance becomes smaller and eventually stop increasing altogether.
Third, \emph{the range} of DVFS settings is much smaller than power settings: the ratio of the maximum to minimum DVFS setting is $11/4=2.75$, but power capping has a ratio of over 6 (as can be seen from the x-axes).

\begin{figure}[t]
  \centering
  % \vskip -1.8em
  \subfloat[DVFS.]%
  {\begin{tikzpicture}

\definecolor{s1}{RGB}{228, 26, 28}
\definecolor{s2}{RGB}{55, 126, 184}
\definecolor{s3}{RGB}{77, 175, 74}
\definecolor{s4}{RGB}{152, 78, 163}
\definecolor{s5}{RGB}{255, 127, 0}

\begin{groupplot}[
    group style={
        group name=plots,
        group size=1 by 1,
        xlabels at=edge bottom,
        xticklabels at=edge bottom,
        vertical sep=5pt
    },
xlabel={\footnotesize $Frequency~(Normalized)$ 
%(normalized to worst case)
},
xlabel near ticks,
height=4cm,
%width=0.5\textwidth,
width = 4.5cm,
xmajorgrids,
ymajorgrids,
grid style={dashed},
xtick={1,1.5,2,2.5,3,3.5,4,4.5},
xticklabels={1,1.5,2,2.5,3,3.5,4,4.5},
xticklabel style={font=\footnotesize},
xmin=0.75,
xmax=3,
yticklabel pos=left,
enlargelimits=false,
tick align = outside,
tick style={white},
xticklabel shift={-5pt},
yticklabel shift={-5pt},
ylabel shift={-2pt},
ylabel style={align=center},
unbounded coords=jump,
]

\nextgroupplot[ylabel={\footnotesize Speedup}, 
%ylabel shift={6mm},
ytick={0,0.5,1.0,1.5,2.0,2.5,3.0},
yticklabels={0,0.5,1.0,1.5,2.0,2.5,3.0},
yticklabel style={font=\footnotesize},
ymin=0.75,
ymax=3.0,
]
\addplot[thick, solid, color=s1, only marks, mark=square*, mark size=1.2] table[x index=6,y index=4,col sep=comma] {img/copper/tradeoffs/tradeoffs-vips-dvfs.csv};


\end{groupplot}

\end{tikzpicture}
%
  \label{fig:copper-tradeoffs-vips-dvfs}}
  \hspace*{0.5cm}
  \subfloat[Power cap.]%
  {\begin{tikzpicture}
\begin{centering}


\definecolor{s1}{RGB}{228, 26, 28}
\definecolor{s2}{RGB}{55, 126, 184}
\definecolor{s3}{RGB}{77, 175, 74}
\definecolor{s4}{RGB}{152, 78, 163}
\definecolor{s5}{RGB}{255, 127, 0}

\begin{groupplot}[
    group style={
        group name=plots,
        group size=1 by 1,
        xlabels at=edge bottom,
        xticklabels at=edge bottom,
        vertical sep=5pt
    },
xlabel={\footnotesize $Power~Cap~(Normalized)$ 
%(normalized to worst case)
},
xlabel near ticks,
height=4cm,
%width=0.5\textwidth,
width = 4.5cm,
xmajorgrids,
ymajorgrids,
grid style={dashed},
xtick={1,1.5,2,2.5,3,3.5,4,4.5,5,5.5,6,6.5},
xticklabels={1,,2,,3,,4,,5,,6},
xticklabel style={font=\footnotesize},
xmin=0.5,
xmax=6.5,
yticklabel pos=left,
enlargelimits=false,
tick align = outside,
tick style={white},
xticklabel shift={-5pt},
yticklabel shift={-5pt},
ylabel shift={-2pt},
ylabel style={align=center},
unbounded coords=jump,
]

\nextgroupplot[ylabel={\footnotesize Speedup}, 
%ylabel shift={6mm},
ytick={0,0.5,1.0,1.5,2.0,2.5,3.0},
yticklabels={0,0.5,1.0,1.5,2.0,2.5,3.0},
yticklabel style={font=\footnotesize},
ymin=0.75,
ymax=3.0,
]
\addplot[thick, solid, color=s1, only marks, mark=square*, mark size=1.2] table[x index=6,y index=4,col sep=comma] {img/copper/tradeoffs/tradeoffs-vips-pwr.csv};


\end{groupplot}
\end{centering}

\end{tikzpicture}
%
  \label{fig:copper-tradeoffs-vips-pwr}}
  \caption{DVFS and power cap performance impacts for \app{vips}.}
  \label{fig:copper-tradeoffs-vips}
\end{figure}

The linear relationship between DVFS and performance makes it easy to apply textbook control theoretic techniques to build a performance management system based on DVFS, and many examples exist in the literature \cite{ICSE2014,SWiFT,KaramanolisEtAl-2005a,lefurgy2008power,josep-isca2016,GRAPE,ControlWare}.
With DVFS, control models assume that---for compute-bound applications---a $2\times$ change in frequency produces a $2\times$ change in performance.
Applying the same techniques to build a performance management system based on power capping is more complicated.
The major issue is that controllers based on time-invariant linear models will have varying error dependent on the current power cap.
The simple solution is to build a linear model that never overestimates the relationship between power and speedup \cite{ICSE2014}.
The downsides to this approach are: (1) a developer must know the maximum error for any application the system might run, and (2) the overestimate slows the control reaction.

\begin{figure}
  \centering
  \begin{tikzpicture}
\begin{centering}

\definecolor{s1}{RGB}{228, 26, 28}
\definecolor{s2}{RGB}{55, 126, 184}
\definecolor{s3}{RGB}{77, 175, 74}
\definecolor{s4}{RGB}{152, 78, 163}
\definecolor{s5}{RGB}{255, 127, 0}

\begin{groupplot}[
    group style={
        group name=plots,
        group size=1 by 1,
        xlabels at=edge bottom,
        xticklabels at=edge bottom,
        vertical sep=5pt
    },
height=3.3cm,
width=0.95\columnwidth,
xmajorgrids,
ymajorgrids,
grid style={dashed},
xmin=0,
xmax=30,
yticklabel pos=left,
enlargelimits=false,
tick align = outside,
tick style={white},
xticklabel shift={-5pt},
yticklabel shift={-5pt},
ylabel shift={-2pt},
ylabel style={align=center},
unbounded coords=jump,
]

\nextgroupplot[ylabel={\footnotesize Performance \\ \footnotesize (Normalized)}, % Performance
%xtick={0,500,1000,1500,2000,2500,3000,3500,4000,4500},
ytick={0.0,0.5,0.75,0.875, 1.0,1.125,1.25,1.5,2.0},
yticklabels={,0.5,0.75,,1.0,,1.25,1.5,2.0},
%xtick={90,95,100,105,110},
%xticklabels={90,95,100,105,110},
%yticklabel style={font=\footnotesize},
xlabel={\footnotesize Control Period},
xmin=0,
xmax=30,
ymin=.75,
ymax=1.25,
legend entries={{Performance~Requirement},{DVFS},{Conservative~Power~Cap}, {Aggressive~Power~Cap}},
legend style={fill=none,draw=none,at={(0.5,1.5)},anchor=north,legend columns=4,
line width=5pt,font=\footnotesize},
]

\addplot[thick, dashed, black] coordinates {(0,1) (399,1)};
\addplot[thick, solid, color=s4, mark=o] table[x index=0,y index=1,col sep=tab] {img/copper/VIPS-example.txt};
\addplot[thick, solid, color=s5, mark=square] table[x index=0,y index=2,col sep=tab] {img/copper/VIPS-example.txt};
\addplot[thick, solid, color=s3, mark=triangle] table[x index=0,y index=2,col sep=tab] {img/copper/VIPS-example-oscillate.txt};
%\addplot[thick, dashed, black] coordinates {(99,0) (99,1.5)};
%\addplot[thick, dashed, black] coordinates {(130,0) (130, 2)};
\end{groupplot}
\end{centering}

\end{tikzpicture}

  % \vskip -1.8em
  \caption{DVFS and power capping with linear models.}
  \label{fig:copper-vips-example}
\end{figure}

\figref{copper-vips-example} shows the difference between a controller based on a linear DVFS model extracted from \figref{copper-tradeoffs-vips-dvfs} and two (one conservative and one aggressive) based on fitting time-invariant linear models to the power capping data from \figref{copper-tradeoffs-vips-pwr}.
All approaches start at the maximum DVFS or power setting and must bring performance down to the required level while minimizing energy.
\figref{copper-vips-example} shows the DVFS controller quickly reaches the desired performance, but the conservative power capping controller is much slower to react.
The conservative approach never violates the performance requirement, but its slow reaction wastes energy.
The aggressive approach over-reacts, oscillating around the performance target instead of settling on it.
These results demonstrate how sensitive the power capping approaches can be to their input models.
The next section describes an adaptive control design for meeting performance goals with power capping that overcomes the difficulties highlighted by this example without requiring a user-specified model.
