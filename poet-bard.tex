\section{Bard -- An Extension of POET}
\label{sec:poet-bard}

In this section, we briefly describe Bard, and extension to POET that adds the ability to meet soft power constraints and maximize performance.
For a complete description and evaluation, see the original Bard publication \cite{Bard}.


\subsection{Motivation}

An application on an embedded device (like a smartphone) may need to meet timing constraints when executing in the foreground to deliver a sufficiently high level of performance to satisfy users.
The same application may have tasks that execute when running in the background that are not time-sensitive and would benefit the user by keeping power consumption low so as not to drain the battery.
Other systems that normally provide timing guarantees must continue operating with reduced capacity during hardware failures (\eg a broken cooling fan) or during periods of extremely low energy reserves (\eg at night or during cloudy weather for a system that harvests energy from a solar panel).

A naive approach to solving this problem is to integrate two different solutions into the application -- one that meets timing constraints, and another for power.
If the libraries are not properly managed, they may compete for control of system resources.
This can result in failing to meet either goal, causing poor performance and high power consumption.
Furthermore, it adds additional complexity to the program and increases the overhead in development and validation testing.
Also, many existing solutions are specific to particular applications or systems, \ie they are not portable.

POET's portability makes it a perfect starting point for solving this problem.
Bard extends POET by adding the ability to track both performance and power behavior and allow the user to decide which constraint/optimization scheme to use.
In summary, Bard (1) meets either timing or power constraints, (2) optimizes the other, and (3) remains portable across systems.


\subsection{Further Generalizing POET's Design}

Bard's formulations are very similar to POET.
The key to rapidly switching from a performance to a power constraint and vice versa is to continually estimates both base values.
The Bard controller then generates its generic control signal, $x(t)$, which can be either \emph{speedup} or \emph{powerup}:
\begin{eqnarray}
  x(t) = x(t-1) + (1-p) \cdot \frac{e_x(t)}{b_x(t)}
  \label{eqn:bard-xup-control}
\end{eqnarray}
The $x(t)$ term is the speedup or powerup computed (generalized $s(t)$ from POET); $b_x(t)$ represents either base speed or power, depending on the constraint.

The constrained optimization problem for meeting a performance goal and minimizing energy consumption remains the same as in POET (\eqnrref{poet-power}{poet-time} in \secref{poet-optimizer}).
To meet a power goal $P_r$ and maximize performance, the problem becomes:
\begin{eqnarray}
  \maximize && \sum_{c=0}^{C-1} \tau_c \cdot s_c \label{eqn:timing} \\
  \st %&& \nonumber\\
  && \sum_{c=0}^{C-1} \tau_c \cdot p_c \cdot b_p(t) \le P_r \label{eqn:power} \\
  && \sum_{c=0}^{C-1} \tau_c = 1 \label{eqn:work} \\
  && \tau_c \ge 0, \qquad \forall c \in \{0,\ldots,C-1\} \label{eqn:time}
\end{eqnarray}

\begin{algorithm}[t]
  \caption{Finding an Optimal Configuration Schedule.}
  \begin{algorithmic}
    \footnotesize
    \Require $C$ \Comment{system configurations}
    \Require $W$ \Comment{workload size}
    \Require $constraint$ \Comment{PERFORMANCE or POWER}
    \Require $x(t)$ \Comment{speedup/powerup, depending on $constraint$}
  \State $U = \{c \mid x_c \le x(t) \}$
    \State $O = \{c \mid x_c > x(t)\}$
    \State $candidates = U \times O = \{\langle u,o \rangle \mid u \in U, o \in O\}$
    \State $cost = \infty$
    \State $optimal = \langle -1,-1 \rangle$
    \State $schedule = \langle -1,-1 \rangle$ \newline
    \For {$\langle u,o \rangle \in candidates$} \Comment{loop over all pairs}
    \State $j_u = W \cdot \frac{x_u \cdot (x_o - x(t))}{x(t) \cdot (x_o - x_u)}$
  \Comment{jobs spent in each config}
    \State $j_o = W - j_u$
  \If {$constraint = POWER$}
  \Comment{cost of this pair}
  \State $newCost = \frac{W}{j_u \cdot s_u + j_o \cdot s_o}$
  \Comment{normalized latency}
  \Else
  \State $newCost = \frac{j_u \cdot p_u + j_o \cdot p_o}{W}$
  \Comment{normalized energy}
  \EndIf
    \If {$newCost < cost$}
  \Comment{compare cost to best so far}
    \State $cost = newCost$
    \State $optimal = \langle u,o \rangle$
    \State $schedule = \langle j_u,j_o \rangle$
    \EndIf
    \EndFor \newline \newline
    \Return $optimal$ \Comment{pair of configurations with minimal cost} \newline
    \Return $schedule$ \Comment{jobs to spend in each configuration}
  \end{algorithmic}
  \label{algo:bard-optimal}
\end{algorithm}

\algoref{bard-optimal} further generalizes the POET optimizer behavior (\algoref{poet-optimal} in \secref{poet-optimizer}) to compute either a minimal-energy or maximal-performance schedule.
It takes as input the configuration set $C$, workload size $W$, the $constraint$ type, and the generic control signal $x(t)$.
% It similarly finds the optimal pair of configurations by partitioning $C$ into two pairwise disjoint subsets, $U$ and $O$.
% $U$ contains the configurations with speedup or powerup values less than or equal to the signal computed by the controller with \eqnref{bard-xup-control}.
% $O$ contains the rest, with values greater than the controller's signal.
% The algorithm then loops over all possible pairs of configurations, with one chosen from each subset (the Cartesian product), and determines the schedule required to achieve the speedup or powerup given by the controller.
The algorithm similarly loops over all possible pairs of configurations and determines the schedule required to achieve the speedup or powerup given by the controller.
% Once the schedule is computed, its cost is determined.
If the constraint is performance, energy is minimized; if the constraint is power, performance is maximized (latency is minimized).
% The algorithm remembers the pair with the lowest cost and returns the optimal pair and their schedule.

We add the \emph{constraint} type (\variable{PERFORMANCE} or \variable{POWER}) to the initialization function of the POET API.
If \variable{PERFORMANCE} is specified, Bard meets a performance target and minimizes energy.
If \variable{POWER} is specified, Bard meets the power target and maximizes performance.
We then add a setter function to support changing the constraint type at runtime.


\subsection{Evaluation}


\begin{figure}[t]
\centering
\begin{minipage}[t]{.45\columnwidth}
\lstset{
  belowskip=0pt,
  aboveskip=0pt,
  numbers=none,
}
\begin{lstlisting}[frame=tlr,%
  caption={System-agnostic.},%
  label={lst:control_config_example}]%

#id        speedup         powerup
0          1               1
1          1.55            1.06
2          2.11            1.11
3          2.16            1.12
4          2.66            1.17
5          3.36            1.22
6          4.51            1.31
7          5.11            1.37
\end{lstlisting}
\end{minipage}\hfill
\begin{minipage}[t]{.45\columnwidth}
\lstset{
  belowskip=0pt,
  aboveskip=0pt,
  numbers=none,
}
\begin{lstlisting}[frame=tlr,%
  caption={System-specific.},%
  label={lst:cpu_config_example}]%

#id cores frequencies
0   0x01  200000,-,-,-,200000,-,-,-
1   0x01  300000,-,-,-,200000,-,-,-
2   0x01  400000,-,-,-,200000,-,-,-
3   0x03  200000,-,-,-,200000,-,-,-
4   0x01  500000,-,-,-,200000,-,-,-
5   0x07  200000,-,-,-,200000,-,-,-
6   0x0F  200000,-,-,-,200000,-,-,-
7   0x07  300000,-,-,-,200000,-,-,-
\end{lstlisting}
\end{minipage}
% \vskip -1.5em
\caption{Snippets of Bard configuration files.}
\label{fig:bard-config-examples}
\end{figure}

We evaluate Bard on the same Vaio tablet as POET, and use a newer ARM big.LITTLE device, an ODROID-XU3, which supports running both the big and LITTLE cluster simultaneously.
Although Bard is independent of the system-specific configurations, we need a new format to use in our evaluation.
In the system-specific configuration in \figref{bard-config-examples} contains an identifier, a core mask (instead of core count), and a comma-delimited list of DVFS frequencies to apply (instead of a single frequency).
A dash indicates that a DVFS frequency does not need to be applied to a core, either because the frequency does not matter or it is already managed through another affected core.
For example, configuration with $id=7$ assigns cpu0, cpu1, and cpu2 (core mask 0x07) and sets the DVFS frequency on cpu0 to 300 MHz and cpu4 to 200 MHz.
% On this particular system, cores 0--3 are in one DVFS domain and cores 4--7 are in another.
% Therefore, applying a DVFS setting of 300 MHz on cpu0 sets the same frequency on cores 1--3, and applying a frequency of 200 MHz on cpu4 sets the same frequency on cores 5--7.
% We specify a dash for cores 1--3 and 5--7 to prevent setting their frequencies explicitly which could result in unnecessary overhead.
% Although cores 4--7 are not assigned, forcing a low DVFS setting reduces their power consumption while they idle.

\begin{figure*}[t]
  \begin{tikzpicture}
\begin{centering}

\definecolor{s1}{RGB}{228, 26, 28}
\definecolor{s2}{RGB}{55, 126, 184}
\definecolor{s3}{RGB}{77, 175, 74}
\definecolor{s4}{RGB}{152, 78, 163}
\definecolor{s5}{RGB}{255, 127, 0}

\begin{groupplot}[
    group style={
        group name=plots,
        group size=4 by 2,
        xlabels at=edge bottom,
        xticklabels at=edge bottom,
        vertical sep=10pt
    },
height=3.5cm,
width=0.26\textwidth,
xmajorgrids,
ymajorgrids,
grid style={dashed},
xmin=40,
xmax=500,
yticklabel pos=left,
enlargelimits=false,
tick align = outside,
tick style={white},
xticklabel shift={-5pt},
yticklabel shift={-5pt},
yticklabel style={font=\footnotesize},
ylabel style={align=center},
legend cell align=left,
legend style={ column sep=1ex },
unbounded coords=jump,
]

% Performance

\nextgroupplot[
ylabel={\footnotesize Normalized \\ Performance}, % Performance
xlabel={\footnotesize x264},
xlabel shift={-2.75cm},
xtick={0,125,251,375,500},
ytick={0,0.25,0.5,0.75,1,1.25,1.5,1.75,2},
yticklabels={0.0,,0.5,,1.0,,1.5,,2.0},
ymin=0,
ymax=2,
xmin=0,
xmax=500,
legend entries={{Vaio},{ODROID-XU3}},
legend style={draw=none,at={(2.5,1.65)},anchor=north,legend columns=4,line width=5pt,font=\footnotesize},
]
\addplot[thick, solid, color=s1] table[x index=1,y index=4,col sep=comma] {img/bard/perf-pwr-x264-v.csv};
\addplot[thick, solid, color=s3] table[x index=1,y index=4,col sep=comma] {img/bard/perf-pwr-x264-o.csv};
\addplot[thick, dashed, black] coordinates {(251,0) (251, 2)};
\addplot[thick, solid, black] coordinates {(0,1) (250, 1)};


\nextgroupplot[
xtick={0,60,119,180,240,260},
xlabel={\footnotesize bodytrack},
xlabel shift={-2.75cm},
ytick={0,0.25,0.5,0.75,1,1.25,1.5,1.75,2},
yticklabels={0.0,,0.5,,1.0,,1.5,,2.0},
ymin=0,
ymax=2,
xmin=0,
xmax=260,
]
\addplot[thick, solid, color=s1] table[x index=0,y index=1,col sep=comma] {img/bard/perf-pwr-bodytrack-v.csv};
\addplot[thick, solid, color=s3] table[x index=0,y index=1,col sep=comma] {img/bard/perf-pwr-bodytrack-o.csv};
\addplot[thick, dashed, black] coordinates {(119,0) (119, 2)};
\addplot[thick, solid, black] coordinates  {(0,1) (119, 1)};


\nextgroupplot[
xlabel shift={-2.75cm},
xlabel={\footnotesize sha},
ytick={0,0.25,0.5,0.75,1,1.25,1.5,1.75,2},
yticklabels={0.0,,0.5,,1.0,,1.5,,2.0},
ymin=0,
ymax=2,
xmin=100,
xmin=0,
xmax=500,
xtick={0,125,249,375,500},
]
\addplot[thick, solid, color=s1] table[x index=0,y index=1,col sep=comma] {img/bard/perf-pwr-sha-v.csv};
\addplot[thick, solid, color=s3] table[x index=0,y index=1,col sep=comma] {img/bard/perf-pwr-sha-o.csv};
\addplot[thick, dashed, black] coordinates {(249,0) (249, 2)};
\addplot[thick, solid, black] coordinates  {(0,1) (248, 1)};


\nextgroupplot[
xlabel shift={-2.75cm},
xlabel={\footnotesize STREAM},
ytick={0,0.25,0.5,0.75,1,1.25,1.5,1.75,2},
yticklabels={0.0,,0.5,,1.0,,1.5,,2.0},
ymin=0,
ymax=2,
xmin=0,
xmax=500,
xtick={0,100,200,299,400,500},
]
\addplot[thick, solid, color=s1] table[x index=0,y index=1,col sep=comma] {img/bard/perf-pwr-stream-v.csv};
\addplot[thick, solid, color=s3] table[x index=0,y index=1,col sep=comma] {img/bard/perf-pwr-stream-o.csv};
\addplot[thick, dashed, black] coordinates {(299,0) (299, 2)};
\addplot[thick, solid, black] coordinates  {(0,1) (298, 1)};

% Power

\nextgroupplot[
ylabel={\footnotesize Normalized \\ Power}, % Power
ylabel shift={6.8pt},
xtick={0,125,250,375,500},
ytick={0,1,2,3,4,5,6,7},
yticklabels={,1,,3,,5,,7},
ymin=0,
ymax=7,
xmin=0,
xmax=500,
xlabel={\footnotesize $time$ [frame]},
xlabel near ticks,
xticklabels={0,,250,,500},
xticklabel style={font=\footnotesize},
]
\addplot[thick, solid, color=s1] table[x index=1,y index=5,col sep=comma] {img/bard/perf-pwr-x264-v.csv};
\addplot[thick, solid, color=s3] table[x index=1,y index=5,col sep=comma] {img/bard/perf-pwr-x264-o.csv};
\addplot[thick, dashed, black] coordinates {(251,4) (251, 12)};
\addplot[thick, solid, color=s1] table[x index=1,y index=5,col sep=comma] {img/bard/perf-pwr-x264-v.csv};
\addplot[thick, solid, color=s3] table[x index=1,y index=5,col sep=comma] {img/bard/perf-pwr-x264-o.csv};
\addplot[thick, dashed, black] coordinates {(251,0) (251, 10)};
\addplot[thick, solid, black] coordinates {(251,1) (500, 1)};



\nextgroupplot[
ytick={0,1,2,3,4,5,6,7},
yticklabels={,1,,3,,5,,7},
ymin=0,
ymax=7,
xmin=0,
xmax=260,
xtick={0,60,119,180,240,260},
xticklabels={0,,119,,240,},
xlabel={\footnotesize $time$ [frame]},
xlabel near ticks,
xticklabel style={font=\footnotesize},
]
\addplot[thick, solid, color=s3] table[x index=0,y index=2,col sep=comma] {img/bard/perf-pwr-bodytrack-o.csv};
\addplot[thick, dashed, black] coordinates {(119,3) (119, 12)};
\addplot[thick, solid, color=s1] table[x index=0,y index=2,col sep=comma] {img/bard/perf-pwr-bodytrack-v.csv};
\addplot[thick, solid, color=s3] table[x index=0,y index=2,col sep=comma] {img/bard/perf-pwr-bodytrack-o.csv};
\addplot[thick, dashed, black] coordinates {(119,0) (119, 10)};
\addplot[thick, solid, black] coordinates {(119,1) (260, 1)};

\nextgroupplot[
ytick={0,1,2,3,4,5,6,7},
yticklabels={,1,,3,,5,,7},
ymin=0,
ymax=7,
xmin=100,
xmin=0,
xmax=500,
xtick={0,125,250,375,500},
xticklabels={0,,249,,500},
xlabel={\footnotesize $time$ [1000 hashes]},
xlabel near ticks,
xticklabel style={font=\footnotesize},
]
\addplot[thick, solid, color=s3] table[x index=0,y index=2,col sep=comma] {img/bard/perf-pwr-sha-o.csv};
\addplot[thick, dashed, black] coordinates {(249,4) (249, 12)};
\addplot[thick, solid, color=s1] table[x index=0,y index=2,col sep=comma] {img/bard/perf-pwr-sha-v.csv};
\addplot[thick, solid, color=s3] table[x index=0,y index=2,col sep=comma] {img/bard/perf-pwr-sha-o.csv};
\addplot[thick, dashed, black] coordinates {(249,0) (249, 10)};
\addplot[thick, solid, black] coordinates {(249,1) (500, 1)};


\nextgroupplot[
ytick={0,1,2,3,4,5,6,7},
yticklabels={,1,,3,,5,,7},
ymin=0,
ymax=7,
xmin=0,
xmax=500,
xtick={0,100,200,299,400,500},
xticklabels={0,,,299,,500},
xlabel={\footnotesize $time$ [10M ops]},
xlabel near ticks,
xticklabel style={font=\footnotesize},
]
\addplot[thick, solid, color=s3] table[x index=0,y index=2,col sep=comma] {img/bard/perf-pwr-stream-o.csv};
\addplot[thick, dashed, black] coordinates {(299,2) (299, 4)};
\addplot[thick, solid, color=s1] table[x index=0,y index=2,col sep=comma] {img/bard/perf-pwr-stream-v.csv};
\addplot[thick, solid, color=s3] table[x index=0,y index=2,col sep=comma] {img/bard/perf-pwr-stream-o.csv};
\addplot[thick, dashed, black] coordinates {(299,0) (299, 10)};
\addplot[thick, solid, black] coordinates {(299,1) (500, 1)};



\end{groupplot}
\end{centering}

\end{tikzpicture}
  
  \caption{Changing constraints from timing to power.}
  \label{fig:bard-perf-pwr-change-pick4}
\end{figure*}

In \figref{bard-perf-pwr-change-pick4}, we observe the behavior of launching four applications with a performance goal, then switching to a power goal.
% On both evaluation systems, we start with a performance target of 75\%.
% About halfway through the execution, we switch to a power target.
% We set the power target to 25\% on the Vaio.
% The extreme convexity of the ODROID-XU3's power/performance tradeoff space means that switching from a 75\% performance target to a 25\% power target does not result in as significant a change in system behavior as it does on the Vaio.
% Instead, we set a more challenging power target of $0.5$ Watts on the ODROID to illustrate taking advantage of the low-power LITTLE cores.
The top portion has performance data, normalized to the target from the first half of the execution.
The bottom portion of the charts shows power data, normalized to the target set in the second half of the execution.
The dashed vertical lines indicate when the switch from performance to power constraints is made.
The solid black horizontal lines represent the goals.

The results demonstrate several important features.
Bard first keeps the performance at or above the target, then keeps power at or below the new target.
MAPE quantifies performance deficit in the first half and power cap violations in the second half.
Similarly, our oracle computes the minimal-energy schedule, then the maximal-performance schedule to determine the total efficiency of the execution.
On the Vaio, Bard achieves $1.58\%$ MAPE and $92\%$ efficiency on average.
On the ODROID, it achieves $1.97\%$ MAPE and $91\%$ efficiency.
As \figref{bard-perf-pwr-change-pick4} shows, the time taken to switch between the timing and power constraints is quite small---just one period.
The small fluctuations seen in the first portion of some executions (like x264) are not caused by Bard, but rather by the variability of the application inputs which makes them difficult to control.
These results demonstrate that Bard achieves its design goal of providing a single, portable framework for meeting either timing or power constraints and dynamically switching between the two.
