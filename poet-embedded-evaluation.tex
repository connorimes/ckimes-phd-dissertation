\section{Embedded Systems Evaluation}
\label{sec:poet-embedded-evaluation}

This chapter describes POET's evaluation on embedded systems---the Vaio and the ODROID-XU+E (henceforth referred to simply as the ODROID)---and is divided into five parts.
First, we demonstrate POET's ability to meet performance goals, formulated as job latency requirements.
Next, we quantify the energy consumption of the resulting system, then compare the energy of POET's general approach to one that controls latency and energy tradeoffs using just DVFS.
We then evaluate POET's ability to adapt to input with multiple phases, and finally, its ability to run subject to the interference of other applications.


\subsection{Meeting Latency Targets}
\label{sec:poet-eval-embedded-performance}

To test POET's ability to meet latency targets, we first \emph{characterize} each application $i$ by executing in all possible configurations on both systems to determine the minimum average latency $m_i$ and derive an \emph{oracle} to be used for our analysis.
This oracle determines an optimal resource schedule for each target without missing any deadlines, and has no computation or configuration switching overhead.
For each application, we impose four latency targets.
The targets cover a wide range of achievable goals, from 25\% to 95\% of each system's performance capacity, \ie a 25\% goal means that the target is set to $4 \times m_i$.
Applications are launched in the maximum-resource configuration (configuration $C-1$ as described in \secref{poet-optimizer}).
POET observes application behavior during the first window period, then begins adapting the system to meet the latency goals while minimizing energy consumption.

We quantify POET's ability to meet the latency goals by measuring each job's latency and comparing it to the goal.
We then compute the Mean Absolute Percentage Error (MAPE), a standard metric in control theory to evaluate the behavior of controllers~\cite{ICSE2014}.
For an application composed of $n$ jobs:
\begin{equation}
\text{MAPE} = 100\% \cdot \frac{1}{n} \sum\limits_{i=1}^{n}
\left \{
\begin{array}{ll}
d_m(i) > D_{ref}  :& \frac{d_m(i) - D_{ref}}{D_{ref}} \\
d_m(i) \le D_{ref}  :& 0
\end{array} \right.
 \label{eqn:poet-mape}
\end{equation}
where $D_{ref}$ is the specified job latency requirement (deadline) and $d_m(i)$ is the measured latency for the $i$-th job.
In other words, for each missed deadline we add a term that depends on the relative tardiness between the target and measured latency.

\begin{figure}[t]
  \centering
  \begin{tikzpicture}

\begin{groupplot}[
    group style={
        group name=plots,
        group size=1 by 2,
        xlabels at=edge bottom,
        xticklabels at=edge bottom,
        vertical sep=15pt
    },
%axis x line* = bottom,
xlabel near ticks,
major x tick style = transparent,
xlabel={},
height=3.5cm,
width=0.95\columnwidth,
xmin=0,
xmax=10,
enlargelimits=false,
tick align = outside,
tick style={white},
ytick=\empty,
ylabel style={align=center},
legend cell align=left, 
legend style={ column sep=1ex },
ymajorgrids,
grid style={dashed},
]
\nextgroupplot[ybar=\pgflinewidth,
bar width=6pt,
ylabel={Vaio\\MAPE (\%)},
ylabel shift={0mm},
ymin=0,
ymax=8,
ytick={0,2,4,6,8},
yticklabels={0,2,4,6,8},
legend entries = {{25\%},{50\%},{75\%},{95\%}},
legend style={draw=none,legend columns=4,at={(0.5,1.4)},anchor=north,font=\footnotesize},
]
\addplot table[x index=0,y index=2, col sep=tab] {img/poet/mape-v.txt};
\addplot table[x index=0,y index=3, col sep=tab] {img/poet/mape-v.txt};
\addplot table[x index=0,y index=4, col sep=tab] {img/poet/mape-v.txt};
\addplot table[x index=0,y index=5, col sep=tab] {img/poet/mape-v.txt};

\nextgroupplot[ybar=\pgflinewidth,
bar width=6pt,
ylabel={ODROID\\MAPE (\%)},
ylabel shift={0mm},
ymin=0,
ymax=8,
ytick={0,2,4,6,8},
yticklabels={0,2,4,6,8},
xticklabel shift={0pt},
x tick label style={rotate=35, anchor=east, font=\scriptsize},
xtick={1,2,3,4,5,6,7,8,9,10},
xticklabels={
{blackscholes},
{bodytrack},
{facesim},
{ferret},
{x264},
{dijkstra},
{sha},
{STREAM},
{\textbf{Average}}},
]
\addplot table[x index=0,y index=2, col sep=tab] {img/poet/mape-o.txt};
\addplot table[x index=0,y index=3, col sep=tab] {img/poet/mape-o.txt};
\addplot table[x index=0,y index=4, col sep=tab] {img/poet/mape-o.txt};
\addplot table[x index=0,y index=5, col sep=tab] {img/poet/mape-o.txt};

\end{groupplot}
\end{tikzpicture}
  
  \caption{Embedded systems latency error (lower is better, 0 is optimal).}
  \label{fig:poet-embedded-mape}
\end{figure}

\figref{poet-embedded-mape} presents the MAPE values for each application for the four latency targets on both the Vaio and the ODROID.
In general, the larger the variance in the application behavior, the larger the error.
This is not surprising since more volatile applications are harder to control.
However, the results indicate that MAPE values are generally low.
On the Vaio, the average MAPE for all applications is well below 2.5\% for all targets, typically closer to 1.5\%.
The ODROID presents similar results.
The MAPE metric is unforgiving since it penalizes every violation of the latency target, yet POET achieves low MAPE even for applications that were not inherently designed to support predictable timing.


\subsection{Energy Minimization}

This section evaluates POET's energy minimization strategy.
As discussed, we have measured latency and energy consumption for all applications in all configurations.
Therefore, the oracle has perfect knowledge of each application's behavior.
We quantify POET's energy consumption by comparing its achieved energy to the oracle derived from the application characterization.
Although the oracle has zero overhead, meeting all latency targets while simultaneously achieving optimal energy is not actually possible in practice as it would require knowledge of the future and no overhead.


\begin{figure}[t]
  \centering
  \begin{tikzpicture}

\begin{groupplot}[
    group style={
        group name=plots,
        group size=1 by 2,
        xlabels at=edge bottom,
        xticklabels at=edge bottom,
        vertical sep=15pt
    },
%axis x line* = bottom,
xlabel near ticks,
major x tick style = transparent,
xlabel={},
height=3.5cm,
width=0.95\columnwidth,
xmin=0,
xmax=10,
enlargelimits=false,
tick align = outside,
tick style={white},
ylabel style={align=center},
ytick=\empty,
legend cell align=left, 
legend style={ column sep=1ex },
ymajorgrids,
grid style={dashed},
]
\nextgroupplot[ybar=\pgflinewidth,
bar width=6pt,
ylabel={Vaio\\Energy},
ylabel shift={0mm},
ymin=1.0,
ymax=1.2,
ytick={1,1.05,1.1,1.15,1.2},
yticklabels={1.00,1.05,1.10,1.15,1.20},
legend entries = {{25\%},{50\%},{75\%},{95\%}},
legend style={draw=none,legend columns=4,at={(0.5,1.4)},anchor=north,font=\footnotesize},
]
\addplot table[x index=0,y index=2, col sep=tab] {img/poet/ee-v.txt};
\addplot table[x index=0,y index=3, col sep=tab] {img/poet/ee-v.txt};
\addplot table[x index=0,y index=4, col sep=tab] {img/poet/ee-v.txt};
\addplot table[x index=0,y index=5, col sep=tab] {img/poet/ee-v.txt};

\nextgroupplot[ybar=\pgflinewidth,
bar width=6pt,
ylabel shift={0mm},
ymin=1.0,
ymax=1.2,
ytick={1,1.05,1.1,1.15,1.2},
yticklabels={1.00,1.05,1.10,1.15,1.20},
ylabel={ODROID\\Energy},
ylabel shift={0mm},
xticklabel shift={0pt},
x tick label style={rotate=35, anchor=east, font=\scriptsize},
xtick={1,2,3,4,5,6,7,8,9,10},
xticklabels={
{blackscholes},
{bodytrack},
{facesim},
{ferret},
{x264},
{dijkstra},
{sha},
{STREAM},
{\textbf{Average}}},
]
\addplot table[x index=0,y index=2, col sep=tab] {img/poet/ee-o.txt};
\addplot table[x index=0,y index=3, col sep=tab] {img/poet/ee-o.txt};
\addplot table[x index=0,y index=4, col sep=tab] {img/poet/ee-o.txt};
\addplot table[x index=0,y index=5, col sep=tab] {img/poet/ee-o.txt};

\end{groupplot}
\end{tikzpicture}  
  \caption{Embedded systems energy consumption (lower is better, 1 is optimal).}
  \label{fig:poet-embedded-ee}
\end{figure}

For each application, we run POET for each latency target and record the achieved energy consumption.
We then compute the ratio of the energy consumption with POET to the effective minimal energy.
Unity represents minimal energy and values greater than 1 show energy consumption above the optimal.
\figref{poet-embedded-ee} presents the normalized energy data for each application on both the Vaio and the ODROID.
The data includes the overhead of POET's runtime, which consumes additional energy executing the control and optimization tasks.
On average across all applications and targets, POET's energy consumption exceeds optimal by 1.3\% on the Vaio and by 2.9\% on the ODROID.
These results demonstrate that POET achieves near-optimal energy consumption in practice.

The most troublesome test is \app{dijkstra} on the ODROID with a latency target of 75\%, which exceeds optimal by about 16\%.
The true optimal schedule just barely achieves this goal by varying the DVFS setting between 1.2 and 1.1 GHz.
Any overhead larger than 2\% requires a clock speed of 1.3 GHz.
Unfortunately, this is in the area of steeply diminishing returns for the ODROID.
Compensating for this overhead almost entirely accounts for the energy difference between POET and optimal.
POET's runtime overhead is small, but non-zero, so POET uses the higher clock speed.
POET's overhead is due in part to its generality; \ie its ability to handle multiple actuators that may affect energy and latency.
The next section highlights the benefits of this generality.


\subsection{Comparison with DVFS}

\begin{figure}[t]
  \centering
  \begin{tikzpicture}

\begin{groupplot}[
    group style={
        group name=plots,
        group size=1 by 2,
        xlabels at=edge bottom,
        xticklabels at=edge bottom,
        vertical sep=15pt
    },
% axis x line* = bottom,
xlabel near ticks,
major x tick style = transparent,
xlabel={},
height=3.5cm,
width=0.95\columnwidth,
xmin=0,
xmax=5,
enlargelimits=false,
tick align = outside,
tick style={white},
ylabel style={align=center},
ytick=\empty,
ymin=1.0,
ymax=1.2,
ytick={1,1.05,1.1,1.15,1.2},
yticklabels={1.00,1.05,1.10,1.15,1.20},
legend cell align=left,
legend style={ column sep=1ex },
ymajorgrids,
grid style={dashed},
]
\nextgroupplot[ybar=\pgflinewidth,
ylabel={Vaio\\Energy\\(Normalized)},
bar width=8pt,
legend entries = {{DVFS-only},{POET}},
legend style={draw=none,legend columns=2,at={(0.5,1.4)},anchor=north,font=\footnotesize},
]
\addplot table[x index=0,y index=4, col sep=tab] {img/poet/dvfs-compare-short.txt};
\addplot table[x index=0,y index=5, col sep=tab] {img/poet/dvfs-compare-short.txt};

\nextgroupplot[ybar=\pgflinewidth,
bar width=8pt,
ylabel={ODROID\\Energy\\(Normalized)},
ylabel shift={0mm},
ymin=1.0,
ymax=5.5,
ytick={1.0,2.0,3.0,4.0,5.0},
yticklabels={1.00,2.00,3.00,4.00,5.00},
xticklabel shift={0pt},
x tick label style={rotate=35, anchor=east, font=\footnotesize},
xtick={1,2,3,4},
xticklabels={
{25\%},
{50\%},
{75\%},
{95\%}},
]
\addplot table[x index=0,y index=2, col sep=tab] {img/poet/dvfs-compare-short.txt};
\addplot table[x index=0,y index=3, col sep=tab] {img/poet/dvfs-compare-short.txt};

\end{groupplot}
\end{tikzpicture}
  \caption{Comparison of average energy consumption with DVFS-only versus POET (lower is better, 1 is optimal).}
  \label{fig:poet-embedded-dvfs-compare}
\end{figure}

Several energy management approaches have been proposed that optimally tune DVFS settings to meet timing constraints while reducing energy consumption \cite{Albers}.
In this section, we compare POET's energy consumption to an approach which only uses DVFS.
To do so, we develop system configuration files that only specify changes in DVFS settings and deploy POET on both embedded platforms with these configurations.
We compare this \emph{DVFS-only} approach to POET's more general approach which coordinates multiple resource types and uses different resources on different platforms.

\figref{poet-embedded-dvfs-compare} summarizes the data comparing a DVFS-only approach to POET.
The charts show latency targets on the x-axes and energy consumption normalized to optimal on the y-axes (for POET, this is the same data shown in \figref{poet-embedded-ee}).
For each latency target, the figure shows the average energy over optimal (across all benchmarks) for both DVFS-only and POET on both platforms.
At the higher latency (lower performance, \eg 25\%) targets, POET saves substantial energy.
The energy savings are especially high on the ODROID as POET is able to take advantage of cluster migration and the low-power LITTLE cores, whereas a DVFS-only approach cannot exploit this feature.
This data clearly demonstrates that systems provisioned for a rarely-seen, worst-case latency can greatly benefit from POET's generalized approach.
This is also further confirmation of prior studies showing that DVFS by itself is not optimal \cite{Hoffmann2012,MeisnerISCA2011}.


\subsection{Responding to Application Phases}

In this test, we examine POET's ability to cope with input whose workload varies with time.
We execute the \app{x264} application using an input that is a combination of three videos of varying difficulty.
The input thus has three distinct phases, each composed of 500 jobs (frames).
\figref{poet-embedded-phases-default} shows time series data for both job latency and power on the Vaio and the ODROID when they run without POET in their highest-performing configurations.
Latency is normalized to the maximum latency measured for any job, \ie the empirically determined worst case.
We use this worst case result to derive the latency target for the POET tests.
Frames that take less time are easier to encode and require fewer system resources to meet a latency goal compared to the frame that takes the most time, presenting opportunities to save energy.
We present the raw data for power; energy is the integral of those curves.

\begin{figure}[t]
  \centering
  \begin{tikzpicture}

\definecolor{s1}{RGB}{228, 26, 28}
\definecolor{s2}{RGB}{55, 126, 184}
\definecolor{s3}{RGB}{77, 175, 74}
\definecolor{s4}{RGB}{152, 78, 163}
\definecolor{s5}{RGB}{255, 127, 0}

\begin{groupplot}[
    group style={
        group name=plots,
        group size=1 by 2,
        xlabels at=edge bottom,
        xticklabels at=edge bottom,
        vertical sep=15pt
    },
height=3.5cm,
width=0.95\columnwidth,
xmajorgrids,
ymajorgrids,
grid style={dashed},
xmin=0,
xmax=1500,
yticklabel pos=left,
enlargelimits=false,
tick align = outside,
tick style={white},
xticklabel shift={-5pt},
yticklabel shift={-5pt},
ylabel shift={-2pt},
ylabel style={align=center},
legend cell align=left,
legend style={ column sep=1ex },
unbounded coords=jump,
]

\nextgroupplot[ylabel={\footnotesize Latency \\ (Normalized)}, % Performance
xtick={0,100,200,300,400,500,600,700,800,900,1000,1100,1200,1300,1400,1500},
ytick={0.6,0.7,0.8,0.9,1.0},
yticklabels={0.6,0.7,0.8,0.9,1.0},
yticklabel style={font=\footnotesize},
ymin=.6,
ymax=1.0,
legend entries={{Vaio},{ODROID}},
legend style={draw=none,at={(0.5,1.4)},anchor=north,legend columns=4,line width=5pt,font=\footnotesize},
]
\addplot[thick, solid, color=s1] table[x index=0,y index=1,col sep=tab] {img/poet/x264-phases-default-vaio.txt};
\addplot[thick, solid, color=s3] table[x index=0,y index=1,col sep=tab] {img/poet/x264-phases-default-odroid.txt};
\addplot[thick, dashed, black] coordinates {(500,0) (500, 1)};
\addplot[thick, dashed, black] coordinates {(1000,0) (1000, 1)};


\nextgroupplot[ylabel={\footnotesize Power \\ (Watts)}, % Power
ytick={6.0,7.0,8.0,9.0},
yticklabels={6.0,7.0,8.0,9.0},
yticklabel style={font=\footnotesize},
ymin=6,
ymax=9,
xlabel={\footnotesize $time$ [frame]},
xlabel near ticks,
xtick={0,100,200,300,400,500,600,700,800,900,1000,1100,1200,1300,1400,1500},
xticklabels={0,,,,,500,,,,,1000,,,,,1500},
xticklabel style={font=\footnotesize},
]
\addplot[thick, solid, color=s1] table[x index=0,y index=2,col sep=tab] {img/poet/x264-phases-default-vaio.txt};
\addplot[thick, solid, color=s3] table[x index=0,y index=2,col sep=tab] {img/poet/x264-phases-default-odroid.txt};
\addplot[thick, dashed, black] coordinates {(500,0) (500, 11)};
\addplot[thick, dashed, black] coordinates {(1000,0) (1000, 11)};

\end{groupplot}

\end{tikzpicture}
  \caption{Processing \app{x264} input with distinct phases on a embedded systems.}
  \label{fig:poet-embedded-phases-default}
\end{figure}
\begin{figure}[t]
  \centering
  \begin{tikzpicture}

\definecolor{s1}{RGB}{228, 26, 28}
\definecolor{s2}{RGB}{55, 126, 184}
\definecolor{s3}{RGB}{77, 175, 74}
\definecolor{s4}{RGB}{152, 78, 163}
\definecolor{s5}{RGB}{255, 127, 0}

\begin{groupplot}[
    group style={
        group name=plots,
        group size=1 by 2,
        xlabels at=edge bottom,
        xticklabels at=edge bottom,
        vertical sep=5pt
    },
height=3.5cm,
width=0.95\columnwidth,
xmajorgrids,
ymajorgrids,
grid style={dashed},
xmin=0,
xmax=1500,
yticklabel pos=left,
enlargelimits=false,
tick align = outside,
tick style={white},
xticklabel shift={-5pt},
yticklabel shift={-5pt},
ylabel shift={-2pt},
ylabel style={align=center},
unbounded coords=jump,
]

\nextgroupplot[ylabel={\footnotesize Latency \\ (Normalized)}, % Performance
xtick={0,100,200,300,400,500,600,700,800,900,1000,1100,1200,1300,1400,1500},
ytick={0.6,0.8,1.0,1.2,1.4},
yticklabels={,0.8,1.0,1.2,1.4},
yticklabel style={font=\footnotesize},
ymin=.6,
ymax=1.4,
legend entries={{Vaio},{ODROID}},
legend style={draw=none,at={(0.5,1.4)},anchor=north,legend columns=4,line width=5pt,font=\footnotesize},
]
\addplot[thick, solid, color=s1] table[x index=0,y index=1,col sep=tab] {img/poet/x264-phases-poet-vaio.txt};
\addplot[thick, solid, color=s3] table[x index=0,y index=1,col sep=tab] {img/poet/x264-phases-poet-odroid.txt};
\addplot[thick, solid, black] coordinates {(0, 1) (1500, 1)};
\addplot[thick, dashed, black] coordinates {(500,0) (500, 1.4)};
\addplot[thick, dashed, black] coordinates {(1000,0) (1000, 1.4)};


\nextgroupplot[ylabel={\footnotesize Power \\ (Watts)}, % Power
ytick={2.0,4.0,6.0,8.0},
yticklabels={2.0,4.0,6.0,8.0},
yticklabel style={font=\footnotesize},
ymin=2,
ymax=8,
xlabel={\footnotesize $time$ [frame]},
xlabel near ticks,
xtick={0,100,200,300,400,500,600,700,800,900,1000,1100,1200,1300,1400,1500},
xticklabels={0,,,,,500,,,,,1000,,,,,1500},
xticklabel style={font=\footnotesize},
]
\addplot[thick, solid, color=s1] table[x index=0,y index=2,col sep=tab] {img/poet/x264-phases-poet-vaio.txt};
\addplot[thick, solid, color=s3] table[x index=0,y index=2,col sep=tab] {img/poet/x264-phases-poet-odroid.txt};
\addplot[thick, dashed, black] coordinates {(500,0) (500, 11)};
\addplot[thick, dashed, black] coordinates {(1000,0) (1000, 11)};

\end{groupplot}

\end{tikzpicture}    
  \caption{POET behavior for \app{x264} input with distinct phases on embedded systems.}
  \label{fig:poet-embedded-phases-x264}
\end{figure}

The phases are clearly distinguishable by the change in latency at frames 500 and 1000.
The two embedded systems do not process each phase with the same relative latency.
The first phase is the most difficult (highest latency) for both systems, but the second phase is the easiest (lowest latency) for the Vaio, while the third one is the easiest for the ODROID.

We enable POET using the maximum measured latency identified in the first experiment as the target.
\figref{poet-embedded-phases-x264} demonstrates that POET is able to meet the latency target on both systems.
Dips and spikes are visible at the beginning of each phase, showing the change in input behavior and POET adapting to the change.
Despite these variations, latency goals are respected: MAPE is 2.2\% on the Vaio and 2.0\% on the ODROID.
At the same time, energy is near-minimal over the course of execution: energy is 1.7\% greater than optimal on the Vaio and 3.6\% over optimal on the ODROID.


\subsection{Adapting to Other Applications}

\begin{figure}[t]
  \centering
  \begin{tikzpicture}

\definecolor{s1}{RGB}{228, 26, 28}
\definecolor{s2}{RGB}{55, 126, 184}
\definecolor{s3}{RGB}{77, 175, 74}
\definecolor{s4}{RGB}{152, 78, 163}
\definecolor{s5}{RGB}{255, 127, 0}

\begin{groupplot}[
    group style={
        group name=plots,
        group size=1 by 2,
        xlabels at=edge bottom,
        xticklabels at=edge bottom,
        vertical sep=5pt
    },
height=3.5cm,
width=0.95\columnwidth,
xmajorgrids,
ymajorgrids,
grid style={dashed},
xmin=1,
xmax=250,
yticklabel pos=left,
enlargelimits=false,
tick align = outside,
tick style={white},
xticklabel shift={-5pt},
yticklabel shift={-5pt},
ylabel shift={-2pt},
ylabel style={align=center},
legend cell align=left,
legend style={ column sep=1ex },
unbounded coords=jump,
]

\nextgroupplot[ylabel={\footnotesize Vaio \\ Latency}, % Performance
ytick={0.4,0.6,0.8,1.0,1.2,1.4},
yticklabels={0.4,,0.8,,1.2,},
yticklabel style={font=\footnotesize},
ymin=.4,
ymax=1.4,
legend entries={{POET}, {static~allocation}},
legend style={draw=none,at={(0.5,1.4)},anchor=north,legend columns=4,line width=5pt,font=\footnotesize},
]
\addplot[thick, solid, color=s2] table[x index=0,y index=2,col sep=tab] {img/poet/multiapp.txt};
\addplot[thick, dashed, color=red] table[x index=0,y index=1,col sep=tab] {img/poet/multiapp.txt};
\addplot[thick,solid, color=black] coordinates {(133,0) (133,1.4)};
\addplot[thick,solid, color=black] coordinates {(0,1) (250,1)};


\nextgroupplot[ylabel={\footnotesize ODROID  \\ Latency}, 
ytick={0.4,0.6,0.8,1.0,1.2,1.4},
yticklabels={0.4,,0.8,,1.2,},
yticklabel style={font=\footnotesize},
ymin=.4,
ymax=1.4,
xlabel={$time$ [frame]},
xlabel near ticks,
xtick={1,50,100,150,200,250},
xticklabels={1,50,100,150,200,250},
xticklabel style={font=\footnotesize},
]
\addplot[thick,solid, color=black] coordinates {(133,0) (133,1.4)};
\addplot[thick,solid, color=black] coordinates {(0,1) (250,1)};
\addplot[thick, dashed, color=red] table[x index=0,y index=5,col sep=tab] {img/poet/multiapp.txt};
\addplot[thick, solid, color=s2] table[x index=0,y index=6,col sep=tab] {img/poet/multiapp.txt};

\end{groupplot}

\end{tikzpicture}
    
  \caption{POET adapting to a background application on embedded systems.}
  \label{fig:poet-embedded-multiapp}
\end{figure}

This test evaluates POET's behavior when other applications are present in the system, using an external load that is not under POET's direct control.
We launch a POET-enabled application (\app{bodytrack}) with a target latency, then halfway through its execution, we launch another application.
This second application consumes resources, slowing down the POET-enabled application.
POET then assigns more resources to its own application so that it continues to meet its latency target.

\figref{poet-embedded-multiapp} demonstrates POET's behavior on the two embedded platforms.
The thick vertical lines indicate when the second application is launched.
On both systems, we see the latency temporarily increase before POET adjusts the resource allocation.
The charts also show a static resource scheduling approach that selects the application's system resources at the beginning of the execution.
In the static case, the introduction of the new application dramatically increases the job latency, but it is not able to adapt like POET.
On the Vaio, POET's MAPE is 2.3\% over the entire execution (including the period of adjustment to the new load), while the static case has a MAPE of 16\%.
On the ODROID, POET achieves 2.4\% MAPE, while the static case is 12\%.
