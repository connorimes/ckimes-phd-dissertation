\section{Implementation}
\label{sec:poet-implementation}

We now describe how the POET framework is implemented as a C library.
We specify the information that must be provided by POET's users, describe the library interface, and then discuss the implementation of the runtime engine.

\subsection{External Inputs}

POET requires three pieces of information from users.
First, it needs the \textbf{resource specification}---system configurations and their associated speedup and powerup (timing and power) characteristics.
Second, it needs \emph{performance feedback}---timing and power measurements during runtime.
Third, it requires the application to specify its \emph{performance goal}.

The first input is the specification of available system configurations.
POET separates the system configurations into two data structures.
The first is system-agnostic and contains a configuration identifier along with \textbf{speedup} and \textbf{powerup} values.
The second is system-specific and can take any form a developer considers appropriate to define a system configuration.
POET includes a default format for this second structure which contains a configuration identifier, the DVFS frequency to apply, and the number of cores to use.
Snippets of actual configuration files representing both these data structures are presented in \figref{config-examples}.
\secref{poet-usage} describes how to characterize a system's timing and power behavior.
These results can be used to create configurations if they are not already available.

\begin{figure}[t]
\centering
\begin{minipage}{.45\columnwidth}
\lstset{
  belowskip=0pt,
  aboveskip=0pt,
}
\begin{lstlisting}[frame=tlr,%
  caption={System-agnostic.},%
  label={lst:control_config_example}]%

#id        speedup         powerup
0          1               1
1          1.20            1.09
2          1.40            1.16
3          1.60            1.30
4          2.12            1.35
5          2.53            1.50
6          2.88            1.64
7          3.18            1.69
\end{lstlisting}
\end{minipage}
% \hfill
\hspace*{0.4cm}
\begin{minipage}{.45\columnwidth}
\lstset{
  belowskip=0pt,
  aboveskip=0pt,
}
\begin{lstlisting}[frame=tlr,%
  caption={System-specific.},%
  label={lst:cpu_config_example}]%

#id        frequency        cores
0          250000           0
1          300000           0
2          350000           0
3          400000           0
4          250000           1
5          300000           1
6          350000           1
7          250000           2
\end{lstlisting}
\end{minipage}
% \vskip -1.5em
\caption{Example of POET configuration files.}
\label{fig:config-examples}
\end{figure}

To measure both performance and power, we modify the Heartbeats API \cite{PTRADE} to record power data along with timing statistics (see also \appsref{profiling}{energymon}).
We then modify applications to issue heartbeats at appropriate points during processing, typically after the completion of every job.
POET can then use this captured timing and power data at runtime.

The third necessary input is a performance target.
Performance goals can change during runtime, and POET automatically adapts.


\subsection{Software Interface}
\label{sec:poet-interface}

Note that this interface described here has changed slightly from the original POET publication, primarily to decouple POET and the Heartbeats API \cite{POET}.
Users typically interact with only three POET functions.
\function{poet\_init} initializes POET and returns a \struct{poet\_state} data structure reference, which stores the control state required to implement the framework described in \secref{poet-framework}.
\function{poet\_apply\_control} executes the controller, runs \algoref{poet-optimal}, and configures the platform.
\function{poet\_destroy} cleans up the \struct{poet\_state} data structure.

The \function{poet\_init} function takes as parameters: the performance goal, the window period over which the controller operates, the system's configurations, and a reference to the function that applies the given system configurations.
It also receives an optional reference to the function that determines the system's current state and a log file name.
The first system configuration data structure (system-agnostic) is of type \variable{poet\_control\_state\_t}, and the second (system-specific) has type \variable{void}.
The two functions passed by reference are the only ones that need to know the format of the second data structure and are therefore passed the \struct{void} type reference given to \function{poet\_init} as parameters.
The first of these two functions must have a signature that matches the \variable{poet\_apply\_func} definition and the second must match the \struct{poet\_curr\_state\_func} definition.

The \function{poet\_apply\_control} function takes as parameters: the \struct{poet\_state} reference, a unique identifier (for logging purposes), and the measured performance and power values (\eg as recorded by the Heartbeats API).
The \function{poet\_destroy} function just takes the \struct{poet\_state} reference as a parameter.
An additional function, \function{poet\_set\_performance\_goal}, allows changing the performance goal at runtime.

A separate header exposes auxiliary functions to load system configurations from files, discover the initial system configuration, and apply system configurations.
These are for the default file formats and system knobs described in the previous section.
The latter two of these functions meet the \struct{poet\_curr\_state\_func} and \struct{poet\_apply\_func} definitions, respectively, and can be passed to \function{poet\_init}.
These auxiliary functions are platform-dependent and thus kept separate to maintain portability, allowing users to easily substitute their own versions.
They are, however, generic enough that most Linux users do not need to write their own unless they want to tune different knobs.


\subsection{Runtime}

After an application measures its performance and power, \eg with a heartbeat, it makes a call to \function{poet\_apply\_control}, which contains POET's core logic.
When a window period completes, the POET runtime calculates the estimated base speed with \eqnref{kalman-filter}, then computes the performance error with \eqnref{poet-error}.
It subsequently computes the control signal from \eqnref{poet-control} to determine the speedup necessary to eliminate the error.
Finally it determines the energy-minimal resource schedule that achieves the necessary speedup using \algoref{poet-optimal} and applies the first configuration in the schedule by executing the provided \function{poet\_apply\_func} function.

The last step, a call to the \variable{poet\_apply\_func} specified by the user, is a platform-dependent operation.
For example, the function included with POET for Linux platforms invokes the \app{taskset} utility to force the process and all of its threads onto the desired number of cores and uses the \app{cpufrequtils} interface to adjust the cores' DVFS frequencies.
Developers can specify their own function---for example, a system may require a different approach to change the number of active cores, or a system may have additional configurable resources that a user wants to tune, like memory and network bandwidth.

The only remaining task is to apply the second configuration in the schedule at the appropriate time during the next window period.
When the computed subset of jobs in the window period are complete ($\omega_u$ in \algoref{poet-optimal}), the \variable{poet\_apply\_func} function executes again, but no further computation is performed.
When window period completes, the control process repeats.

The code snippet in \lstref{poet-example} is an example of application code, highlighting the POET function calls.
The complete modification of an existing application requires only a handful of function calls plus associated variable declarations.
The min and max heartrate, accuracy, and power values in the Heartbeats initialization are not used, so they can safely be set to any value.
When initializing POET, the user should specify the system's configurations, encoded in \variable{control\_states} and \variable{cpu\_states}.

\lstset{emph={%  
    poet_state, poet_init, poet_apply_control, poet_destroy%
    },emphstyle={\color{black}\bfseries\underbar}%
}%
\begin{lstlisting}[language=C,%
  caption={Example of POET application code.},%
  label={lst:poet-example}]%

// initialization
heartbeat_t* hb =
  heartbeat_acc_pow_init(window_size, log_buffer_depth, "heartbeat.log",
                         min_heartrate, max_heartrate, min_accuracy, max_accuracy,
                         1, hb_energy_impl_alloc(), min_power, max_power);
get_control_states(NULL, &control_states, &nstates);
get_cpu_states(NULL, &cpu_states, &nstates);
poet_state* state = poet_init(perf_goal, nstates, control_states, cpu_states,
                              &apply_cpu_config, &get_current_cpu_state,
                              window_size, log_buffer_depth, "poet.log");
// execution of main loop
while(running) {
  heartbeat_acc(hb, count++, 1);
  poet_apply_control(state, count, hb_get_windowed_rate(hb), hb_get_windowed_power(hb));
  doWork();
}
// cleanup
poet_destroy(state);
free(control_states);
free(cpu_states);
heartbeat_finish(hb);
\end{lstlisting}
