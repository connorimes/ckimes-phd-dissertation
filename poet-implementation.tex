\section{Implementation}
\label{sec:poet-implementation}

We describe how the framework in the previous section is realized in a C library.
We specify the information that must be provided by POET's users, describe the library interface, and then discuss the implementation of the runtime engine.

\subsection{POET's External Inputs}

POET requires three pieces of information from users.
First, it needs the available system configurations and their associated timing and power characteristics.
Second, it needs a means to measure timing and power consumption during runtime.
Third, it requires the application to specify the desired latency target.

The first input is the specification of available system configurations.
POET separates the system configurations into two data structures.
The first is system-agnostic and contains a configuration identifier along with \textbf{speedup} and \textbf{powerup} values.
The second is system-specific and can take any form a developer considers appropriate to define a system configuration.
To simplify the programmer's job, POET includes a default format for this second structure which contains a configuration identifier, the DVFS frequency to apply, and the number of cores to use.
Snippets of actual configuration files representing both these data structures are presented in \figref{config-examples}.
\secref{poet-usage} describes how to characterize a system's timing and power behavior.
These results can be used to create configurations if they are not already available.

\begin{figure}[t]
\begin{minipage}[t]{.45\columnwidth}
\lstset{
  belowskip=0pt,
  aboveskip=0pt,
}
\begin{lstlisting}[frame=tlr,%
  %caption={System-agnostic.},%
  label={lst:control_config_example}]%

#id        speedup         powerup
0          1               1
1          1.20            1.09
2          1.40            1.16
3          1.60            1.30
4          2.12            1.35
5          2.53            1.50
6          2.88            1.64
7          3.18            1.69
\end{lstlisting}
\end{minipage}\hfill
\begin{minipage}[t]{.45\columnwidth}
\lstset{
  belowskip=0pt,
  aboveskip=0pt,
}
\begin{lstlisting}[frame=tlr,%
  %caption={System-specific.},%
  label={lst:cpu_config_example}]%

#id        frequency        cores
0          250000           0
1          300000           0
2          350000           0
3          400000           0
4          250000           1
5          300000           1
6          350000           1
7          250000           2
\end{lstlisting}
\end{minipage}
\vskip -1.5em
\caption{Example of POET system-agnostic (left) and system-specific (right) configuration files.}
\label{fig:config-examples}
\end{figure}

To measure both latency and power, we modify the Heartbeats API \cite{PTRADE} to record power data along with timing statistics.
We then modify applications to issue heartbeats at appropriate points during processing, typically after the completion of every job.
The issued heartbeats contain power and timing data that POET queries at runtime.

The third necessary input is a latency target.
The user provides the target through the Heartbeats API, specifying a minimum and maximum latency goal.
The timing targets can change during runtime, and POET will take care of the adaptation automatically.


\subsection{POET's Interface}

Users interact with only three POET functions.
\function{poet\_init} initializes POET and returns a \variable{poet\_state} data structure reference.
\function{poet\_apply\_control} executes the controller, runs \algoref{poet-optimal}, and configures the platform.
\function{poet\_destroy} cleans up the \variable{poet\_state} data structure.

POET's initialization function takes, as parameters, references to: the heartbeat data structure used to store the timing and power of the application, the system's configurations, and the function that applies the given system configurations.
It also receives an optional reference to the function that determines the system's current state and a log file name.
The first system configuration data structure (system-agnostic) is of type \variable{poet\_control\_state\_t}, and the second (system-specific) has type \variable{void}.

The two functions passed by reference are the only ones that need to know the format of the second data structure and are therefore passed the \variable{void} type reference given to \function{poet\_init} as parameters.
The first of these two functions must have a signature that matches the \variable{poet\_apply\_func} definition and the second must match the \variable{poet\_curr\_state\_func} definition.

The other two API functions, \function{poet\_apply\_control} and \function{poet\_destroy}, take the \variable{poet\_state} reference as their only parameter.
This variable contains all the control state required to implement the framework described in \secref{poet-framework}.

Auxiliary functions are also provided to load system configurations from files, discover the initial system configuration, and apply system configurations.
The latter two of these meet the \variable{poet\_curr\_state\_func} and \variable{poet\_apply\_func} definitions, respectively, and can be passed to \function{poet\_init}.
These auxiliary functions are platform-dependent and thus kept separate to maintain portability, allowing users to easily substitute their own versions.
They are, however, generic enough that most Linux users do not need to write their own.


\subsection{POET's Runtime}

After an application signals its power and latency with a heartbeat, it makes a call to \function{poet\_apply\_control}, which contains POET's core logic.
Heartbeats are initialized with a {\em window size} value that indicates the number of jobs to complete in a given {\em time interval}.
The window size is analogous to $I(t)$ from \eqnref{poet-work}, while the time interval is equivalent to $\tau$ from \eqnref{poet-time} and \algoref{poet-optimal}.
At the completion of a window, the POET runtime calculates the estimated base speed with \eqnref{kalman-filter}, then computes the latency error with \eqnref{poet-error}.
It subsequently applies the controller of \eqnref{poet-control} to determine the speedup necessary to eliminate the computed error.
Finally it determines the energy-minimal resource schedule that achieves the necessary speedup using \algoref{poet-optimal} and applies the first configuration in the schedule by executing the provided \variable{poet\_apply\_func} function.

The last step, a call to the \variable{poet\_apply\_func} specified by the user, is a platform-dependent operation.
For example, the function included with POET for Linux platforms invokes the \mbox{\texttt{taskset}} utility to force the process and all of its threads onto the desired number of cores and uses the \mbox{\texttt{cpufrequtils}} interface to adjust the cores' clock speeds.
Developers can specify their own function.
For example, a system may require a different approach to change the number of active cores.
Also, a system may have additional configurable resources that could be adjusted, like memory and network bandwidth.

The only remaining task is to apply the second configuration in the schedule at the appropriate time during the next window.
When the computed number of heartbeats to wait passes, the \variable{poet\_apply\_func} function executes again, but no further computation is performed.
At the completion of the heartbeat window, the control process repeats.
