\chapter{POET: The Performance with Optimal Energy Toolkit}
\label{sec:poet}

This chapter describes POET (the \textbf{P}erformance with \textbf{O}ptimal \textbf{E}nergy \textbf{T}oolkit).
\secref{related} discussed existing work in managing performance and power/energy awareness.
Like POET, a number of these approaches use feedback control to manage timing constraints \cite{Agilos,Bertini2007,grace2,CoAdapt,Maggio13,TCST,ControlWare,PTRADE,Fu2012,Li2007}.
POET is most related to prior approaches that abstract resource management into a middleware or runtime, like \cite{ControlWare,Sojka,PTRADE,Rajkumar}.
Control techniques provide a formal framework for reasoning about the dynamic behavior of the system.
POET is unique in its energy awareness, in its design for portability, and the incorporation of a true minimal-energy resource allocation algorithm.

\TODO{Combine RTAS and MCSoC papers, including motivation, usage, and evaluation.}

\section{Motivation}
\label{sec:poet-motivation}

To motivate the need for POET, we summarize prior work \cite{Imes2014}.
We evaluate timing and energy tradeoffs on two embedded platforms: a Sony Vaio SVT11226CXB tablet system with an Intel Haswell processor and an ODROID-XU+E ARM big.LITTLE development board.
The two platforms have: (1) different configurable resources for performance/power management, and (2) timing/energy tradeoffs with different topologies.
Resource allocation strategies that save energy on one are wasteful on the other.

\tblref{poet-machines} presents each system's configurable resources.
The Vaio allows configuring the number of active cores, the use of hyperthreads, processor clock speed, and the use of TurboBoost.
The ODROID supports configuring the number of active cores, whether the application uses the ``big'' (Cortex-A15 high-performance, high-power) or ``LITTLE'' (Cortex-A7 low-performance, low-power) cores, and the independent clock speeds of the big and LITTLE clusters.

\begin{table}[t]
\caption{Two embedded platforms with different configurable components.}
\label{tbl:poet-machines}
\tiny
\centering
\begin{tabular}{lcccccccc}
  \textbf{Platform} & 
  \textbf{Processor} &
  \textbf{Cores} & 
  \textbf{Core Types} &
  \textbf{Speeds (GHz)} &
  \textbf{TurboBoost} &
  \textbf{HyperThreads} & 
  \textbf{Num. Configs} \\
  % \textbf{Configurations} \\
  \hline
  \hline
  SVT11226CXB & Intel Haswell  & 2 & 1             & 0.6--1.5          & yes & yes & 46 \\
  ODROID-XU+E & ARM big.LITTLE & 8 & 2 (A15, A7)   & 0.8--1.6, 0.5--1.2 & no  & no  & 70 \\
  \hline 
  \hline
\end{tabular}
% \vskip -.7em
\end{table}

Our example features a video encoder, composed of jobs, where each job encodes a frame.
We instrument the encoder to report the latency and platform energy consumption for each job.
\figref{poet-x264-motivation-tradeoffs} shows the tradeoffs between job latency and system energy consumption for each platform, where each point represents the average behavior of a different configuration.
The x-axis shows latency, normalized to 1---the empirically determined worst case.
The y-axis shows energy, normalized to 1---the highest measured energy.
The tradeoffs are obviously very different for the Vaio and the ODROID.
For the Vaio, energy increases as frame latency increases; \ie a slower job wastes energy.
For the ODROID, energy decreases as frame latency increases; \ie slower encodings save energy.

\begin{figure}[t]
  % \vskip -1.8em
  \centering
  \subfloat[Latency/Energy tradeoffs.]%
  {\begin{tikzpicture}

\definecolor{s1}{RGB}{228, 26, 28}
\definecolor{s2}{RGB}{55, 126, 184}
\definecolor{s3}{RGB}{77, 175, 74}
\definecolor{s4}{RGB}{152, 78, 163}
\definecolor{s5}{RGB}{255, 127, 0}

\begin{groupplot}[
    group style={
        group name=plots,
        group size=1 by 1,
        xlabels at=edge bottom,
        xticklabels at=edge bottom,
        vertical sep=5pt
    },
xlabel={\footnotesize Latency},
xlabel near ticks,
height=4.2cm,
%width=0.5\textwidth,
width = 4.6cm,
xmajorgrids,
ymajorgrids,
grid style={dashed},
xtick={0,0.25,0.5,0.75,1.0},
xticklabels={0,0.25,0.5,0.75,1.0},
xticklabel style={font=\footnotesize},
xmin=0,
xmax=1.1,
yticklabel pos=left,
enlargelimits=false,
tick align = outside,
tick style={white},
xticklabel shift={-5pt},
yticklabel shift={-5pt},
ylabel shift={-2pt},
ylabel style={align=center},
unbounded coords=jump,
]

\nextgroupplot[ylabel={\footnotesize Energy}, 
%ylabel shift={6mm},
ytick={0,0.25,0.5,0.75,1.0},
yticklabels={0,0.25,0.5,0.75,1.0},
yticklabel style={font=\footnotesize},
ymin=0,
ymax=1.1,
legend entries={{Vaio},{ODROID}},
legend style={draw=none,at={(0.5,1.35)},anchor=north,legend columns=4,line width=5pt,font=\footnotesize},
]
\addplot[thick, solid, color=s1, only marks, mark=square*] table[x index=2,y index=5,col sep=tab] {img/poet/tradeoffs-vaio.txt};
\addplot[thick, solid, color=s3, only marks, mark=*] table[x index=2,y index=5,col sep=tab] {img/poet/tradeoffs-odroid.txt};


\end{groupplot}

\end{tikzpicture}
%
  \label{fig:poet-x264-motivation-tradeoffs}}
  \subfloat[Heuristic energy consumption.]%
  {\begin{tikzpicture}

\begin{groupplot}[
    group style={
        group name=plots,
        group size=1 by 1,
        xlabels at=edge bottom,
        xticklabels at=edge bottom,
        vertical sep=5pt
    },
% axis x line* = bottom,
xlabel near ticks,
major x tick style = transparent,
height=4.2cm,
%width=0.95\columnwidth,
width=4.6cm,
xmin=0,
xmax=3,
enlargelimits=false,
tick align = outside,
tick style={white},
ytick=\empty,
xticklabel shift={-5pt},
%x tick label style={rotate=0, anchor=south},
xlabel={\footnotesize Platform},xtick={1,2,3},
xticklabels={
{\scriptsize Vaio},
{\scriptsize ODROID}
},
ymin=1,
ymax=2.25,
ytick={1.0,1.25,1.5,1.75,2.0},
yticklabels={1.00,1.25,1.50,1.75,2.00,2.25},
legend cell align=left, 
legend style={ column sep=1ex },
ymajorgrids,
grid style={dashed},
]
\nextgroupplot[ylabel={\footnotesize Energy},
ybar=\pgflinewidth,
bar width=8pt,
legend entries = {{never-idle},{race-to-idle}},
legend style={draw=none,legend columns=2,at={(0.5,1.35)},anchor=north,font=\footnotesize},
]
\addplot table[x index=0,y index=2, col sep=space] {img/poet/heuristics2.txt};
\addplot table[x index=0,y index=3, col sep=space] {img/poet/heuristics2.txt};


\end{groupplot}

\end{tikzpicture}
%
  \label{fig:poet-x264-motivation-heuristics}}
  \caption{Timing and energy behavior for encoding video on the Vaio and ODROID.}
  \label{fig:poet-x264-motivation}
\end{figure}

The different shapes of these tradeoff spaces lead to different optimal resource allocation strategies.
Empirical studies show that the \emph{race-to-idle} heuristic, which makes all resources available and then idles after completing a job, is near-optimal on systems like the Vaio~\cite{google,Hoelzle2009,HotPower,Imes2014,PowerSlope}.
On systems like the ODROID, recent approaches save energy by keeping the system constantly busy, like the \emph{never-idle} heuristic~\cite{Carroll13,HotPower,Imes2014,LeSueur11,Lin2010}.

To demonstrate the importance of choosing the right strategy, we analyze the two heuristics on both platforms and compare their energy consumption to optimal (found by measuring every possible resource configuration).
We set a latency target equal to twice the minimum latency and measure the energy consumption of encoding 500 video frames using each heuristic.
\figref{poet-x264-motivation-heuristics} shows the results, normalized to optimal.
Both heuristics meet the latency target, but their energy consumptions vary tremendously.
On the Vaio, \emph{race-to-idle} is near-optimal, but \emph{never-idle} consumes 13\% more energy.
Conversely, \emph{never-idle} is near-optimal for the ODROID, but \emph{race-to-idle} consumes $2 \times$ more energy.

These results demonstrate that resource allocation strategy greatly affects energy consumption, and more importantly, that heuristic solutions are not portable across systems.
These two points motivate the need for an approach like POET, which provides near-optimal resource allocation while remaining platform-independent.
% POET's runtime uses control theory to meet timing constraints and linear programming to minimize energy consumption.
% A POET user does not need to be a control or optimization expert, but simply make small changes to their application code.
% POET makes it easy for embedded developers to write portable applications providing predictable timing and minimal energy across a range of devices.

\section{General and Portable Resource Allocation}
\label{sec:poet-framework}

The goal of POET's resource allocation framework is twofold.
First, it must provide predictable timing so application jobs meet their deadlines.
Second, it should minimize energy consumption given the timing requirement.
While these two subproblems are intrinsically connected, they can be decoupled to provide a general solution.
The complexity arises from the need to keep resource allocation general with respect to the running application and the hardware platform.
We tackle the problem of providing predictable timing using control theory by computing a \emph{generic control signal}.
Using the computed control signal, we then solve the energy minimization problem using mathematical optimization.

\figref{poet-runtime} illustrates our approach.
The \textbf{application} informs the runtime of its job \emph{performance goal}.
Measuring each job start and completion time (\emph{performance feedback}), POET's runtime computes the \emph{performance error} and passes it to the \textbf{controller}.
The controller uses the error to calculate a \emph{generic control signal}, indicating how much the application speed should be altered.
This signal is used by the \textbf{optimizer}, together with the \textbf{resource specification}, to produce a \emph{resource schedule} that achieves the desired performance goal while minimizing energy consumption.
Both the controller and the optimizer are designed independently of any particular application and system.
The only assumption made is that applications are composed of repeated jobs with a soft real-time performance goal.
% As we target multicore platforms, we assume each job may be processed by multiple, communicating threads.

\begin{figure}[t]
  \centering
  \tikzset{%
  app/.style    = {draw, thin, rectangle, minimum height = 2em,
    minimum width = 2em, fill=black!25},
  block/.style    = {draw, thick, rectangle, minimum height = 2.5em,
    minimum width = 2.5em},
  blockres/.style    = {draw, thick, rectangle, minimum height = 2.5em,
    minimum width = 2.5em, fill=green!25},
  biblock/.style  = {draw, thick, rectangle, minimum height = 5.5em,
    minimum width = 6em, fill=red!25},
  sum/.style      = {draw, circle, node distance = 2cm}, % Adder
  input/.style    = {coordinate}, % Input
  output/.style   = {coordinate} % Output
}

\begin{tikzpicture}[scale=1.0,transform shape, auto, thick, node distance=1.5cm, >=triangle 45]

\draw
  % Drawing the top blocks
  node [input, name=goalaccuracy] {} 
  node [left of=goalaccuracy, node distance=0.35mm]{}
  node [sum, right of=goalaccuracy] (sumaccuracy) {} % negative feedback
  node [block, right of=sumaccuracy, align=center, node distance=3.5cm] (controlaccuracy) 
    {Controller}
  node [block, right of=controlaccuracy, align=center, node distance=4.2cm] (translateaccuracy) 
    {Optimizer}
  node [blockres, above of=translateaccuracy, align=center, node distance=1.8cm] (resourcefile) 
    {Resource\\Specification}
;
  % Connectng lines
\draw[->](goalaccuracy) -- node[align=center] {Performance\\Goal}(sumaccuracy);
\draw[->](sumaccuracy) -- node[align=center] {Performance\\Error}(controlaccuracy);
\draw[->](controlaccuracy) -- node[align=center] {Generic\\Control\\Signal}(translateaccuracy);
\draw[->](resourcefile) -- (translateaccuracy);

% Draw software system
\draw
  node [biblock, right of=translateaccuracy, node distance=4.5cm, align=center] (system)
    {\\System\\\\\\}
;
\draw
  node [app, right of=translateaccuracy, node distance=4.5cm, align=center, yshift=-0.5cm] (software)
    {Application}
;

% lines from translators to software
\draw[->](translateaccuracy.east) -- node [name=ka,align=center]{Resource\\Schedule} (translateaccuracy.east -| system.west);

% Connectng lines
\coordinate (feedbackup) at ([yshift=-0.6cm]sumaccuracy.south);
\draw (software.west |- feedbackup) -| node [near end,align=center] {Performance\\Feedback} (feedbackup);
\draw[->](feedbackup) -- node[pos=0.99] {$-$} (sumaccuracy);

\end{tikzpicture}
  % \vskip -.5em
  \caption{Overview of the POET runtime.}
  \label{fig:poet-runtime}
% \vskip -.3em
\end{figure}


\subsection{Controller}

The application provides a \emph{performance goal} (work rate) $R_{ref}$, which is easily computed from a workload size (number of jobs) $\omega$ and desired latency goal (deadline) $\tau$ for those $\omega$ jobs:
\begin{equation}
  R_{ref} = \frac{\omega}{\tau}
  \label{eqn:poet-latency-to-performance}
\end{equation}
The controller cancels the error between the desired performance, $R_{ref}$, and the measured performance, $r_m(t)$, which it models as:
\begin{equation}
r_m(t) = s(t-1) \cdot b_r(t-1)
\label{eqn:poet-performance}
\end{equation}
The error $e_r(t)$ is then easily computed as:
\begin{equation}
e_r(t) = R_{ref} - r_m(t)
\label{eqn:poet-error}
\end{equation}
The controller performs its calculations at discrete work (job) intervals to produce a new desired speedup, $s(t)$, implementing the \emph{integral control law}~\cite{Hellerstein2004a}:
\begin{equation}
  s(t) = s(t-1) + (1-\alpha) \cdot \frac{e_r(t)}{b_r(t)}
  \label{eqn:poet-control}
\end{equation}
where $\alpha$ is a \emph{pole} of the closed loop characteristic equation~\cite{ICSE2014} such that $\alpha$ lies within the unit circle:
\begin{equation}
  0 \le \alpha < 1
  \label{eqn:pole}
\end{equation}
The pole is configurable.
Small $\alpha$ values make the controller highly reactive, while large values make it slow to respond to external changes.
However, a large $\alpha$ ensures robustness with respect to transient fluctuations and may be beneficial for very noisy systems.
A small $\alpha$ will cause the controller to react quickly, potentially producing a noisy control signal.

The parameter $b_r(t)$ represents the application's base speed, which directly influences the controller.
Different applications will have different base speeds.
Applications may also experience \emph{phases}, where base speed changes over time.
To accommodate these situations, POET continually estimates base speed using a Kalman filter~\cite{welch2006kalman}, which adapts $b_r(t)$ of \eqnref{poet-control} to the current application behavior.
Assuming minimal measurement variance (\ie even if an application is noisy, the signaling framework does not add additional noise) and denoting the application timing variance as $q_b(t)$, the Kalman filter formulation is standard:
\begin{equation}
\left \lbrace
\begin{array}{rcl}
\hat{b}^{-}(t) & = & \hat{b}(t-1) \\
e^{-}_{b}(t) & = & e_{b}(t-1) + q_b(t) \\
k_b(t) 
  & = & \frac{e^{-}_{b}(t) \cdot s(t)}{[s(t)]^2
        \cdot e^{-}_{b}(t)} \\
\hat{b}(t) 
  & = & \hat{b}^{-}(t) + k_b(t) 
        \, \left[ r_m(t) - s(t) \cdot \hat{b}^{-}(t) \right] \\
e_{b}(t) & = & [1 - k_b(t) \cdot s(t-1)] \, e^{-}_{b}(t)
\end{array}
\right .
\label{eqn:kalman-filter}
\end{equation}
This formulates Kalman gain for job latency as $k_b(t)$, the \emph{a priori} and \emph{a posteriori} estimates of the base speed as $\hat{b}^{-}(t)$ and $\hat{b}(t)$, and the \emph{a priori} and \emph{a posteriori} estimates of the error variance as $e^{-}_{b}(t)$ and $e_{b}(t)$.
The Kalman filter produces a statistically optimal estimate of the system's parameters and is provably exponentially convergent~\cite{CaoSchwartz2003}.

Unlike prior work, the POET controller does not reason about a particular set of resources, but computes a generic control signal $s(t)$.
The advantage of using the Kalman filter is that POET's formulation is independent of particular applications and systems.
POET provides formal guarantees about its steady-state convergence and robustness without requiring users to understand control theory or Kalman filtering---$s(t)$ is computed by the controller, $r_m(t)$ and $q_b(t)$ are measured, and all other parameters are derived.


\subsection{Optimizer}
\label{sec:poet-optimizer}

The optimizer turns the generic control signal computed by the controller into a system-specific resource allocation strategy, translating the speedup $s(t)$ computed with \eqnref{poet-control} into a \emph{schedule} for the available resources.
The \emph{schedule} is computed for the next $\omega$ work units (jobs).

As shown in \figref{poet-runtime}, the optimizer takes, as input, a resource specification containing the set of available system configurations.
There are $C$ possible configurations in the system and by convention, we number the configurations from $0$ to $C-1$.
We use $c = 0$ to indicate the configuration where the least amount of resources is given to the application. %, corresponding to a low-power idle state or sleep state when available.
In contrast, configuration $C-1$ maximizes the resource availability.
Each configuration $c$ is associated with performance and power values, speedup $s_c$ and powerup $p_c$ respectively, which are normalized to $c = 0$.

Given this information, POET solves the following optimization problem:
\begin{eqnarray}
\minimize && \sum_{c=0}^{C-1} \tau_c \cdot p_c \label{eqn:poet-power} \\
\st %&& \nonumber\\
&& \sum_{c=0}^{C-1} \tau_c \cdot s_c \cdot b_r(t) =  \omega \label{eqn:poet-work} \\
&& \sum_{c=0}^{C-1} \tau_c =  \tau \label{eqn:poet-deadline} \\
&& 0 \le \tau_c \le \tau, \qquad \forall c \in \{0,\ldots,C-1\} \label{eqn:poet-time}
\end{eqnarray}
\eqnref{poet-power} minimizes the total energy consumption.
\eqnref{poet-work} constrains all jobs to complete in the next control period.
\eqnref{poet-deadline} ensures that the time is fully scheduled and \eqnref{poet-time} imposes that a non-negative time is assigned to each configuration.
Solving linear optimization problems is, in general, hard, but this particular problem has a structure that makes it practical to solve.
Feasible solutions are confined to a polytope in the positive quadrant defined by the two constraints \eqnsref{poet-work}{poet-deadline}.
Thus, linear programming theory states an optimal solution exists for this problem when all the $\tau_c$ are equal to zero except for (at most) two configurations~\cite{LP}.

\begin{algorithm}[t]
  \caption{Finding a Minimal-Energy Schedule.}
  \begin{algorithmic}
    \footnotesize
    \Require $C$ \Comment{system configurations, given by user}
    \Require $\omega$ \Comment{discrete work units, given by application}
    \Require $s(t)$ \Comment{speedup, given by \eqnref{poet-control}}
    \State $U = \{c \mid s_c \le s(t) \}$
    \State $O = \{c \mid s_c > s(t)\}$
    \State $candidates = U \times O = \{\langle u, o \rangle \mid u \in U, o \in O\}$
    \State $energy = \infty$
    \State $optimal = \langle -1, -1 \rangle$
    \State $schedule = \langle -1, -1 \rangle$ \newline
    \For {$\langle u, o \rangle \in candidates$} \Comment{loop over all pairs}
    \State $\omega_u = \omega \cdot \frac{s_u \cdot (s_o - s(t))}{s(t) \cdot (s_o - s_u)}$ \Comment{compute the work units to spend in each configuration in pair}
    \State $\omega_o = \omega - \omega_u$
    \State $newEnergy = \omega_u \cdot p_u + \omega_o \cdot p_o$ \Comment{compute energy of this pair}
    \If {$newEnergy < energy$} \Comment{compare energy to best found so far}
    \State $energy = newEnergy$
    \State $optimal = \langle u, o \rangle$
    \State $schedule = \langle \omega_u,\omega_o \rangle$
    \EndIf
    \EndFor \newline \newline
    \Return $optimal$ \Comment{pair of configurations with minimal energy} \newline
    \Return $schedule$ \Comment{work units to spend in each configuration}
  \end{algorithmic}
  \label{algo:poet-optimal}
\end{algorithm}

\algoref{poet-optimal} takes the set of configurations $C$, the controller's speedup $s(t)$, and the number of work units $\omega$ in a control period.
It then divides the configurations in two distinct subsets.
The first subset contains all configurations with a speedup less than or equal to the target.
The second contains the remaining configurations, \ie those with speedups greater than required.
Subsequently, \algoref{poet-optimal} loops over all feasible pairs of configurations, with one from each subset, to determine how much time should be spent in each configuration given speedup constraint.
If the energy of the pair is lower than any previous energy, the algorithm stores the current best pair, its energy, and its schedule.
When the algorithm terminates, its output is the pair of chosen configurations and their assigned times.
The algorithm tests all possible pairs from the two subsets, each of which contains at most $C$ elements, so an upper bound to the algorithm complexity is $O(C^2)$.
We know that there is an optimal solution to the linear program with at most two non-zero $\tau_c$ (as the dual problem has two dimensions \cite{LP}).
Therefore, \algoref{poet-optimal} will find a minimal-energy schedule.


\subsection{Generality and Robustness}

The controller and the optimizer both reason about speedup instead of absolute performance or latency.
The absolute performance of the application, measured by the average latency of its jobs, will vary as a function of the application itself and the platform it executes on.
However, speedup is a general concept and can be applied to any application and system, providing a more general metric for control.
Moreover, the controller customizes the behavior of a specific application using the estimate of its base speed produced by the Kalman filter
The optimizer operates in a platform-independent manner, using the available configurations provided as input to find the optimal solution, without relying on a particular heuristic that may be system-specific or application-dependent.
Finally, the customizable pole $\alpha$ in \eqnref{poet-control} allows for flexibility and robustness to inaccuracies and noise.

The ability to control robustness to inaccuracies and model errors is a major advantage of feedback control systems~\cite{ICSE2014}.
In particular, POET is stable and converges to the desired latency without oscillations provided that $0 \le \alpha < 1$.
Formal analysis of this behavior can be obtained by applying standard control techniques---see the original POET publication for further details \cite{POET}.

In addition to provable convergence, the control formulation allows us to analyze POET's robustness to user error.
In particular, suppose $\Delta$ is a multiplicative error term, indicating the largest error in the speedup values provided in the system configurations.
That is, if the provided speedup is $s_p$, the real value is $s_p \cdot \Delta$.
POET cancels the error despite inaccurate information if and only if $0 < \Delta < \frac{2}{1-\alpha}$.
The value of $\alpha$ therefore determines how robust POET is to errors in speedup specifications.
For example, when $\alpha = 0.1$, $s_p$ can be off by a factor of $2$ and the system is still guaranteed to converge.
Users who can provide good system models will therefore use a small $\alpha$, while less confident users can select a larger $\alpha$.
All the experiments in our evaluation use $\alpha=0$ to test our implementation in the least forgiving setting.
A detailed analysis of POET's robustness is presented in the original POET publication \cite{POET}.

\section{Implementation}
\label{sec:poet-implementation}

We describe how the framework in the previous section is realized in a C library.
We specify the information that must be provided by POET's users, describe the library interface, and then discuss the implementation of the runtime engine.

\subsection{External Inputs}

POET requires three pieces of information from users.
First, it needs the available system configurations and their associated timing and power characteristics.
Second, it needs a means to measure timing and power consumption during runtime.
Third, it requires the application to specify the desired latency target.

The first input is the specification of available system configurations.
POET separates the system configurations into two data structures.
The first is system-agnostic and contains a configuration identifier along with \textbf{speedup} and \textbf{powerup} values.
The second is system-specific and can take any form a developer considers appropriate to define a system configuration.
To simplify the programmer's job, POET includes a default format for this second structure which contains a configuration identifier, the DVFS frequency to apply, and the number of cores to use.
Snippets of actual configuration files representing both these data structures are presented in \figref{config-examples}.
\secref{poet-usage} describes how to characterize a system's timing and power behavior.
These results can be used to create configurations if they are not already available.

\begin{figure}[t]
\centering
\begin{minipage}{.45\columnwidth}
\lstset{
  belowskip=0pt,
  aboveskip=0pt,
}
\begin{lstlisting}[frame=tlr,%
  caption={System-agnostic.},%
  label={lst:control_config_example}]%

#id        speedup         powerup
0          1               1
1          1.20            1.09
2          1.40            1.16
3          1.60            1.30
4          2.12            1.35
5          2.53            1.50
6          2.88            1.64
7          3.18            1.69
\end{lstlisting}
\end{minipage}
% \hfill
\hspace*{0.4cm}
\begin{minipage}{.45\columnwidth}
\lstset{
  belowskip=0pt,
  aboveskip=0pt,
}
\begin{lstlisting}[frame=tlr,%
  caption={System-specific.},%
  label={lst:cpu_config_example}]%

#id        frequency        cores
0          250000           0
1          300000           0
2          350000           0
3          400000           0
4          250000           1
5          300000           1
6          350000           1
7          250000           2
\end{lstlisting}
\end{minipage}
% \vskip -1.5em
\caption{Example of POET configuration files.}
\label{fig:config-examples}
\end{figure}

To measure both latency and power, we modify the Heartbeats API \cite{PTRADE} to record power data along with timing statistics.
We then modify applications to issue heartbeats at appropriate points during processing, typically after the completion of every job.
The issued heartbeats contain power and timing data that POET queries at runtime.

The third necessary input is a latency target.
The user provides the target through the Heartbeats API, specifying a minimum and maximum latency goal.
The timing targets can change during runtime, and POET will take care of the adaptation automatically.


\subsection{Software Interface}
\label{sec:poet-interface}

Users interact with only three POET functions.
\function{poet\_init} initializes POET and returns a \variable{poet\_state} data structure reference.
\function{poet\_apply\_control} executes the controller, runs \algoref{poet-optimal}, and configures the platform.
\function{poet\_destroy} cleans up the \variable{poet\_state} data structure.

POET's initialization function takes, as parameters, references to: the heartbeat data structure used to store the timing and power of the application, the system's configurations, and the function that applies the given system configurations.
It also receives an optional reference to the function that determines the system's current state and a log file name.
The first system configuration data structure (system-agnostic) is of type \variable{poet\_control\_state\_t}, and the second (system-specific) has type \variable{void}.

The two functions passed by reference are the only ones that need to know the format of the second data structure and are therefore passed the \variable{void} type reference given to \function{poet\_init} as parameters.
The first of these two functions must have a signature that matches the \variable{poet\_apply\_func} definition and the second must match the \variable{poet\_curr\_state\_func} definition.

The other two API functions, \function{poet\_apply\_control} and \function{poet\_destroy}, take the \variable{poet\_state} reference as their only parameter.
This variable contains all the control state required to implement the framework described in \secref{poet-framework}.

Auxiliary functions are also provided to load system configurations from files, discover the initial system configuration, and apply system configurations.
The latter two of these meet the \variable{poet\_curr\_state\_func} and \variable{poet\_apply\_func} definitions, respectively, and can be passed to \function{poet\_init}.
These auxiliary functions are platform-dependent and thus kept separate to maintain portability, allowing users to easily substitute their own versions.
They are, however, generic enough that most Linux users do not need to write their own.


\subsection{Runtime}

After an application signals its power and latency with a heartbeat, it makes a call to \function{poet\_apply\_control}, which contains POET's core logic.
Heartbeats are initialized with a {\em window size} value that indicates the number of jobs to complete in a given {\em time interval}.
The window size is analogous to $I(t)$ from \eqnref{poet-work}, while the time interval is equivalent to $\tau$ from \eqnref{poet-time} and \algoref{poet-optimal}.
At the completion of a window, the POET runtime calculates the estimated base speed with \eqnref{kalman-filter}, then computes the latency error with \eqnref{poet-error}.
It subsequently applies the controller of \eqnref{poet-control} to determine the speedup necessary to eliminate the computed error.
Finally it determines the energy-minimal resource schedule that achieves the necessary speedup using \algoref{poet-optimal} and applies the first configuration in the schedule by executing the provided \variable{poet\_apply\_func} function.

The last step, a call to the \variable{poet\_apply\_func} specified by the user, is a platform-dependent operation.
For example, the function included with POET for Linux platforms invokes the \app{taskset} utility to force the process and all of its threads onto the desired number of cores and uses the \app{cpufrequtils} interface to adjust the cores' clock speeds.
Developers can specify their own function.
For example, a system may require a different approach to change the number of active cores.
Also, a system may have additional configurable resources that could be adjusted, like memory and network bandwidth.

The only remaining task is to apply the second configuration in the schedule at the appropriate time during the next window.
When the computed number of heartbeats to wait passes, the \variable{poet\_apply\_func} function executes again, but no further computation is performed.
At the completion of the heartbeat window, the control process repeats.

The code snippet shown in \lstref{poet-example} provides an example of application code, highlighting the POET function calls.
% Adding POET to existing applications does not require many modifications to the original code.
The complete modification of an existing application requires nine function calls plus associated variable declarations, for a total of 14 lines of code.
The user provides a desired latency target via the Heartbeats API using the \variable{min\_heartrate} and \variable{max\_heartrate} variables, which represent a desired minimum and maximum speed in terms of jobs completed per second.
POET simply takes the average of these two values, meaning they can be the same.
Given $I(t)$ jobs in a window and a target latency $\tau$, the desired rates are easily computed as:
\begin{equation}
  min\_heartrate = max\_heartrate = \frac{I(t)}{\tau} 
  \label{eqn:latency-to-performance}
\end{equation}
As demonstrated below, Heartbeats initialization also accepts requests for minimum and maximum accuracy and power -- POET does not use these, so they can safely be set to any value.
When initializing POET, the user should specify the system's configurations, encoded in \variable{control\_states} and \variable{cpu\_states}.

\lstset{emph={%  
    poet_state, poet_init, poet_apply_control, poet_destroy%
    },emphstyle={\color{black}\bfseries\underbar}%
}%
\begin{lstlisting}[language=C,%
  caption={Example of POET application code.},%
  label={lst:poet-example}]%

// initialization
heartbeat_t* heart =
  heartbeat_acc_pow_init(window_size, buffer_depth, "heartbeat.log",
                         min_heartrate, max_heartrate, min_accuracy, max_accuracy,
                         1, hb_energy_impl_alloc(), min_power, max_power);
get_control_states(NULL, &control_states, &nstates);
get_cpu_states(NULL, &cpu_states, &nstates);
poet_state* state = poet_init(heart, nstates, control_states, cpu_states,
                              &apply_cpu_config, &get_current_cpu_state,
                              buffer_depth, "poet.log");
// execution of main loop
while(running) {
  heartbeat_acc(heart, count++, 1);
  poet_apply_control(state);
  doWork();
}
// cleanup
poet_destroy(state);
free(control_states);
free(cpu_states);
heartbeat_finish(heart);
\end{lstlisting}

\TODO{Also present updated interface?}

\section{Experimental Design}
\label{sec:poet-usage}

This section details the applications used to evaluate POET and the different evaluation platforms.
The evaluation is broken down into two categories: an embedded systems evaluation and a server-class system evaluation.


\subsection{Applications}

To represent a wide variety of embedded applications, we use eight different benchmarks, none of which were originally written to provide predictable timing.
We choose applications that do not enforce any timing guarantees to challenge POET's approach as much as possible.

The first five applications are included in the PARSEC benchmark suite~\cite{parsec}.
Specifically, we use \app{blackscholes}, \app{bodytrack}, \app{facesim}, \app{ferret}, and \app{x264}.
\app{Bodytrack} and \app{x264} process video input and could be required to match the frame rate of a live feed (\eg from an on-board camera).
\app{Ferret} is a toolkit for content-based similarity search of non-text data and should satisfy a latency requirement on how fast results are returned to users.
\app{Facesim} creates realistic animations of a human face from a model and time sequence of muscle movements, and must maintain a real-time frame rate.
The sixth and seventh applications are \app{dijkstra} and \app{sha}, from the ParMiBench benchmark suite~\cite{parmibench}.
\app{Dijkstra} computes single-source shortest paths in a graph (we use the parallelized multiple queue implementation).
\app{SHA} (Secure Hash Algorithm) is used for secure storage and transmission of data and must maintain response time to ensure timely communication.
The last application is \app{STREAM}~\cite{stream}, a synthetic benchmark for measuring sustainable memory bandwidth, representing a variety of memory-bound applications.

All benchmarks are modified as discussed in~\secref{poet-implementation}, adding Heartbeats and POET calls.
All application inputs used are packaged with the original benchmarks, with the exception of the \app{x264} input which comes from a set of standard test sequences.


\subsection{Evaluation Platforms}

The first part of the evaluation uses two modern embedded devices with different hardware---a Sony VAIO SVT11226CXB tablet PC with a mobile Intel Haswell processor, and a Hardkernel ODROID-XU+E ARM big.LITTLE development platform.
We selected these two platforms because prior work has shown that they expose different timing and energy tradeoffs~\cite{Imes2014}.
% \tblref{poet-embedded-systemknobs} shows the hardware details for both, highlighting the configurable resources, the cardinality of the set of alternatives for each, and the maximum speedup achievable by manipulating that resource alone.
Both platforms run Ubuntu Linux 14.04 LTS.
The Vaio uses kernel 3.13.0, while the ODROID runs kernel 3.4.104.

% \begin{table}[t]
% \caption{Embedded system configurations.}
% \centering
% \begin{tabular}{clcc}
%   & \textbf{Resource} & \textbf{Settings} & \textbf{Max Speedup} \\
% \hline
% \hline
%   \multirow{3}{*}{\begin{turn}{90}\textbf{Vaio}\end{turn}} 
%   & cores        &  2 & 1.81 \\
%   & core speeds  & 11 & 2.72 \\
%   & hyperthreads &  2 & 1.10 \\ 
% \hline
% \hline
%   \multirow{4}{*}{\begin{turn}{90}\textbf{O-XU+E}\end{turn}} 
%   & big cores          & 4 & 6.10 \\
%   & big core speeds    & 9 & 1.97 \\
%   & LITTLE cores       & 4 & 3.94 \\
%   & LITTLE core speeds & 8 & 2.40 \\
% \hline
% \hline
% \end{tabular}
% \label{tbl:poet-embedded-systemknobs}
% \end{table}

The second part of the evaluation uses a dual-socket server system, where each socket contains 8 cores.
With HyperThreading, the system exposes 32 virtual cores.
There are 16 DVFS settings available, including TurboBoost.
The server system also runs Ubuntu 14.04 LTS with Linux kernel 3.13.0.

A \textbf{configuration} is a unique combination of allowable values for the system resources.
In all cases, \app{cpufrequtils} controls processor clock speeds and \app{taskset} controls core allocation.
While the Vaio claims to support different frequency settings on different virtual cores, our experience leads us to conclude that this is not the case.
Thus, we allow only configurations where all cores are set to the same frequency.
The ODROID's version of the Exynos5 Octa does not support executing on the big and LITTLE clusters simultaneously, and all cores in a cluster must operate at the same frequency.
We use a mainline Linux kernel with the default In Kernel Switcher for managing cluster migration.
%IKS maintains compatibility with existing schedulers by modifying existing DVFS systems, but at a cost of supporting execution on only one cluster at a time. This is achieved on the ODROID by using dummy frequency values for the LITTLE cores so as not to overlap with the frequencies provided by the big cores.
On the server system, we set the DVFS frequency on all cores, like on the Vaio.

\begin{table}[t]
\small
\centering
\caption{System power characteristics.}
\begin{tabular}{cccc}
  \textbf{System} & \textbf{Idle Power} & \textbf{Min Power} & \textbf{Max Power} \\
  \hline
  \hline
  Vaio        & 2.50 W  & 3.04 W  & 8.05 W \\
  ODROID-XU+E & 0.12 W  & 0.17 W  & 8.14 W \\
  Server      & 17.90 W & 37.80 W & 199.26 W \\
  % ODROID-XU3  & 0.21 W  & 0.19 W  & 6.37 W \\
  \hline
  \hline
\end{tabular}
\label{tbl:poet-power}
\end{table}

Capturing power/energy metrics naturally requires hardware resources that expose power or energy data to software.
To capture power measurements on the Vaio and the server system, we use the Model-Specific Registers (MSRs) provided by Intel~\cite{SandyBridge}.
On the ODROID, we poll INA-231 power sensors to capture power data for the A15 and A7 clusters as well as for the DRAM and for the GPU~\cite{ina231}.
Basic power figures for the three platforms are shown in~\tblref{poet-power}.
The modified version of Heartbeats includes energy readers for some common hardware (\eg the MSR) and exposes a simple interface for extending to new hardware.
Collecting power data on new platforms with different power or energy monitors is easy and does not require any modifications to POET.

\begin{table}[t]
\small
\centering
\caption{Embedded systems evaluation inputs and configuration details.}
\begin{tabular}{cccc}
  \textbf{Application} & \textbf{Input} & \textbf{Jobs} & \textbf{Window Size} \\
  \hline
  \hline
  blackscholes   & 1 million options              & 400 batches   & 20 \\
  bodytrack      & sequenceB                      & 261 frames    & 20 \\
  facesim        & Storytelling                   & 100 frames    & 20 \\
  ferret         & corel:lsh                      & 2,000 queries & 20 \\
  x264           & ducks\_take\_off               & 500 frames    & 20 \\
  dijkstra       & input\_small                   & 1,000 paths   & 20 \\
  sha            & in\_file(1-16)                 & 1,000 hashes  & 50 \\
  STREAM         & self-generated                 & 1,000 updates & 50 \\
  \hline
  \hline
\end{tabular}
\label{tbl:poet-embedded-inputs}
\end{table}

\begin{figure}[t]
  \centering
  \begin{tikzpicture}

\begin{groupplot}[
    group style={
        group name=plots,
        group size=1 by 1,
        xlabels at=edge bottom,
        xticklabels at=edge bottom,
        vertical sep=5pt
    },
%axis x line* = bottom,
xlabel near ticks,
major x tick style = transparent,
xlabel={},
height=3.5cm,
width=0.95\columnwidth,
xmin=0,
xmax=9,
enlargelimits=false,
tick align = outside,
tick style={white},
ytick=\empty,
xticklabel shift={0pt},
x tick label style={rotate=35, anchor=east, font=\scriptsize},
xtick={1,2,3,4,5,6,7,8,9},
xticklabels={
{blackscholes},
{bodytrack},
{facesim},
{ferret},
{x264},
{dijkstra},
{sha},
{STREAM}},
ymin=0,
ymax=0.6,
ytick={0,0.15,0.30,0.45,0.60},
yticklabels={0,0.15,0.30,0.45,0.60},
legend cell align=left,
legend style={ column sep=1ex },
ymajorgrids,
grid style={dashed},
ylabel shift={0mm},
ylabel style={align=center},
]
\nextgroupplot[ylabel={\footnotesize Coeff. of Variation \\ \scriptsize (Std. Deviation/Mean)},
ybar=\pgflinewidth,
bar width=8pt,
legend entries = {{Vaio},{ODROID-XU+E}},
legend style={draw=none,legend columns=2,at={(0.5,1.4)},anchor=north,font=\footnotesize},
]
\addplot table[x index=0,y index=3, col sep=tab] {img/poet/variability-embedded.txt};
\addplot table[x index=0,y index=2, col sep=tab] {img/poet/variability-embedded.txt};


\end{groupplot}

\end{tikzpicture}

  \caption{Embedded systems application latency variability.}
  \label{fig:poet-embedded-variation}
\end{figure}

\tblref{poet-embedded-inputs} shows the inputs used for each of the applications on the embedded platforms.
We quantify this inherent unpredictability by measuring the latency of each job and computing the coefficient of variation (ratio of standard deviation to the mean) over all jobs in an application.
\figref{poet-embedded-variation} demonstrates this unpredictability for each application when running without POET.
The figure shows that our applications have a range of natural behavior from low variance (implying natural predictability, \eg \app{blackscholes}) to high variance (meaning that the application naturally has widely distributed latencies, \eg \app{x264}).
The variability in the applications is largely the same across platforms, indicating that it is a fundamental property of the applications and not the devices.

\begin{table}[t]
\small
\centering
\caption{Server-class system evaluation inputs and configuration details.}
\begin{tabular}{cccc}
  \textbf{Application} & \textbf{Input} & \textbf{Jobs} & \textbf{Window Size} \\
  \hline
  \hline
  blackscholes   & 10 million options             & 400 batches   & 50 \\
  bodytrack      & sequenceB                      & 261 frames    & 50 \\
  facesim        & Storytelling                   & 100 frames    & 20 \\
  ferret         & corel:lsh                      & 2,000 queries & 50 \\
  x264           & rush\_hour                     & 1,500 frames  & 100 \\
  dijkstra       & input\_large                   & 1,000 paths   & 50 \\
  sha            & in\_file(1-16)                 & 1,000 hashes  & 50 \\
  STREAM         & self-generated                 & 1,000 updates & 50 \\
  \hline
  \hline
\end{tabular}
\label{tbl:poet-server-inputs}
\end{table}

\begin{figure}[t]
  \begin{tikzpicture}

\begin{groupplot}[
    group style={
        group name=plots,
        group size=1 by 1,
        xlabels at=edge bottom,
        xticklabels at=edge bottom,
        vertical sep=5pt
    },
%axis x line* = bottom,
xlabel near ticks,
major x tick style = transparent,
xlabel={},
height=3.5cm,
width=0.95\columnwidth,
xmin=0,
xmax=9,
enlargelimits=false,
tick align = outside,
tick style={white},
ytick=\empty,
xticklabel shift={0pt},
x tick label style={rotate=35, anchor=east, font=\scriptsize},
xtick={1,2,3,4,5,6,7,8,9},
xticklabels={
{blackscholes},
{bodytrack},
{facesim},
{ferret},
{x264},
{dijkstra},
{sha},
{STREAM}},
ymin=0,
ymax=0.6,
ytick={0,0.15,0.30,0.45,0.60},
yticklabels={0.00,0.15,0.30,0.45,0.60},
legend cell align=left,
legend style={ column sep=1ex },
ymajorgrids,
grid style={dashed},
ylabel shift={0mm},
ylabel style={align=center},
]
\nextgroupplot[ylabel={\footnotesize Coeff. of Variation \\ \scriptsize (Std. Deviation/Mean)},
ybar=\pgflinewidth,
bar width=8pt,
legend entries = {{Server}},
legend style={draw=none,legend columns=1,at={(0.5,1.4)},anchor=north,font=\footnotesize},
]
\addplot table[x index=0,y index=2, col sep=tab] {img/poet/variability-server.txt};


\end{groupplot}

\end{tikzpicture}

  \caption{Server system application latency variability.}
  \label{fig:poet-server-variation}
\end{figure}

\tblref{poet-server-inputs} lists the inputs used for each of the applications on the server system.
The server-class system is significantly more powerful than the embedded systems.
The overhead of changing resource allocations is also higher due to the larger core count.
As a result, we increased both the size or length of some inputs and the window period size.
Again, we quantify the variability in the applications and present the results in \figref{poet-server-variation}.
As expected, they are similar to those from the embedded systems.

\section{Evaluation}
\label{sec:poet-evaluation}



\section{Inspired Projects}
\label{sec:poet-inspired}

\TODO{Find appropriate place for this material, \eg an Appendix?}

POET has also been used as the foundation for other controllers and projects.
We expanded on POET to create Bard, which adds the ability to instead meet soft power constraints and maximize performance \cite{Bard}.
Farrell and Hoffmann build on POET by additionally changing application accuracy to provide hard real-time guarantees \cite{meantime}.
Mishra \etal combine POET with machine learning in a project called CALOREE to reduce the amount of offline work needed to generate resource specifications, and update both the specification and pole value online as new application behavior is learned \cite{CALOREE}.
POET was also the starting point in an ongoing project called Proteus, in which we are developing a programming language called FAST that allows programmers to specify more general constraints and optimizations called \emph{intents}, which uses both system and application knobs to satisfy.

Additionally, POET (and the work that motivated it \cite{Imes2014}) ultimately led to developing EnergyMon, a portable software interface for accessing energy metrics at runtime \cite{energymon}.
EnergyMon was integrated with Mozilla's Servo web browsing engine \cite{servo}.
It is also utilized in other research projects, like the aforementioned FAST language, and in the next project in this thesis, CoPPer.
