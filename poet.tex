\chapter{POET: The Performance with Optimal Energy Toolkit}
\label{sec:poet}

This chapter describes POET (the \textbf{P}erformance with \textbf{O}ptimal \textbf{E}nergy \textbf{T}oolkit).
\secref{related} discussed existing work in managing performance and power/energy awareness.
Like POET, a number of these approaches use feedback control to manage timing constraints \cite{Agilos,Bertini2007,grace2,CoAdapt,Maggio13,TCST,ControlWare,PTRADE,Fu2012,Li2007}.
POET is most related to prior approaches that abstract resource management into a middleware or runtime, like \cite{ControlWare,Sojka,PTRADE,Rajkumar}.
Control techniques provide a formal framework for reasoning about the dynamic behavior of the system.
POET is unique in its energy awareness, in its design for portability, and the incorporation of a true minimal-energy resource allocation algorithm.

\TODO{Combine RTAS and MCSoC papers, including motivation, usage, and evaluation.}

\section{Motivation}
\label{sec:poet-motivation}

To motivate the need for POET, we summarize prior work \cite{Imes2014}.
We evaluate the timing and energy tradeoffs of a video encoder on two embedded platforms, a Sony Vaio tablet and an ODROID development board.
The two platforms not only have different resources for management, but also have latency and energy tradeoffs with different topologies.
Thus, resource allocation strategies that save energy on one are wasteful on the other.

\begin{table}[t]
\caption{Two embedded platforms with different configurable components.}
\label{tbl:poet-machines}
\tiny
\centering
\begin{tabular}{lcccccccc}
  \textbf{Platform} & 
  \textbf{Processor} &
  \textbf{Cores} & 
  \textbf{Core Types} &
  \textbf{Speeds (GHz)} &
  \textbf{TurboBoost} &
  \textbf{HyperThreads} & 
  \textbf{Num. Configs} \\
  % \textbf{Configurations} \\
  \hline
  \hline
  SVT11226CXB & Intel Haswell  & 2 & 1             & 0.6--1.5          & yes & yes & 46 \\
  ODROID-XU+E & ARM big.LITTLE & 8 & 2 (A15, A7)   & 0.8--1.6, 0.5--1.2 & no  & no  & 70 \\
  \hline 
  \hline
\end{tabular}
% \vskip -.7em
\end{table}

Our example features a video encoder, composed of jobs, where each job encodes a frame.
We instrument the encoder to report job latency and we measure the platform's energy consumption over time.
The two platforms have different configurable resources, shown in \tblref{poet-machines}.
The Vaio allows configuration of the number of active cores, the number of hyperthreads per core, the speed of the processors, and the use of TurboBoost.
The ODROID, supports configuration of the number of active cores, their clock speed, and whether the application uses the ``big'' (Cortex-A15 high performance, high power) or ``LITTLE'' (Cortex-A7 low performance, low power) cores.

\figref{poet-x264-motivation-tradeoffs} shows the tradeoffs between energy consumption and latency.
The x-axis shows the average latency (normalized to 1---the empirically determined worst case).
The y-axis shows energy (normalized to 1---the highest measured energy).
The two plots show the very different tradeoffs for the Vaio and the ODROID with each point representing a different configuration.
For the Vaio, energy increases as frame latency increases; \ie a slower job wastes energy.
For the ODROID, energy decreases as frame latency increases; \ie slower encodings save energy.

\begin{figure}[t]
  % \vskip -1.8em
  \centering
  \subfloat[Energy/Latency Tradeoffs]%
  {\begin{tikzpicture}

\definecolor{s1}{RGB}{228, 26, 28}
\definecolor{s2}{RGB}{55, 126, 184}
\definecolor{s3}{RGB}{77, 175, 74}
\definecolor{s4}{RGB}{152, 78, 163}
\definecolor{s5}{RGB}{255, 127, 0}

\begin{groupplot}[
    group style={
        group name=plots,
        group size=1 by 1,
        xlabels at=edge bottom,
        xticklabels at=edge bottom,
        vertical sep=5pt
    },
xlabel={\footnotesize Latency},
xlabel near ticks,
height=4.2cm,
%width=0.5\textwidth,
width = 4.6cm,
xmajorgrids,
ymajorgrids,
grid style={dashed},
xtick={0,0.25,0.5,0.75,1.0},
xticklabels={0,0.25,0.5,0.75,1.0},
xticklabel style={font=\footnotesize},
xmin=0,
xmax=1.1,
yticklabel pos=left,
enlargelimits=false,
tick align = outside,
tick style={white},
xticklabel shift={-5pt},
yticklabel shift={-5pt},
ylabel shift={-2pt},
ylabel style={align=center},
unbounded coords=jump,
]

\nextgroupplot[ylabel={\footnotesize Energy}, 
%ylabel shift={6mm},
ytick={0,0.25,0.5,0.75,1.0},
yticklabels={0,0.25,0.5,0.75,1.0},
yticklabel style={font=\footnotesize},
ymin=0,
ymax=1.1,
legend entries={{Vaio},{ODROID}},
legend style={draw=none,at={(0.5,1.35)},anchor=north,legend columns=4,line width=5pt,font=\footnotesize},
]
\addplot[thick, solid, color=s1, only marks, mark=square*] table[x index=2,y index=5,col sep=tab] {img/poet/tradeoffs-vaio.txt};
\addplot[thick, solid, color=s3, only marks, mark=*] table[x index=2,y index=5,col sep=tab] {img/poet/tradeoffs-odroid.txt};


\end{groupplot}

\end{tikzpicture}
%
  \label{fig:poet-x264-motivation-tradeoffs}}
  \subfloat[Energy Consumption]%
  {\begin{tikzpicture}

\begin{groupplot}[
    group style={
        group name=plots,
        group size=1 by 1,
        xlabels at=edge bottom,
        xticklabels at=edge bottom,
        vertical sep=5pt
    },
% axis x line* = bottom,
xlabel near ticks,
major x tick style = transparent,
height=4.2cm,
%width=0.95\columnwidth,
width=4.6cm,
xmin=0.5,
xmax=2.5,
enlargelimits=false,
tick align = outside,
tick style={white},
ytick=\empty,
xticklabel shift={-5pt},
xticklabel style={font=\footnotesize},
%x tick label style={rotate=0, anchor=south},
xlabel={\footnotesize Platform},
xtick={1,2,3},
xticklabels={{Vaio},{ODROID}},
ymin=1,
ymax=2.25,
ytick={1.0,1.25,1.5,1.75,2.0,2.25},
yticklabels={1.00,1.25,1.50,1.75,2.00,2.25},
legend cell align=left, 
legend style={ column sep=1ex },
ymajorgrids,
grid style={dashed},
]
\nextgroupplot[ylabel={\footnotesize Energy},
ybar=\pgflinewidth,
bar width=8pt,
legend entries = {{never-idle},{race-to-idle}},
legend style={draw=none,legend columns=2,at={(0.5,1.35)},anchor=north,font=\footnotesize},
]
\addplot table[x index=0,y index=2, col sep=space] {img/poet/heuristics2.txt};
\addplot table[x index=0,y index=3, col sep=space] {img/poet/heuristics2.txt};


\end{groupplot}

\end{tikzpicture}
%
  \label{fig:poet-x264-motivation-heuristics}}
  \caption{Energy/latency tradeoffs.}
  \label{fig:poet-x264-motivation}
\end{figure}

The different shapes of these tradeoff spaces lead to different optimal resource allocation strategies.
Empirical studies show that the \emph{race-to-idle} heuristic, which makes all resources available and then idles after completing a job, is near optimal on systems like the Vaio~\cite{PowerSlope,Hoelzle2009,google,Imes2014,HotPower}.
On systems like the ODROID, recent approaches save energy by keeping the system constantly busy and \emph{never-idle}~\cite{Carroll13,LeSueur11,Lin2010,Imes2014,HotPower}.

To demonstrate the importance of choosing the right strategy, we analyze the two heuristics on both platforms and compare their energy consumption to optimal.
We set a latency target equal to twice the minimum latency and measure the energy consumption of encoding 500 video frames using each heuristic.
\figref{poet-x264-motivation-heuristics} shows the results, normalized to the optimal energy found by measuring every possible resource configuration.
Both heuristics meet the latency target, but their energy consumptions vary tremendously.
On the Vaio, \emph{race-to-idle} is near optimal, but \emph{never-idle} consumes 13\% more energy.
Conversely, \emph{never-idle} is near optimal for the ODROID, but \emph{race-to-idle} consumes $2 \times$ more energy.

These results demonstrate that resource allocation strategy greatly affects energy consumption, and more importantly, that heuristic solutions are \textbf{not portable} across devices.
These two points motivate the need for an approach like POET, which provides near optimal resource allocation while remaining platform-independent.
POET's runtime uses control theory to meet timing constraints and linear programming to minimize energy consumption.
A POET user does not need to be a control or optimization expert, but simply make small changes to their application code.
POET makes it easy for embedded developers to write portable applications providing predictable timing and minimal energy across a range of devices.

\section{Design and Implementation}
\label{sec:poet-framework}

\section{Implementation}
\label{sec:poet-implementation}

We describe how the framework in the previous section is realized in a C library.
We specify the information that must be provided by POET's users, describe the library interface, and then discuss the implementation of the runtime engine.

\subsection{External Inputs}

POET requires three pieces of information from users.
First, it needs the available system configurations and their associated timing and power characteristics.
Second, it needs a means to measure timing and power consumption during runtime.
Third, it requires the application to specify the desired latency target.

The first input is the specification of available system configurations.
POET separates the system configurations into two data structures.
The first is system-agnostic and contains a configuration identifier along with \textbf{speedup} and \textbf{powerup} values.
The second is system-specific and can take any form a developer considers appropriate to define a system configuration.
To simplify the programmer's job, POET includes a default format for this second structure which contains a configuration identifier, the DVFS frequency to apply, and the number of cores to use.
Snippets of actual configuration files representing both these data structures are presented in \figref{config-examples}.
\secref{poet-usage} describes how to characterize a system's timing and power behavior.
These results can be used to create configurations if they are not already available.

\begin{figure}[t]
\centering
\begin{minipage}{.45\columnwidth}
\lstset{
  belowskip=0pt,
  aboveskip=0pt,
}
\begin{lstlisting}[frame=tlr,%
  caption={System-agnostic.},%
  label={lst:control_config_example}]%

#id        speedup         powerup
0          1               1
1          1.20            1.09
2          1.40            1.16
3          1.60            1.30
4          2.12            1.35
5          2.53            1.50
6          2.88            1.64
7          3.18            1.69
\end{lstlisting}
\end{minipage}
% \hfill
\hspace*{0.4cm}
\begin{minipage}{.45\columnwidth}
\lstset{
  belowskip=0pt,
  aboveskip=0pt,
}
\begin{lstlisting}[frame=tlr,%
  caption={System-specific.},%
  label={lst:cpu_config_example}]%

#id        frequency        cores
0          250000           0
1          300000           0
2          350000           0
3          400000           0
4          250000           1
5          300000           1
6          350000           1
7          250000           2
\end{lstlisting}
\end{minipage}
% \vskip -1.5em
\caption{Example of POET configuration files.}
\label{fig:config-examples}
\end{figure}

To measure both latency and power, we modify the Heartbeats API \cite{PTRADE} to record power data along with timing statistics.
We then modify applications to issue heartbeats at appropriate points during processing, typically after the completion of every job.
The issued heartbeats contain power and timing data that POET queries at runtime.

The third necessary input is a latency target.
The user provides the target through the Heartbeats API, specifying a minimum and maximum latency goal.
The timing targets can change during runtime, and POET will take care of the adaptation automatically.


\subsection{Software Interface}
\label{sec:poet-interface}

Users interact with only three POET functions.
\function{poet\_init} initializes POET and returns a \variable{poet\_state} data structure reference.
\function{poet\_apply\_control} executes the controller, runs \algoref{poet-optimal}, and configures the platform.
\function{poet\_destroy} cleans up the \variable{poet\_state} data structure.

POET's initialization function takes, as parameters, references to: the heartbeat data structure used to store the timing and power of the application, the system's configurations, and the function that applies the given system configurations.
It also receives an optional reference to the function that determines the system's current state and a log file name.
The first system configuration data structure (system-agnostic) is of type \variable{poet\_control\_state\_t}, and the second (system-specific) has type \variable{void}.

The two functions passed by reference are the only ones that need to know the format of the second data structure and are therefore passed the \variable{void} type reference given to \function{poet\_init} as parameters.
The first of these two functions must have a signature that matches the \variable{poet\_apply\_func} definition and the second must match the \variable{poet\_curr\_state\_func} definition.

The other two API functions, \function{poet\_apply\_control} and \function{poet\_destroy}, take the \variable{poet\_state} reference as their only parameter.
This variable contains all the control state required to implement the framework described in \secref{poet-framework}.

Auxiliary functions are also provided to load system configurations from files, discover the initial system configuration, and apply system configurations.
The latter two of these meet the \variable{poet\_curr\_state\_func} and \variable{poet\_apply\_func} definitions, respectively, and can be passed to \function{poet\_init}.
These auxiliary functions are platform-dependent and thus kept separate to maintain portability, allowing users to easily substitute their own versions.
They are, however, generic enough that most Linux users do not need to write their own.


\subsection{Runtime}

After an application signals its power and latency with a heartbeat, it makes a call to \function{poet\_apply\_control}, which contains POET's core logic.
Heartbeats are initialized with a {\em window size} value that indicates the number of jobs to complete in a given {\em time interval}.
The window size is analogous to $I(t)$ from \eqnref{poet-work}, while the time interval is equivalent to $\tau$ from \eqnref{poet-time} and \algoref{poet-optimal}.
At the completion of a window, the POET runtime calculates the estimated base speed with \eqnref{kalman-filter}, then computes the latency error with \eqnref{poet-error}.
It subsequently applies the controller of \eqnref{poet-control} to determine the speedup necessary to eliminate the computed error.
Finally it determines the energy-minimal resource schedule that achieves the necessary speedup using \algoref{poet-optimal} and applies the first configuration in the schedule by executing the provided \variable{poet\_apply\_func} function.

The last step, a call to the \variable{poet\_apply\_func} specified by the user, is a platform-dependent operation.
For example, the function included with POET for Linux platforms invokes the \app{taskset} utility to force the process and all of its threads onto the desired number of cores and uses the \app{cpufrequtils} interface to adjust the cores' clock speeds.
Developers can specify their own function.
For example, a system may require a different approach to change the number of active cores.
Also, a system may have additional configurable resources that could be adjusted, like memory and network bandwidth.

The only remaining task is to apply the second configuration in the schedule at the appropriate time during the next window.
When the computed number of heartbeats to wait passes, the \variable{poet\_apply\_func} function executes again, but no further computation is performed.
At the completion of the heartbeat window, the control process repeats.

The code snippet shown in \lstref{poet-example} provides an example of application code, highlighting the POET function calls.
% Adding POET to existing applications does not require many modifications to the original code.
The complete modification of an existing application requires nine function calls plus associated variable declarations, for a total of 14 lines of code.
The user provides a desired latency target via the Heartbeats API using the \variable{min\_heartrate} and \variable{max\_heartrate} variables, which represent a desired minimum and maximum speed in terms of jobs completed per second.
POET simply takes the average of these two values, meaning they can be the same.
Given $I(t)$ jobs in a window and a target latency $\tau$, the desired rates are easily computed as:
\begin{equation}
  min\_heartrate = max\_heartrate = \frac{I(t)}{\tau} 
  \label{eqn:latency-to-performance}
\end{equation}
As demonstrated below, Heartbeats initialization also accepts requests for minimum and maximum accuracy and power -- POET does not use these, so they can safely be set to any value.
When initializing POET, the user should specify the system's configurations, encoded in \variable{control\_states} and \variable{cpu\_states}.

\lstset{emph={%  
    poet_state, poet_init, poet_apply_control, poet_destroy%
    },emphstyle={\color{black}\bfseries\underbar}%
}%
\begin{lstlisting}[language=C,%
  caption={Example of POET application code.},%
  label={lst:poet-example}]%

// initialization
heartbeat_t* heart =
  heartbeat_acc_pow_init(window_size, buffer_depth, "heartbeat.log",
                         min_heartrate, max_heartrate, min_accuracy, max_accuracy,
                         1, hb_energy_impl_alloc(), min_power, max_power);
get_control_states(NULL, &control_states, &nstates);
get_cpu_states(NULL, &cpu_states, &nstates);
poet_state* state = poet_init(heart, nstates, control_states, cpu_states,
                              &apply_cpu_config, &get_current_cpu_state,
                              buffer_depth, "poet.log");
// execution of main loop
while(running) {
  heartbeat_acc(heart, count++, 1);
  poet_apply_control(state);
  doWork();
}
// cleanup
poet_destroy(state);
free(control_states);
free(cpu_states);
heartbeat_finish(heart);
\end{lstlisting}

\section{Using POET}
\label{sec:poet-usage}

This section details the platforms and applications used to evaluate POET.


\subsection{Testing Platforms}

We use two modern embedded devices with different hardware.
We selected these two platforms because prior work has shown that they expose different timing and energy tradeoffs~\cite{Imes2014}.
\tblref{poet-systemknobs} shows the hardware details for both, highlighting the configurable resources, the cardinality of the set of alternatives for each, and the maximum speedup achievable by manipulating that resource alone.
Both platforms run Ubuntu Linux 14.04.
The Vaio uses kernel 3.13.0, while the ODROID runs kernel 3.4.104.
In both cases, \mbox{\texttt{cpufrequtils}} controls processor clock speeds.
A \textbf{configuration} is a unique combination of allowable values for the system resources.

\begin{table}[t]
\caption{System configurations.}
\centering
\begin{tabular}{clcc}
  & \textbf{Resource} & \textbf{Settings} & \textbf{Max Speedup} \\
\hline
\hline
  \multirow{3}{*}{\begin{turn}{90}\textbf{Vaio}\end{turn}} 
  & cores        &  2 & 1.81 \\
  & core speeds  & 11 & 2.72 \\
  & hyperthreads &  2 & 1.10 \\ 
\hline
\hline
  \multirow{4}{*}{\begin{turn}{90}\textbf{ODROID}\end{turn}} 
  & big cores          & 4 & 6.10 \\
  & big core speeds    & 9 & 1.97 \\
  & LITTLE cores       & 4 & 3.94 \\
  & LITTLE core speeds & 8 & 2.40 \\
\hline
\hline
\end{tabular}
\label{tbl:poet-systemknobs}
\end{table}

While the Vaio claims to support different frequency settings on different virtual cores, our experience leads us to conclude that this is not the case.
Thus, we allow only configurations where all cores are set to the same frequency.

The ODROID's version of the Exynos5 Octa does not support executing on the big and LITTLE clusters simultaneously, and all cores in a cluster must operate at the same frequency.
We use a mainline Linux kernel with the default In Kernel Switcher for managing cluster migration.
%IKS maintains compatibility with existing schedulers by modifying existing DVFS systems, but at a cost of supporting execution on only one cluster at a time. This is achieved on the ODROID by using dummy frequency values for the LITTLE cores so as not to overlap with the frequencies provided by the big cores.

To capture power measurements on the Vaio, we use the Model-Specific Register (MSR) of the Haswell processor~\cite{SandyBridge}.
On the ODROID, we poll INA-231 power sensors~\cite{ina231} available on the XU+E model to capture power data for the A15 and A7 clusters as well as for the DRAM and for the GPU.
Basic power figures of the two platforms are shown in~\tblref{poet-power}.

Capturing these metrics naturally requires hardware resources that expose power or energy data to software.
The modified version of Heartbeats includes energy readers for some common hardware (\eg the MSR) and exposes a simple interface for extending to new hardware.
Collecting power data on new platforms with different power or energy monitors is easy and does not require any modifications to POET.

\begin{table}[tb]
% \vskip -1.5em
\centering
\caption{System power characteristics.}
\begin{tabular}{cccc}
  \textbf{System} & \textbf{Idle Power} & \textbf{Min Power} & \textbf{Max Power} \\
  \hline
  \hline
  Vaio   & 2.50 W & 3.04 W & 8.05 W \\
  ODROID & 0.12 W & 0.17 W & 8.14 W \\
  \hline
  \hline
\end{tabular}
\label{tbl:poet-power}
\end{table}


\subsection{Applications}

To represent a wide variety of embedded applications, we use eight different benchmarks, none of which were originally written to provide predictable timing.
We choose applications that do not enforce any timing guarantees to challenge POET's approach as much as possible.

The first five applications are included in the PARSEC benchmark suite~\cite{parsec}.
Specifically, we use \app{blackscholes}, \app{bodytrack}, \app{facesim}, \app{ferret}, and \app{x264}.
Bodytrack and x264 process video input and could be required to match the frame rate of a live feed (\eg from an on-board camera).
Ferret is a toolkit for content-based similarity search of non-text data and should satisfy a latency requirement on how fast results are returned to users.
Facesim creates realistic animations of a human face from a model and time sequence of muscle movements, and must maintain a real-time frame rate.
The sixth and seventh applications are \app{dijkstra} and \app{sha}, from the ParMiBench benchmark suite~\cite{parmibench}.
Dijkstra\footnote{We use the parallelized multiple queue implementation provided with the ParMiBench benchmark suite.} computes single-source shortest paths in a graph.
SHA (Secure Hash Algorithm) is used for secure storage and transmission of data and must maintain response time to ensure timely communication.
The last application is \app{STREAM}~\cite{stream}, a synthetic benchmark for measuring sustainable memory bandwidth, representing a variety of memory-bound applications.

All benchmarks are modified as discussed in~\secref{poet-implementation}, adding Heartbeats and POET calls.
The code snippet shown in \lstref{poet-example} provides an example of application code, highlighting the POET function calls.

Adding POET to existing applications does not require many modifications to the original code.
The complete modification of an existing application requires nine function calls plus associated variable declarations, for a total of 14 lines of code.
The user provides a desired latency target via the Heartbeats API using the \variable{min\_heartrate} and \variable{max\_heartrate} variables, which represent a desired minimum and maximum speed in terms of jobs completed per second.
POET simply takes the average of these two values, meaning they can be the same.
Given $I(t)$ jobs in a window and a target latency $\tau$, the desired rates are easily computed as:
\begin{equation}
  min\_heartrate = max\_heartrate = \frac{I(t)}{\tau} 
  \label{eqn:latency-to-performance}
\end{equation}
As demonstrated below, Heartbeats initialization also accepts requests for minimum and maximum accuracy and power -- POET does not use these, so they can safely be set to any value.
When initializing POET, the user should specify the system's configurations, encoded in \variable{control\_states} and \variable{cpu\_states}.

\lstset{emph={%  
    poet_state, poet_init, poet_apply_control, poet_destroy%
    },emphstyle={\color{black}\bfseries\underbar}%
}%
\begin{lstlisting}[language=C,%
  caption={Example of POET application code.},%
  label={lst:poet-example}]%

// initialization
heartbeat_t* heart =
  heartbeat_acc_pow_init(window_size, buffer_depth, "heartbeat.log",
                         min_heartrate, max_heartrate, min_accuracy, max_accuracy,
                         1, hb_energy_impl_alloc(), min_power, max_power);
get_control_states(NULL, &control_states, &nstates);
get_cpu_states(NULL, &cpu_states, &nstates);
poet_state* state = poet_init(heart, nstates, control_states, cpu_states,
                              &apply_cpu_config, &get_current_cpu_state,
                              buffer_depth, "poet.log");
// execution of main loop
while(running) {
  heartbeat_acc(heart, count++, 1);
  poet_apply_control(state);
  doWork();
}
// cleanup
poet_destroy(state);
free(control_states);
free(cpu_states);
heartbeat_finish(heart);
\end{lstlisting}


\subsection{Application Inputs}
\label{sec:poet-inputs}


\tblref{poet-inputs} shows the inputs used for each of the applications.
All inputs used are packaged with the original benchmarks, with the exception of the x264 input which comes from a set of standard test sequences.
Recall that these applications were not originally designed to provide predictable timing.
We quantify this inherent unpredictability by measuring the latency of each job and computing the standard deviation and mean over all jobs in an application.
\figref{poet-variation} shows the ratio of standard deviation to mean for each application when running without POET.
The figure shows that our applications have a range of natural behavior from low variance (implying natural predictability, \eg blackscholes) to high variance (meaning that the application naturally has widely distributed latencies, \eg x264).
The variability in the applications is largely the same across platforms, indicating that it is a fundamental property of the applications and not the devices.

\begin{table}[t]
\small
\centering
\caption{Input and Configuration Details.}
\begin{tabular}{cccc}
  \textbf{Application} & \textbf{Input} & \textbf{Jobs} & \textbf{Window Size} \\
  \hline
  \hline
  blackscholes   & 1 million options              & 400 batches   & 20 \\
  bodytrack      & sequenceB                      & 261 frames    & 20 \\
  facesim        & Storytelling                   & 100 frames    & 20 \\
  ferret         & corel:lsh                      & 2,000 queries & 20 \\
  x264           & ducks\_take\_off               & 500 frames    & 20 \\
  dijkstra       & input\_small                   & 1,000 paths   & 20 \\
  sha            & in\_file(1-16)                 & 1,000 hashes  & 50 \\
  STREAM         & self-generated                 & 1,000 updates & 50 \\
  \hline
  \hline
\end{tabular}
\label{tbl:poet-inputs}
\end{table}

\begin{figure}[t]
  \centering
  \begin{tikzpicture}

\begin{groupplot}[
    group style={
        group name=plots,
        group size=1 by 1,
        xlabels at=edge bottom,
        xticklabels at=edge bottom,
        vertical sep=5pt
    },
% axis x line* = bottom,
xlabel near ticks,
major x tick style = transparent,
xlabel={},
height=3.5cm,
width=0.95\columnwidth,
xmin=0,
xmax=21,
enlargelimits=false,
tick align = outside,
tick style={white},
ytick=\empty,
xticklabel shift={0pt},
x tick label style={rotate=35, anchor=east, font=\scriptsize},
xtick={1,2,3,4,5,6,7,8,9,10,11,12,13,14,15,16,17,18,19,20},
xticklabels={
{blackscholes},
{bodytrack},
{facesim},
{ferret},
{fluidanimate},
{frequmine},
{raytrace},
{swaptions},
{vips},
{x264},
{canneal},
{dedup},
{streamcluster},
{STREAM},
{SWISH++},
{HOP},
{KMeans},
{KMeans-Fuzzy},
{ScalParC},
{SVM-RFE},
},
ymin=0,
ymax=1.0,
ytick={0,0.2,0.4,0.6,0.8,1.0},
yticklabels={0.0,0.2,0.4,0.6,0.8,1.0},
% legend cell align=left, 
% legend style={ column sep=1ex },
ymajorgrids,
grid style={dashed},
ylabel shift={0mm},
ylabel style={align=center},
]
\nextgroupplot[ylabel={\footnotesize Coeff. of Variation \\ \scriptsize (Std. Deviation/Mean)},
ybar=\pgflinewidth,
bar width=8pt,
]
\addplot table[x index=0,y index=2, col sep=tab] {img/copper/variability.txt};
\node at (axis cs:6.0,0.9) [rotate=35] {\scriptsize{$2.8$}};


\end{groupplot}

\end{tikzpicture}

  \caption{Application Latency Variability.}
  \label{fig:poet-variation}
\end{figure}

\section{Evaluation}
\label{sec:poet-evaluation}

The experimental evaluation of POET is divided into five parts.
First, we demonstrate POET's ability to meet the latency requirements.
Next, we quantify the energy consumption of the resulting system, then compare the energy of POET's general approach to one that controls latency and energy tradeoffs using just DVFS.
We then evaluate POET's ability to adapt to input with multiple phases, and finally, its ability to run subject to the interference of other applications.


\subsection{Meeting Latency Targets}
\label{sec:poet-eval-performance}

To test POET's ability to meet latency targets, we run each application $i$ in all possible configurations on both systems to determine the minimum average latency $m_i$.
For each application, we impose four latency targets.
The targets cover a wide range of achievable goals, from 25\% to 95\% of each system's performance capacity, \ie a 25\% goal means that the target is set to $4 \times m_i$.

We quantify POET's ability to meet the latency goals by measuring each job's latency and comparing it to the goal.
We then compute the Mean Absolute Percentage Error (MAPE), a standard metric in control theory to evaluate the behavior of controllers~\cite{ICSE2014}.
For an application composed of $n$ jobs:
\begin{equation}
MAPE = 100\% \cdot \frac{1}{n} \sum\limits_{i=1}^{n} 
\left \{
\begin{array}{ll}
d_m(i) > d_{r}  :& \left|\frac{d_m(i) - d_{r}}{d_r} \right| \\
d_m(i) \le d_{r}  :& 0
\end{array} \right.
\end{equation}
where $d_r$ is the specified latency requirement and $d_m(i)$ is the measured latency for the $i$-th job.
In other words, for each missed deadline we add a term that depends on the relative tardiness between the target and measured latency.

\begin{figure}[t]
  \centering
  \begin{tikzpicture}

\begin{groupplot}[
    group style={
        group name=plots,
        group size=1 by 2,
        xlabels at=edge bottom,
        xticklabels at=edge bottom,
        vertical sep=5pt
    },
%axis x line* = bottom,
xlabel near ticks,
major x tick style = transparent,
xlabel={},
height=3.5cm,
width=0.95\columnwidth,
xmin=0,
xmax=10,
enlargelimits=false,
tick align = outside,
tick style={white},
ytick=\empty,
ylabel style={align=center},
ymin=0,
ymax=8,
ytick={0,2,4,6,8},
yticklabels={,2,4,6,8},
legend cell align=left, 
legend style={ column sep=1ex },
ymajorgrids,
grid style={dashed},
]
\nextgroupplot[ybar=\pgflinewidth,
bar width=6pt,
ylabel={Vaio\\MAPE (\%)},
ylabel shift={0mm},
ymin=0,
ymax=8,
legend entries = {{25\%},{50\%},{75\%},{95\%}},
legend style={draw=none,legend columns=4,at={(0.5,1.35)},anchor=north,font=\footnotesize},
]
\addplot table[x index=0,y index=2, col sep=tab] {img/poet/mape-v.txt};
\addplot table[x index=0,y index=3, col sep=tab] {img/poet/mape-v.txt};
\addplot table[x index=0,y index=4, col sep=tab] {img/poet/mape-v.txt};
\addplot table[x index=0,y index=5, col sep=tab] {img/poet/mape-v.txt};

\nextgroupplot[ybar=\pgflinewidth,
bar width=6pt,
ylabel={ODROID\\MAPE (\%)},
ylabel shift={0mm},
ymin=0,
ymax=8,
xticklabel shift={0pt},
x tick label style={rotate=35, anchor=east, font=\scriptsize},
xtick={1,2,3,4,5,6,7,8,9,10},
xticklabels={
{blackscholes},
{bodytrack},
{facesim},
{ferret},
{x264},
{dijkstra},
{sha},
{STREAM},
{\textbf{Average}}},
]
\addplot table[x index=0,y index=2, col sep=tab] {img/poet/mape-o.txt};
\addplot table[x index=0,y index=3, col sep=tab] {img/poet/mape-o.txt};
\addplot table[x index=0,y index=4, col sep=tab] {img/poet/mape-o.txt};
\addplot table[x index=0,y index=5, col sep=tab] {img/poet/mape-o.txt};

\end{groupplot}
\end{tikzpicture}
  
  \caption{Latency error (lower is better, 0 is optimal).}
  \label{fig:poet-mape}
\end{figure}

\figref{poet-mape} presents the MAPE values for each application for the four latency targets on both the Vaio and the ODROID.
In general, the larger the variance in the application behavior, the larger the error.
This is not surprising since more volatile applications are harder to control.
However, the results indicate that MAPE values are generally low.
On the Vaio, the average MAPE for all applications is well below 2.5\% for all targets, typically closer to 1.5\%.
The ODROID presents similar results.
The MAPE metric is unforgiving since it penalizes every violation of the latency target, yet POET achieves low MAPE even for applications that were not inherently designed to support predictable timing.


\subsection{Energy Minimization}
\label{sec:poet-eval-energy}

This section evaluates POET's energy minimization strategy.
As discussed, we have measured latency and energy consumption for all applications in all configurations and therefore have perfect knowledge of each application's behavior.
We use this data to compute the minimal energy required for a latency target, \ie the energy consumed when we choose the best configuration for each job with foreknowledge of that job's needs and no overhead.
We quantify POET's energy consumption by comparing its achieved energy to this effective minimal energy.
Optimal energy is not attainable in practice as it would require knowledge of the future and no overhead.

\begin{figure}[t]
  \centering
  \begin{tikzpicture}

\begin{groupplot}[
    group style={
        group name=plots,
        group size=1 by 2,
        xlabels at=edge bottom,
        xticklabels at=edge bottom,
        vertical sep=15pt
    },
%axis x line* = bottom,
xlabel near ticks,
major x tick style = transparent,
xlabel={},
height=3.5cm,
width=0.95\columnwidth,
xmin=0,
xmax=10,
enlargelimits=false,
tick align = outside,
tick style={white},
ylabel style={align=center},
ytick=\empty,
legend cell align=left, 
legend style={ column sep=1ex },
ymajorgrids,
grid style={dashed},
]
\nextgroupplot[ybar=\pgflinewidth,
bar width=6pt,
ylabel={Vaio\\Energy},
ylabel shift={0mm},
ymin=1.0,
ymax=1.2,
ytick={1,1.05,1.1,1.15,1.2},
yticklabels={1.00,1.05,1.10,1.15,1.20},
legend entries = {{25\%},{50\%},{75\%},{95\%}},
legend style={draw=none,legend columns=4,at={(0.5,1.4)},anchor=north,font=\footnotesize},
]
\addplot table[x index=0,y index=2, col sep=tab] {img/poet/ee-v.txt};
\addplot table[x index=0,y index=3, col sep=tab] {img/poet/ee-v.txt};
\addplot table[x index=0,y index=4, col sep=tab] {img/poet/ee-v.txt};
\addplot table[x index=0,y index=5, col sep=tab] {img/poet/ee-v.txt};

\nextgroupplot[ybar=\pgflinewidth,
bar width=6pt,
ylabel shift={0mm},
ymin=1.0,
ymax=1.2,
ytick={1,1.05,1.1,1.15,1.2},
yticklabels={1.00,1.05,1.10,1.15,1.20},
ylabel={ODROID\\Energy},
ylabel shift={0mm},
xticklabel shift={0pt},
x tick label style={rotate=35, anchor=east, font=\scriptsize},
xtick={1,2,3,4,5,6,7,8,9,10},
xticklabels={
{blackscholes},
{bodytrack},
{facesim},
{ferret},
{x264},
{dijkstra},
{sha},
{STREAM},
{\textbf{Average}}},
]
\addplot table[x index=0,y index=2, col sep=tab] {img/poet/ee-o.txt};
\addplot table[x index=0,y index=3, col sep=tab] {img/poet/ee-o.txt};
\addplot table[x index=0,y index=4, col sep=tab] {img/poet/ee-o.txt};
\addplot table[x index=0,y index=5, col sep=tab] {img/poet/ee-o.txt};

\end{groupplot}
\end{tikzpicture}  
  \caption{Energy (lower is better, 1 is optimal).}
  \label{fig:poet-ee}
\end{figure}

For each application, we run POET for each latency target and record the achieved energy consumption.
We then compute the ratio of the energy consumption with POET to the effective minimal energy.
Unity represents minimal energy and values greater than 1 show energy consumption above the optimal.
\figref{poet-ee} presents the normalized energy data for each application on both the Vaio and the ODROID.
The data includes the overhead of POET's runtime, which consumes additional energy executing the control and optimization tasks.
On average across all applications and targets, POET's energy consumption exceeds optimal by 1.3\% on the Vaio and by 2.9\% on the ODROID.
These results demonstrate that POET achieves near-optimal energy consumption in practice.

The most troublesome test is \app{dijkstra} on the ODROID with a latency target of 75\%, which exceeds optimal by about 16\%.
The true optimal schedule just barely achieves this goal by varying the DVFS setting between 1.2 and 1.1 GHz.
Any overhead larger than 2\% requires a clockspeed of 1.3 GHz.
Unfortunately, this is in the area of steeply diminishing returns for the ODROID.
Compensating for this overhead almost entirely accounts for the energy difference between POET and optimal.
POET's runtime overhead is small, but non-zero, so POET uses the higher clockspeed.
POET's overhead is due in part to its generality; \ie its ability to handle multiple actuators that may affect energy and latency.
The next section highlights the benefits of this generality.


\subsection{Comparison with DVFS}
\label{sec:poet-eval-dvfs}

\begin{figure}[t]
  \centering
  \begin{tikzpicture}

\begin{groupplot}[
    group style={
        group name=plots,
        group size=1 by 2,
        xlabels at=edge bottom,
        xticklabels at=edge bottom,
        vertical sep=5pt
    },
% axis x line* = bottom,
xlabel near ticks,
major x tick style = transparent,
xlabel={},
height=3.5cm,
width=0.95\columnwidth,
xmin=0,
xmax=5,
enlargelimits=false,
tick align = outside,
tick style={white},
ylabel style={align=center},
ytick=\empty,
ymin=.98,
ymax=1.2,
ytick={1,1.05,1.1,1.15,1.2},
yticklabels={1.0,,1.1,,1.2},
legend cell align=left,
legend style={ column sep=1ex },
ymajorgrids,
grid style={dashed},
]
\nextgroupplot[ybar=\pgflinewidth,
ylabel={Vaio\\Energy\\(Normalized)},
bar width=8pt,
legend entries = {{DVFS-only},{POET}},
legend style={draw=none,legend columns=2,at={(0.5,1.4)},anchor=north,font=\footnotesize},
]
\addplot table[x index=0,y index=4, col sep=tab] {img/poet/dvfs-compare-short.txt};
\addplot table[x index=0,y index=5, col sep=tab] {img/poet/dvfs-compare-short.txt};

\nextgroupplot[ybar=\pgflinewidth,
bar width=8pt,
ylabel={ODROID\\Energy\\(Normalized)},
ylabel shift={0mm},
ymin=.98,
ymax=5.5,
ytick={1,1.5,2,2.5,3,3.5,4,4.5,5,5.5},
yticklabels={1.0,,2.0,,3.0,,4.0,,5.0,},
xticklabel shift={0pt},
x tick label style={rotate=35, anchor=east, font=\footnotesize},
xtick={1,2,3,4},
xticklabels={
{25\%},
{50\%},
{75\%},
{95\%}},
]
\addplot table[x index=0,y index=2, col sep=tab] {img/poet/dvfs-compare-short.txt};
\addplot table[x index=0,y index=3, col sep=tab] {img/poet/dvfs-compare-short.txt};

\end{groupplot}
\end{tikzpicture}
  \caption{Comparison of average energy consumption with DVFS-only versus POET (lower is better, 1 is optimal).}
  \label{fig:poet-dvfs-compare}
\end{figure}

Several energy management approaches have been proposed that optimally tune DVFS settings to meet timing constraints while reducing energy consumption \cite{Albers}.
In this section, we compare POET's energy consumption to an approach which only uses DVFS.
Specifically, we develop system configuration files that only specify changes in DVFS settings and deploy POET on both hardware platforms with these configurations.
We compare this \emph{DVFS-only} approach to POET's more general approach which coordinates multiple resource types and uses different resources on different platforms.

\figref{poet-dvfs-compare} summarizes the data comparing a DVFS-only approach to POET.
The charts show latency targets on the x-axes and energy consumption normalized to optimal on the y-axes (for POET, this is the same data shown in \figref{poet-ee}).
For each latency target, the figure shows the average (across all benchmarks) energy over optimal for both DVFS-only and POET on both platforms.
At the higher latency (lower performance, \eg 25\%) targets, POET saves substantial energy.
The energy savings are especially high on the ODROID as POET is able to take advantage of cluster migration and the low-power LITTLE cores, whereas a DVFS-only approach cannot exploit this feature.
This data clearly demonstrates that systems that are provisioned for a rarely seen worst case latency can greatly benefit from POET's generalized approach.
This is also further confirmation of prior studies showing that DVFS by itself is not optimal \cite{Hoffmann2012,MeisnerISCA2011}.

The exact energy savings compared to DVFS vary considerably for each application and latency target.
Therefore, we include more detailed charts in \appref{poet-dvfs-compare}.


\subsection{Responding to Application Phases}
\label{sec:poet-eval-phases}

In this test, we examine POET's ability to cope with input whose workload varies with time.
We execute the \app{x264} application using an input that is a combination of three videos of varying difficulty.
The input thus has three distinct phases, each composed of 500 jobs (frames).
\figref{poet-phases-default} shows time series data for both latency and power for the Vaio and the ODROID when they run without POET in their highest performing configurations.
Latency is normalized to the maximum latency measured for any iteration (\ie the empirically determined worst case).
We use this worst case result to derive the latency target for the POET tests.
Frames that need fewer resources to achieve the latency goal present opportunities to save energy.
We present the raw data for power; energy is the integral of those curves.

\begin{figure}[t]
  \centering
  \input{img/poet/x264-phases-default.tex}
  \caption{x264 processing of input with distinct phases.}
  \label{fig:poet-phases-default}
\end{figure}
\begin{figure}[t]
  \centering
  \begin{tikzpicture}

\definecolor{s1}{RGB}{228, 26, 28}
\definecolor{s2}{RGB}{55, 126, 184}
\definecolor{s3}{RGB}{77, 175, 74}
\definecolor{s4}{RGB}{152, 78, 163}
\definecolor{s5}{RGB}{255, 127, 0}

\begin{groupplot}[
    group style={
        group name=plots,
        group size=1 by 2,
        xlabels at=edge bottom,
        xticklabels at=edge bottom,
        vertical sep=15pt
    },
height=3.5cm,
width=0.95\columnwidth,
xmajorgrids,
ymajorgrids,
grid style={dashed},
xmin=0,
xmax=1500,
yticklabel pos=left,
enlargelimits=false,
tick align = outside,
tick style={white},
xticklabel shift={-5pt},
yticklabel shift={-5pt},
ylabel shift={-2pt},
ylabel style={align=center},
legend cell align=left,
legend style={ column sep=1ex },
unbounded coords=jump,
]

\nextgroupplot[ylabel={\footnotesize Latency \\ (Normalized)}, % Performance
xtick={0,100,200,300,400,500,600,700,800,900,1000,1100,1200,1300,1400,1500},
ytick={0.6,0.8,1.0,1.2,1.4},
yticklabels={0.6,0.8,1.0,1.2,1.4},
yticklabel style={font=\footnotesize},
ymin=.6,
ymax=1.4,
legend entries={{Vaio},{ODROID}},
legend style={draw=none,at={(0.5,1.4)},anchor=north,legend columns=4,line width=5pt,font=\footnotesize},
]
\addplot[thick, solid, color=s1] table[x index=0,y index=1,col sep=tab] {img/poet/x264-phases-poet-vaio.txt};
\addplot[thick, solid, color=s3] table[x index=0,y index=1,col sep=tab] {img/poet/x264-phases-poet-odroid.txt};
\addplot[thick, solid, black] coordinates {(0, 1) (1500, 1)};
\addplot[thick, dashed, black] coordinates {(500,0) (500, 1.4)};
\addplot[thick, dashed, black] coordinates {(1000,0) (1000, 1.4)};


\nextgroupplot[ylabel={\footnotesize Power \\ (Watts)}, % Power
ytick={2.0,4.0,6.0,8.0},
yticklabels={2.0,4.0,6.0,8.0},
yticklabel style={font=\footnotesize},
ymin=2,
ymax=8,
xlabel={\footnotesize $time$ [frame]},
xlabel near ticks,
xtick={0,100,200,300,400,500,600,700,800,900,1000,1100,1200,1300,1400,1500},
xticklabels={0,,,,,500,,,,,1000,,,,,1500},
xticklabel style={font=\footnotesize},
]
\addplot[thick, solid, color=s1] table[x index=0,y index=2,col sep=tab] {img/poet/x264-phases-poet-vaio.txt};
\addplot[thick, solid, color=s3] table[x index=0,y index=2,col sep=tab] {img/poet/x264-phases-poet-odroid.txt};
\addplot[thick, dashed, black] coordinates {(500,0) (500, 11)};
\addplot[thick, dashed, black] coordinates {(1000,0) (1000, 11)};

\end{groupplot}

\end{tikzpicture}    
  \caption{POET behavior for x264 input with distinct phases.}
  \label{fig:poet-phases-x264}
\end{figure}

The phases are clearly distinguishable by the change in latency at frames 500 and 1000.
The two systems do not process each phase with the same relative latency.
The first phase is the most difficult (highest latency) for both systems, but the second phase is the easiest (lowest latency) for the Vaio, while the third one is the easiest for the ODROID.

We enable POET using the maximum measured latency identified in the first experiment as the target.
The results of the executions are shown in \figref{poet-phases-x264}.
POET is able to meet the latency target on both systems.
Dips and spikes are visible at the beginning of each phase, showing the change in behavior of the input and POET adapting to the change.
Despite these variations, latency goals are respected: MAPE is 2.2\% on the Vaio and 2.0\% on the ODROID.
At the same time, energy is near minimal over the course of execution: energy is 1.7\% greater than optimal on the Vaio and 3.6\% over optimal on the ODROID.


\subsection{Adapting to Other Applications}
\label{sec:poet-eval-multiapp}

\begin{figure}[t]
  \centering
  \begin{tikzpicture}

\definecolor{s1}{RGB}{228, 26, 28}
\definecolor{s2}{RGB}{55, 126, 184}
\definecolor{s3}{RGB}{77, 175, 74}
\definecolor{s4}{RGB}{152, 78, 163}
\definecolor{s5}{RGB}{255, 127, 0}

\begin{groupplot}[
    group style={
        group name=plots,
        group size=1 by 2,
        xlabels at=edge bottom,
        xticklabels at=edge bottom,
        vertical sep=5pt
    },
height=3.5cm,
width=0.95\columnwidth,
xmajorgrids,
ymajorgrids,
grid style={dashed},
xmin=1,
xmax=250,
yticklabel pos=left,
enlargelimits=false,
tick align = outside,
tick style={white},
xticklabel shift={-5pt},
yticklabel shift={-5pt},
ylabel shift={-2pt},
ylabel style={align=center},
legend cell align=left,
legend style={ column sep=1ex },
unbounded coords=jump,
]

\nextgroupplot[ylabel={\footnotesize Vaio \\ Latency}, % Performance
ytick={0.4,0.6,0.8,1.0,1.2,1.4},
yticklabels={0.4,,0.8,,1.2,},
yticklabel style={font=\footnotesize},
ymin=.4,
ymax=1.4,
legend entries={{POET}, {static~allocation}},
legend style={draw=none,at={(0.5,1.4)},anchor=north,legend columns=4,line width=5pt,font=\footnotesize},
]
\addplot[thick, solid, color=s2] table[x index=0,y index=2,col sep=tab] {img/poet/multiapp.txt};
\addplot[thick, dashed, color=red] table[x index=0,y index=1,col sep=tab] {img/poet/multiapp.txt};
\addplot[thick,solid, color=black] coordinates {(133,0) (133,1.4)};
\addplot[thick,solid, color=black] coordinates {(0,1) (250,1)};


\nextgroupplot[ylabel={\footnotesize ODROID  \\ Latency}, 
ytick={0.4,0.6,0.8,1.0,1.2,1.4},
yticklabels={0.4,,0.8,,1.2,},
yticklabel style={font=\footnotesize},
ymin=.4,
ymax=1.4,
xlabel={$time$ [frame]},
xlabel near ticks,
xtick={1,50,100,150,200,250},
xticklabels={1,50,100,150,200,250},
xticklabel style={font=\footnotesize},
]
\addplot[thick,solid, color=black] coordinates {(133,0) (133,1.4)};
\addplot[thick,solid, color=black] coordinates {(0,1) (250,1)};
\addplot[thick, dashed, color=red] table[x index=0,y index=5,col sep=tab] {img/poet/multiapp.txt};
\addplot[thick, solid, color=s2] table[x index=0,y index=6,col sep=tab] {img/poet/multiapp.txt};

\end{groupplot}

\end{tikzpicture}
    
  \caption{POET running with another application.}
  \label{fig:poet-multiapp}
\end{figure}

This test shows POET's behavior when other applications are present in the system.
This external load is not under the direct control of POET.
We launch a POET-enabled application with a target latency.
Halfway through its execution, we launch another application.
This second application consumes resources, slowing down the POET-enabled application.
POET then assigns more resources to its own application so that it continues to meet its latency target.
We have tested this capability on all applications, but we only report the results for \app{bodytrack} due to space limitations.
The results for the other applications are similar.

\figref{poet-multiapp} shows POET's behavior, the top half displaying the Vaio and the bottom half the ODROID.
The thick vertical lines show when the second application was launched.
In both cases, we see the latency temporarily increase before POET adjusts the resource allocation.
To show the benefits of POET, the charts also show a statically managed system that just selects the resources to be given to the application at the beginning of its execution.
In the static case, the introduction of the new application dramatically increases the job latency.
To quantify this effect, we compute the MAPE metric for both the static allocation and with POET.
On the Vaio, POET's MAPE is 2.3\% over the entire execution (including the period of adjustment to the new load), while the static case has a MAPE of 16\%.
On the ODROID, POET's MAPE is 2.4\%, while the static case achieves 12\%.


\subsection{Discussion of Results and Limitations}

Our results show that POET achieves the goals of providing predictable timing and near minimal energy across multiple platforms.
These results are obtained despite the facts that 1) the tested applications were not originally designed to offer predictable latency and 2) the test platforms have completely different latency/energy tradeoffs.
The applications require only minimal modifications to run with POET, but no other changes are needed to exploit the different resources and latency/energy tradeoffs that different platforms offer.
In summary, POET achieves our design goal of enabling predictable timing with near-optimal energy in a portable library.
% The code for POET and the configurations used for the experiments are available to reproduce the results.

These results also demonstrate some limitations of the approach.
POET supports only soft real-time constraints.
The controller is guaranteed to converge to the desired latency and is provably robust to errors, but latency goals may be violated during the settling time, as seen in \figref{poet-multiapp} when POET adapts to the presence of the new application.
In addition, highly variable applications can still cause temporary latency violations before the control action settles again, as seen in \figref{poet-phases-x264} when controlling the high-variance x264 application.
This is further evidence that there is a tension between timeliness and energy reduction \cite{Abeni} -- the tremendous energy savings on the ODROID come at a cost of some latency errors compared to \emph{race to idle}.
%While the system is
%provably robust to errors, more accurate speedup estimates produce
%better results.


POET may be sensitive to the resource specifications provided by the user.
While the controller can tolerate large errors, in practice it is best to classify applications as compute or memory-bound and use one configuration set for each class of application.
POET's models do not currently account for the time required to switch between configurations.
Instead, this overhead is modeled as an inaccuracy in the specified speedup.
Our results show that this simplification works well in practice, but it may not be sufficient with different resources that have extremely long latencies.
In that case, the POET controller and optimizer should be extended to account explicitly for the overhead of switching configurations.
 
Finally, POET currently assumes that only one of the running applications (consisting of multiple, possibly communicating threads) should meet a deadline.
POET's Kalman filter guarantees that even when other applications are present in the system, the controller will compute the correct speedup to be applied, as demonstrated in \figref{poet-multiapp}.
However, future work could extend POET with a priority scheme allowing multiple POET-enabled applications to work concurrently.
In that scheme, high priority applications would be allocated the needed resources and lower priority applications would run in best-effort mode.



\section{Inspired Projects}
\label{sec:poet-inspired}

\TODO{Find appropriate place for this material, \eg an Appendix?}

POET has also been used as the foundation for other controllers and projects.
We expanded on POET to create Bard, which adds the ability to instead meet soft power constraints and maximize performance \cite{Bard}.
Farrell and Hoffmann build on POET by additionally changing application accuracy to provide hard real-time guarantees \cite{meantime}.
Mishra \etal combine POET with machine learning in a project called CALOREE to reduce the amount of offline work needed to generate resource specifications, and update both the specification and pole value online as new application behavior is learned \cite{CALOREE}.
POET was also the starting point in an ongoing project called Proteus, in which we are developing a programming language called FAST that allows programmers to specify more general constraints and optimizations called \emph{intents}, which uses both system and application knobs to satisfy.

Additionally, POET (and the work that motivated it \cite{Imes2014}) ultimately led to developing EnergyMon, a portable software interface for accessing energy metrics at runtime \cite{energymon}.
EnergyMon was integrated with Mozilla's Servo web browsing engine \cite{servo}.
It is also utilized in other research projects, like the aforementioned FAST language, and in the next project in this thesis, CoPPer.
