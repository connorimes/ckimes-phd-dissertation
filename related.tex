\chapter{Related Work}
\label{sec:related}

This chapter discusses work related to performance, power, and energy awareness.
We first discuss work relating to POET and CoPPer---meeting performance constraints and optimizing energy consumption.
% Research in this area is common in the embedded systems domain, though some work has explored approaches for systems at larger scales.
We then discuss power and energy-aware work in the High Performance Computing domain, where trading performance for power/energy savings is less common.
Finally, we describe research projects by other researchers that build off work in this dissertation.


\section{Timing Constraints and Energy Awareness}

Computing systems are often underutilized, leading to significant portions of time where application performance requirements can be met using less than the full system capacity \cite{google,MeisnerISCA2011}.
This trend has led to flourishing scheduling research that tailors resource usage to meet performance requirements, often while optimizing another dimension like power or energy consumption.
Modern systems expose a variety of configurable resources, which energy-aware schedulers can adjust to tune performance and power behavior.
This flexibility allows the system to adapt to different circumstances or different application needs, but also increases software complexity.
The problem is exacerbated when approaches must achieve portability across a range of different systems, which expose different resources to software and have unique performance/power tradeoff spaces.

% DVFS and core allocation
Software-based DVFS management is essential for many energy-aware scheduling algorithms \cite{Albers,YDS,Kumar2014}.
At the processor level, DVFS has been used to meet performance requirements \cite{Lu2002,Wu2004} and implement power capping \cite{lefurgy2008power}.
Allowing DVFS to be set separately on different cores provides further benefits \cite{Isci2006,Rangan2009}.
Similarly, managing DRAM DVFS increases energy efficiency \cite{MemScale,Diniz2007}.
Recent survey papers devote entire sections to the various ways DVFS has been used in scheduling systems \cite{MittalSurvey,ZhuralevSurvey}.
In modern multicore systems, allocating a subset of cores and sockets (with aggressive clock-gating to save power) is also common, and more recently comes with the additional challenge of scheduling for heterogeneous architectures \cite{Petrucci2012}.

% Naive approaches
One simple heuristic for minimizing energy is \emph{race-to-idle}, which allocates all resources until a job completes and then idles the system until the next job arrives \cite{google}.
This heuristic is portable since it does not require knowledge about the system, but empirical studies show that it is not optimal \cite{google,Cheng2009,Yang2007,Yun2010}.
A recent study by Kim \etal demonstrates that an optimal solution requires knowledge of how the different configurable resources in a system affect the specific application under control---information which \emph{race-to-idle} does not use \cite{kim-cpsna2015}.
The same study shows that \emph{race-to-idle} is dominated by a \emph{pace-to-idle} heuristic, \ie \emph{pace-to-idle} is theoretically never worse than \emph{race-to-idle} and can be much better.

% Multiple knobs
It is not surprising that a number of different frameworks have arisen for intelligently controlling multiple resources to minimize energy.
Many empirical studies have shown that it is more energy-efficient to coordinate multiple resources than to manage any one alone \cite{google,Cheng2009,Yang2007,Yun2010}.
Examples include systems that coordinate DVFS with core usage \cite{packandcap-old,TCST,packandcap-new}, coordination of processor and DRAM DVFS \cite{Chen2011,CoScale,Felter2005,Li2007}, and DVFS with thread scheduling \cite{Rangan2009,Winter2010}.
Several other approaches coordinate processor-wide DVFS with adaptations to the memory system and processor pipeline \cite{Bitirgen2008,dubach2010,METE}.
For example, Dubach \etal coordinate several microarchitectural features~\cite{dubach2010},
Petrucci \etal coordinate thread scheduling and the use of heterogeneous cores \cite{Petrucci2012}, while Maggio \etal coordinate core allocation and clock speed~\cite{TCST}.
Bertini \etal coordinate tiers of a multi-tier webserver for e-commerce \cite{Bertini2007}.
AbouGhazaleh \etal coordinate the speed of the processor and cache \cite{AbouGhazaleh2007}, while Yun \etal also coordinate the speeds of multiple on-chip components \cite{Yun2010}.
Liu \etal coordinate job scheduling and clock speed on clusters \cite{Liu2008}.
Bitirgen \etal coordinate clock speed, cache, and memory bandwidth in a multicore~\cite{Bitirgen2008}.
The METE system manages clock speed, memory bandwidth, and core usage \cite{METE}.
Sinangil \etal co-design a processor architecture which exposes both monitoring and configurable resources with an operating system that dynamically manages those resources \cite{sinangil2014self}.
All of these approaches coordinate multiple resources, but do so using system-specific implementations, \eg METE's controller would have to be redesigned to work with Bitirgen et al.'s architecture.

% General frameworks for multiple resources
Other frameworks have been proposed to meet real-time constraints by managing multiple resources.
These approaches are typically implemented as middleware that take a specification of available resources and a performance goal, and then meet that goal.
Rajkumar \etal propose a general framework (with system-specific implementation) for allocating resources to achieve real-time requirements, but this approach is not energy-aware \cite{Rajkumar}.
Sojka \etal propose a portable middleware layer for allocating resources to meet soft real-time constraints, but this system does not minimize energy \cite{Sojka}.
ControlWare is another middleware approach that uses control theory to meet quality-of-service constraints, but does not address energy concerns \cite{ControlWare}.

These approaches provide portable real-time guarantees, which is itself a hard problem, but they do not provide energy savings.
LEO is a machine learning system that can meet performance constraints with minimal energy consumption \cite{LEO}.
LEO is very accurate and provides high energy savings, even with no prior knowledge of the application currently running.
Its approach is extremely portable, but also incurs very high overhead.
Interestingly, LEO and POET have complementary weaknesses---POET has low runtime overhead, but requires prior knowledge in the form of a configuration model while LEO has high overhead, but requires no prior knowledge.
Follow-up work to LEO addressed this issue with CALOREE (discussed further in \secref{related-inspired}) \cite{CALOREE}.
PTRADE also uses control theory to manage general collections of resources \cite{PTRADE}, and is perhaps the most similar to POET.
PTRADE minimizes power consumption, but not necessarily energy.
In addition, PTRADE uses heuristic optimizations, which are not portable, while POET uses a true minimal-energy scheduling algorithm.

% Complementary approaches
We also note two approaches that are complementary to POET.
Zhao \etal manage processor speed to meet both reliability and timing constraints with minimal energy \cite{Zhao2012}.
This problem is complicated by the effect DVFS scaling has on hardware reliability.
It is possible that POET's general approach to resources other than DVFS might allow additional energy savings if incorporated in Zhao et al.'s work.
He \etal propose adaptive energy management in the power circuitry itself to meet timing constraints while adapting the delivered energy to increase battery efficiency \cite{He2013}.
It is possible that combining POET's runtime-level resource management with this supply-level approach would further increase efficiency.

% Automation / generalization
In the time since POET and CoPPer were developed, more recent works have focused on making it easier to create and use self-adaptive systems.
Filieri \etal discuss a general approach to the controller design process \cite{Filieri2017} while Maggio \etal describe automatic methods for generating multiple-input multiple-output (MIMO) controllers \cite{Maggio2017}.
Recent works leverage MIMO controllers to efficiently use system resources \cite{josep-isca2016,SPECTR} and could even be combined with approaches like CoPPer.
Furthermore, software-level knobs (instead of system-level) are becoming more common.
While the concept is not new, it still seems to be in its infancy---Shevtsov \etal perform a systematic literature review on software self-adaptation in order to outline current challenges and begin to generalize approaches \cite{Shevtsov2017}.
One such general approach is SmartConf, which automatically adjusts software knobs to meet performance-sensitive application configurations and provide other optimizations \cite{SmartConf}.


\section{High Performance Computing}

Power and energy in HPC systems is a growing concern, though prior work in the area has often not allowed trading performance for power or energy savings.
For example, Adagio uses DVFS to save energy with less than 1\% increase in runtime \cite{RountreeAdagio}.
Other work depends on accurate prediction of a code's critical path to reduce power where it will not slow down an application \cite{Jitter,Marathe2015}.
Patki \etal propose to better utilize available power with hardware over-provisioning to increase total system throughput \cite{PatkiRMAP}.
Sarood \etal have shown similar results: hardware over-provisioning increases performance given a power cap \cite{Sarood2013}.
Hardware over-provisioning acknowledges that compute resources are no longer the primary factor limiting cluster size---power is, allowing us to more aggressively trade performance and power/energy consumption.

Other works demonstrate that low-level hardware performance counters can drive solutions for modeling and improving power/energy consumption \cite{Libutti2014,Sasaki,Chetsa,WuHPCComputer}.
Using Dynamic Concurrency Throttling and DVFS to reduce energy consumption without performance loss, Curtis-Maury \etal use hardware events to create an energy-aware logistic regression model for predicting performance and power \cite{Curtis-Maury2008}.

As the number of system settings increases and their interaction becomes more complicated, several approaches have turned to machine learning to manage them.
Paragon \cite{Paragon} and Quasar \cite{quasar} guarantee quality-of-service constraints in heterogeneous data centers using a scheduler based on the learning system that won the Netflix prize \cite{NetflixPrize}.
LEO (mentioned previously) uses a Hierarchical Bayesian Model to minimize energy for different system utilization requirements \cite{LEO}.
These approaches all estimate the performance and power of every possible system configuration, then search those estimates to find the best configuration that meets their operating constraints.
The need to estimate every configuration's behavior is expensive, requiring half a second \cite{LEO} to several seconds \cite{Paragon} of overhead.
In contrast, our proposed approach simply returns the best system settings without predicting their actual energy efficiency, an approach that requires orders of magnitude less overhead (see \secref{classifiers-eval-overhead}).
Ferroni \etal use classification based on hardware events to select the best power model for an application from a predetermined set, and then assign resources to that application in a multi-tenant virtualized infrastructure \cite{FerroniTACO}.
This approach and our proposed approach are complementary, in that Ferroni \etal assign resources at the node-level, while our proposed approach can fine-tune resource usage within a node.

Our approach for maximizing energy efficiency is most closely related to other node-level approaches for managing performance and energy.
PUPiL maximizes node performance given a power cap by adjusting system settings to the particular needs of an application \cite{pupil}.
Chasapis \etal maximize performance for power-capped NUMA nodes by recognizing the effect that manufacturing variability can have on individual core performance \cite{Chasapis2016}.
Both of these approaches maximize performance for a given power constraint.
Neither, however, is capable of minimizing energy, which requires changing both power and performance as in our proposed approach.
ParallelismDial is a node-level approach for managing application-level parallelism to increase energy efficiency \cite{Sridharan2013}.
ParallelismDial has similar goals to our proposed approach, but they are complementary---it works at the application level, while our approach operates at the system level.
In future work, it would likely be beneficial to combine the two to further reduce overhead and improve energy savings.


\section{Inspired Projects}
\label{sec:related-inspired}

Research in this dissertation, and POET in particular, has been used as the foundation for projects by other researchers.

Farrell and Hoffmann build on POET to create MEANTIME, which provides hard real-time guarantees instead of soft performance guarantees \cite{meantime}.
They use the concept of approximate computing to tune application accuracy when their runtime determines that a performance constraint may not be met.
The approach usually only requires small changes to application accuracy, \eg peak signal-to-noise ratio and the bitrate of a video encoder, while still tuning system-level knobs like in POET to save energy.

Aforementioned work by Mishra \etal combines POET with machine learning in a project called CALOREE, which reduces the amount of offline work needed to generate the controller's resource specification \cite{CALOREE}.
CALOREE uses transfer learning, taking information from both previously seen applications and the current application, to model how interacting resources affect the current application's speedup.
It then makes online updates to the controller's resource specification and pole value using more accurate models.

POET and Bard were also the starting point in an ongoing project called Proteus.
Proteus is developing a programming language called FAST, which allows programmers to provide \emph{intent specifications} instructing the runtime of desired constraints and optimizations.
FAST satisfies these intents by tuning both application and system knobs, \eg to manage application performance, application accuracy, and system power or energy consumption.
